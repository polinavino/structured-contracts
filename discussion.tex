\section{Related work}
\label{sec:related}

Code errors and design flaws have been very costly for users
since the introduction of smart contracts on the Ethereum platform \cite{survey}.
Among the most high-profile and costly being the DAO hack \cite{DAO},
and more recently, a faulty NFT contract \cite{nftbug}. Formal methods are being
used to find, prevent, and mitigate vulnerabilities on different ledger models, and via different
approaches \cite{formal}.

Scilla \cite{scilla} is a intermediate-level language for writing smart contracts as state
machines on an account-based ledger model.
The Scilla authors have used Coq to reason about contracts written in Scilla, proving a variety of
temporal properties such as safety, liveness, and others. The goal of this work,
however, not to study stateful program behaviour, but rather, it is to formalize
the notion of correct implementations
of stateful programs on a platform where programs are inherently stateless. The purpose of the
properties we discuss is to exemplify how reasoning about a
specification trace guarantees the conclusions to hold for any
ledger representation inducing a correct its implementation. A full treatment
of lifting safety and liveness properties from specification to implementation
is the subject of future work. Additionally, we believe that the subsystem approach
we present in this work may be used to model on-chain interactions between
Scilla contracts.

The Bitcoin Modelling Language (BitML) \cite{bitml} allows the definition of smart
contracts running on Bitcoin by means of a restricted class of state machines.
The BitML state machines are less expressive than the class of specifications
considered in our model, since we assume that our stateless scripts are written
in a Turing-complete language. However, the goal of this language is similar to
ours - to guarantee that the behaviour of certain state machines (in the case of BitML, ones defined
using this language) if in accordance with the changes made by valid transactions, i.e. soundness.
The difference is that we present a framework in which one can define a state
transition system with an implementation (both using a Turing-complete language),
then prove soundness to achieve a "correct" implementation, whereas BitML allows users to
define a sound state machine from smaller class, then compile it to a \emph{specific}
implementation. With BitML, LTL formulas can be automatically
verified using a dedicated model checker. In the future, we plan to add support for LTL formulas in
our framework.

VeriSolid \cite{verisolid} synthesises Solidity smart contracts from a state machine specification, and
verifies temporal properties of the state machine using CTL. The underlying
ledger model for VeriSolid is, however, an account-based model, rather than the
EUTxO model we work with. Moreover, in contrast
to the VeriSolid approach, our approach
relies on the contract author to themselves to synthesize an implementation
that meets the requirements specific to the contract being built, and then
provides a proof obligation to show that implementation is correct. This allows
for more flexibility in the implementation, as well as in the logic
used in checking properties. Here, again, it may be of interest to use an approach
similar to the structured contracts framework to model interactions between
structured contracts.

CoSplit, presented in \cite{sharding}, is a static analysis tool for implementing
\emph{sharding} in an account-based blockchain. Sharding is the act of
separating contract state into smaller
fragments that can be affected by commuting operations, usually for the purposes of
increasing parallelism and scalability.
Our work allows users to compose contracts whose state is distributed across multiple
UTxOs and tokens on the ledger, and provides a way to formally guarantee that the update of the
full aggregated state is in accordance with the update of its ledger representation.
This application of structured contracts serves a similar purpose as for the EUTxO
ledger as sharding does for an account-based one, as it can be used to increase
parallelism and scalability. We note that the UTxO model is a natural fit for
such state separation, since one of the benefits of such a ledger is that
all operations either commute or fail \cite{parallelism}. Therefore, any
UTxO state representation,
with any (correct) implementation, will afford the relevant properties for a given
distributed contract.

On the EUTxO ledger, the constraint-emitting machine design pattern \cite{eutxoma} makes
a formal correctness guarantee
similar to the proof obligation we require as part of the definition of smart contracts.
However, it is limited to a ledger representation of contract state that is
strictly the datum and value of a
single UTxO entry, expressing dependencies on other scripts via a limited set of
possible constraints on
transactions. Our model allows the contract state to be computed from multiple
UTxO entries and tokens aggregated across the ledger state, with its evolution
coordinated by multiple different scripts. Another notable difference is that
the ledger model presented in \cite{eutxoma} is a list of UTxO-stype transactions,
rather than the UTxO set itself, with a unique initial state (the empty ledger).
Here, we are not able to review and reason about the full transaction history,
as is the case for existing realistic ledgers.

The K framework \cite{kframework} is a unifying formal semantics framework for
all programming languages, which has been used as a tool to perform
audits of smart contracts \cite{runtimev}. QuckCheck, a property-testing
library, has also been applied for the purposes of auditing
existing stateful EUTxO-implemented contracts \cite{quviq}. Other formal
methods audits of individual contracts acknowledge the complexity and uniqueness
of Turing-complete code in the EUTxO model \cite{tweag}. While formal in nature,
these individual audits and services do not present an overarching principle of
EUTxO smart contract verification in their approaches.

\section{Applications}
\label{sec:applications}

We have presented a framework which can be used to formalize not only the contracts
themselves, but also the
problems in the domain of smart contract verification.
We discuss several applications of this general framework
to existing contract verification issues.
We give examples of the problems that are most promising to be addressed
with this new tool. Each of these requires an in-depth investigation, which is
outside the scope of this paper :

\paragraph{Properties. } Program verification involves analyzing and providing guaranees
about the behaviour of program. In our case, these programs are structured
contracts, as well as the ledger itself. Trace-based properties, such as liveness
and safety properties,
are the standard for analyzing program behaviour \cite{liveness}.
Applying and/or
adjusting the definitions and theorems about properties to be used in the context of
structured contracts and ledgers will give high-assurance guarantees about
the behaviour of both \cite{properties}. Moreover, the subsystem relation in the SCF allows
for establishing a correspondence between ledger and contract properties via the
state projection function. This may give additional behaviour guarantees for
a contract's ledger representations.

\paragraph{Double satisfaction. } Double satisfaction is a situation in which
a single action performed by a transaction on the ledger satisfies the constraints
of more than one script being executed. This is a very broad, informal description,
and the situation it describes may not be a problem.
However, there are situations in which it would be - such as when a single payout
made by a transaction to a given address satisfies the payout requirements of
two distinct scripts run by the transaction.

So far, there has not been any formal treatment of exactly when double satisfaction
may be undesirable, even if it is a fairly intuitive answer in most cases.
The SCF may present a solution to making precise this distinction : it provides
a way to separate transaction \emph{constraint checking} from \emph{state updates}.
For example, one may include in a specification state the collection of pay-ins to be
consumed by the contract and pay-outs to be collected by the intended recipient.
This ensures that any correct implementation will mark such payments as made for or
by a \emph{specific contract instance}, thus mitigating problematic double satisfaction.

This approach of marking some data or assets as "for a specific contract, and from
a specific contract" is a
scheme that can be described as a kind of \emph{message-passing} \cite{messages}.
We can implement
it as and instance of the SCF. We can also use this same scheme for the following
challenge to tackle via the SCF:

\paragraph{Asynchronous or partial contract execution. } Asynchronous contract
execution refers to dependence between scripts implemented in a particular way.
It is achieved
by relaxing the requirement that dependent scripts must be executed within the same transaction,
in two distinct ones. We can allowing one script to execute a step,
simultaneously constructing a kind of proof artefact of its
execution on a given input. In the second transaction, which consumes the artefact,
the dependent script executes using the consumption of the artefact as proof of
the first script’s validation within a prior transaction.
This second script can behave as if the first contract was executed within the same transaction
with that specific input.
This type of scheme may be useful, for example, as a way to transfer assets
between contracts without having to run them both in a single transaction.

Another application of this scheme is allowing contracts whose code is "too large"
to be run in a single transaction to be split up into what is effectively
\emph{function calls}. Since message-passing records the contract that generated
a given message, as well as the input that contract was given, messages can
be used as artefacts of computations of function calls \cite{messages}.

\paragraph{Eliminating dependencies on opaque scripts. }
A script may include a constraint requiring another script to be run
within the same transaction, e.g. a particular token to be minted,
or some UTxO to be spent. Those scripts, may, in turn, contain additional
constraints requiring yet more scripts to be run. The SCF could allow us to
define ledger subsystems in which guarantees can be made about what scripts
will be required to run when the subsystem executes any step. While a good
implementation of a structured contract will intuitively not depend on any unnecessary
script executions, we may now be able to formalize this property.

Eliminating reliance on the validation of opaque scripts to advance a contract
state on the ledger is an important goal, especially in the context of proving
liveness properties such as liquidity. Achieving it is a step towards being
able to guarantee the existence of a valid ledger transaction corresponding to
each step in the specification of a contract state transition.


\section{Conclusion}
\label{sec:conclusion}

We have presented a novel approach to specifying and reasoning about behaviour of
stateful programs running on a EUTxO ledger, which we call the structured contract
formalism. Our formalism defines a robust
way to relate stateless predicate scripts executed at the ledger level to the
corresponding corresponding executions of a specific stateful program.
We used the well-established concept of a subsystem
to define this relation, and a small-step semantics style already in use in
existing systems (i.e. Cardano) for the specification of the ledger and contracts.
This work presents a broadly applicable and principled
way of reasoning about stateful programs on the EUTxO ledger.

This paper lays the groundwork for treating ledger-implemented programs as
formal subsystems of the ledger. This is done on a level that is very specific
to the details of the EUTxO ledger. In the future, we aim to generalize our
research to be applicable to other kinds of ledger transition systems.
We would also like to better align our findings with existing concepts in the
theory of simulation, concurrency, and distributed computation, in order to
apply the full gamut of results in those areas for studying stateful
programs on the ledger.

Another goal for future work is to
find a way to use the findings presented here to verify existing contracts.
This may be done by first constructing the ledger and transaction representation
projection functions, then building a state transition specification induced by
$\LEDGER$ for those projections. Properties of the resulting system can then
be studied.

A full mechanization in Agda of the results and definitions in this work is currently
under way, as it is a natural next step. We also intend to build a mechanization of this work
integrated with the Agda mechanization of a more sophisticated and realistic
ledger, the Cardano ledger. The Cardano
ledger is in a unique position to be amenable to the structured contract framework
approach to verification due to the existence of a mechanized small-steps specification
in Agda of the entire ledger (currently in development) \cite{agdaspec}.
Agda automation of proof generation for the subsystem proof obligations, such
as using an SMT-solver, will be a natural progression of the project.

% Another
% aspirational aspect of automation of this project is that of the translation
% from contract implementations in Agda to Plutus, the language used in the
% between Agda contract implementations and Plutus contracts
