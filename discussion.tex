\section{Related work}
\label{sec:related}

%cite algorand and contract automata
%message-passing


Scilla \cite{scilla} is a intermediate-level language for expressing smart contracts as state
machines on an account-based ledger model. It is formalized in Coq, and the
contracts written in it are amenable to formal verification. In our work
we pursue the same goal of building stateful contracts and formally studying
their behaviour. However, the contribution of this work is a framework for stateful
contract implementation on the EUTxO ledger.

CoSplit, presented in \cite{sharding}, is a static analysis tool for implementing
\emph{sharding} in an account-based blockchain. Sharding is the act of
separating contract state into smaller
fragments that can be affected by commuting operations, usually for the purposes of
increasing parallelism and scalability.
Our work allows users to build contracts whose state is distributed across multiple
UTxOs and tokens on the ledger. One of the benefits of an EUTxO ledger is that
transaction application commutes \cite{parallelism}. So, no additional work is
required to ensure commutativity when updating parts of a contract with
distributed state in multiple steps.

The Bitcoin Modelling Language (BitML) \cite{bitml} enables the definition of smart
contracts in a particular restricted class of state machines on the Bitcoin ledger.
The BitML state machines are less expressive than the class of specifications
considered in our model. The goal here is again is similar to
ours - to guarantee that the behaviour of certain state machines is in accordance
with the changes made by valid transactions, i.e. soundness. The approach
we present in this work is more general (applying to arbitrary Turing complete
contracts), but does not, at this time, support automation for
constructing implementations and verifying their properties.

VeriSolid \cite{verisolid} synthesises Solidity smart contracts from a state
machine specification, and
verifies temporal properties of the state machine using CTL. The underlying
ledger model for VeriSolid is an account-based model, rather than the
EUTxO model we work with. Like BitML, VeriSolid is less flexible in the types of
state machines that can be implemented, and how they can be implemented, but
offers more automation than structured contracts.

The K framework \cite{kframework} is a unifying formal semantics framework for
all programming languages, which has been used as a tool to perform
audits of smart contracts \cite{runtimev}, as well as specifying Solidity
operational semantics \cite{solor}. Auditing is a common approach
to smart contract verification \cite{quviq} \cite{tweag}. We present a framework
which, when instantiated, provides certain guarantees about the implemented
contracts.



% On the EUTxO ledger, the constraint-emitting machine design pattern \cite{eutxoma} makes
% a formal correctness guarantee
% similar to the proof obligation we require as part of the definition of smart contracts.
% However, it is limited to a ledger representation of contract state that is
% strictly the datum and value of a
% single UTxO entry, expressing dependencies on other scripts via a limited set of
% possible constraints on
% transactions. Our model allows the contract state to be computed from multiple
% UTxO entries and tokens aggregated across the ledger state, with its evolution
% coordinated by multiple different scripts. Another notable difference is that
% the ledger model presented in \cite{eutxoma} is a list of UTxO-stype transactions,
% rather than the UTxO set itself, with a unique initial state (the empty ledger).
% Here, we are not able to review and reason about the full transaction history,
% as is the case for existing realistic ledgers.



% \section{Applications}
% \label{sec:applications}
%
% We have presented a framework which can be used to formalize not only the contracts
% themselves, but also the
% problems in the domain of smart contract verification.
% We discuss several applications of this general framework
% to existing contract verification issues.
% We give examples of the problems that are most promising to be addressed
% with this new tool. Each of these requires an in-depth investigation, which is
% outside the scope of this paper :
%
% \paragraph{Properties. } Program verification involves analyzing and providing guaranees
% about the behaviour of program. In our case, these programs are structured
% contracts, as well as the ledger itself. Trace-based properties, such as liveness
% and safety properties,
% are the standard for analyzing program behaviour \cite{liveness}.
% Applying and/or
% adjusting the definitions and theorems about properties to be used in the context of
% structured contracts and ledgers will give high-assurance guarantees about
% the behaviour of both \cite{properties}. Moreover, the subsystem relation in the SCF allows
% for establishing a correspondence between ledger and contract properties via the
% state projection function. This may give additional behaviour guarantees for
% a contract's ledger representations.
%
% \paragraph{Double satisfaction. } Double satisfaction is a situation in which
% a single action performed by a transaction on the ledger satisfies the constraints
% of more than one script being executed. This is a very broad, informal description,
% and the situation it describes may not be a problem.
% However, there are situations in which it would be - such as when a single payout
% made by a transaction to a given address satisfies the payout requirements of
% two distinct scripts run by the transaction.
%
% So far, there has not been any formal treatment of exactly when double satisfaction
% may be undesirable, even if it is a fairly intuitive answer in most cases.
% The SCF may present a solution to making precise this distinction : it provides
% a way to separate transaction \emph{constraint checking} from \emph{state updates}.
% For example, one may include in a specification state the collection of pay-ins to be
% consumed by the contract and pay-outs to be collected by the intended recipient.
% This ensures that any correct implementation will mark such payments as made for or
% by a \emph{specific contract instance}, thus mitigating problematic double satisfaction.
%
% This approach of marking some data or assets as "for a specific contract, and from
% a specific contract" is a
% scheme that can be described as a kind of \emph{message-passing} \cite{messages}.
% We can implement
% it as and instance of the SCF. We can also use this same scheme for the following
% challenge to tackle via the SCF:
%
% \paragraph{Asynchronous or partial contract execution. } Asynchronous contract
% execution refers to dependence between scripts implemented in a particular way.
% It is achieved
% by relaxing the requirement that dependent scripts must be executed within the same transaction,
% in two distinct ones. We can allowing one script to execute a step,
% simultaneously constructing a kind of proof artefact of its
% execution on a given input. In the second transaction, which consumes the artefact,
% the dependent script executes using the consumption of the artefact as proof of
% the first script’s validation within a prior transaction.
% This second script can behave as if the first contract was executed within the same transaction
% with that specific input.
% This type of scheme may be useful, for example, as a way to transfer assets
% between contracts without having to run them both in a single transaction.
%
% Another application of this scheme is allowing contracts whose code is "too large"
% to be run in a single transaction to be split up into what is effectively
% \emph{function calls}. Since message-passing records the contract that generated
% a given message, as well as the input that contract was given, messages can
% be used as artefacts of computations of function calls \cite{messages}.
%
% \paragraph{Eliminating dependencies on opaque scripts. }
% A script may include a constraint requiring another script to be run
% within the same transaction, e.g. a particular token to be minted,
% or some UTxO to be spent. Those scripts, may, in turn, contain additional
% constraints requiring yet more scripts to be run. The SCF could allow us to
% define ledger subsystems in which guarantees can be made about what scripts
% will be required to run when the subsystem executes any step. While a good
% implementation of a structured contract will intuitively not depend on any unnecessary
% script executions, we may now be able to formalize this property.
%
% Eliminating reliance on the validation of opaque scripts to advance a contract
% state on the ledger is an important goal, especially in the context of proving
% liveness properties such as liquidity. Achieving it is a step towards being
% able to guarantee the existence of a valid ledger transaction corresponding to
% each step in the specification of a contract state transition.


\section{Conclusion}
\label{sec:conclusion}

% Code errors and design flaws have been very costly for users
% since the introduction of smart contracts on the Ethereum platform \cite{survey}.
% Among the most high-profile and costly being the DAO hack \cite{DAO},
% and more recently, a faulty NFT contract \cite{nftbug}. Formal methods are being
% used to find, prevent, and mitigate vulnerabilities on different ledger models, and via different
% approaches \cite{formal}.
%
% We note that the EUTxO model guarantees transaction
% commutativity \cite{properties}, making it a natural fit for building contracts
% with state distributed across different parts of the ledger. These state components
% be updated independently with predictable outcomes, regardless of the order in
% which transactions are processed.

The key contribution of this work is a new, versatile, and principled approach to
modelling stateful contracts on the EUTxO ledger. We generalize the application of
\emph{simulation} for demonstrating integrity of consolidated (single-UTxO) state \cite{eutxoma}
to simulation of arbitrarily implemented state machines with varying ledger
representations.

We do this by introducing a formalism for modelling stateful contracts, which
we call the structured contract framework.
It is instantiated by first specifying a labelled transition system expressed in terms of
small-step semantics. Then, functions must be defined that compute the contract state
and input for a given ledger state and transaction, respectively. The functions include
information about the implementing scripts used to control evolution of the relevant ledger
data. Finally, a proof
obligation of the integrity of the implementation must be fulfilled for the given specification
and projection functions.

This approach opens up the possibility of formal verification of the behaviour
a much larger class of contracts, which would previously have been implemented ad-hoc.
We presented examples of such contracts and safety properties satisfied by these examples.
These examples include a distributed implementation of a contract, and a stateful
model of an NFT policy.
A full analysis of structured contract behaviour in terms of trace-based
properties and their expression at the ledger level is the subject of future work.

An Agda mechanization of the SCF has been completed \cite{agda-structured}, and
the mechanization of the examples we presented in under way. The completion of
these examples, as well as of more realistic ones, is a natural next step.
We also intend to build a mechanization of this work
integrated with the Agda mechanization of small-steps semantics specification
of the deployed Cardano ledger \cite{cardano-agda}.


% to
%
%
% system specification of the contract,
% We discuss how the SCF can be
% used to formalize and address existing challenges in the field of smart contract
% verification in the EUTxO model.
%
% We have presented a novel approach to specifying and reasoning about behaviour of
% stateful programs running on a EUTxO ledger, which we call the structured contract
% formalism. Our formalism defines a robust
% way to relate stateless predicate scripts executed at the ledger level to the
% corresponding corresponding executions of a specific stateful program.
% We used the well-established concept of a subsystem
% to define this relation, and a small-step semantics style already in use in
% existing systems (i.e. Cardano) for the specification of the ledger and contracts.
% This work presents a broadly applicable and principled
% way of reasoning about stateful programs on the EUTxO ledger.
%
% This paper lays the groundwork for treating ledger-implemented programs as
% formal subsystems of the ledger. This is done on a level that is very specific
% to the details of the EUTxO ledger. In the future, we aim to generalize our
% research to be applicable to other kinds of ledger transition systems.
% We would also like to better align our findings with existing concepts in the
% theory of simulation, concurrency, and distributed computation, in order to
% apply the full gamut of results in those areas for studying stateful
% programs on the ledger.
%
% Another goal for future work is to
% find a way to use the findings presented here to verify existing contracts.
% This may be done by first constructing the ledger and transaction representation
% projection functions, then building a state transition specification induced by
% $\LEDGER$ for those projections. Properties of the resulting system can then
% be studied.



% Another
% aspirational aspect of automation of this project is that of the translation
% from contract implementations in Agda to Plutus, the language used in the
% between Agda contract implementations and Plutus contracts
