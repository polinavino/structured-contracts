\section{Introduction}
\label{sec:intro}

Many modern cryptocurrency blockchains are smart contract-enabled, meaning, generally, that
they provide some support for executing user-defined code as part of block or transaction processing.
This code is used to specify agreements between untrusted parties that can be
automatically enforced without a trusted intermediary. Examples of such
contracts may include distributed exchanges (DEXs), escrow contracts, auctions,
etc.

There is a lot of variation
in the details of how smart contract support is implemented across different platforms.
In particular, on account-based platforms such as Ethereum \cite{ethereum}
and Tezos \cite{tezos}, smart contracts are inherently stateful, and their states
can be updated by transactions. Smart contracts in
the extended UTxO (EUTxO) model, such as Cardano \cite{alonzo} and
\cite{ergo}, on the other hand, take the form of boolean predicates
on the data of the transaction executing them, and are inherently stateless.
In this model, transactions specify all the changes being done to the ledger
state, while contract predicates are used only to specify permissions
for performing the UTxO set updates specified by the transaction, such as spending
UTxOs or minting tokens.

Just within the UTxO model, there are multiple existing approaches to implementing and formalizing specific
designs of stateful programs running on the ledger \cite{eutxoma} \cite{marlowe} \cite{hydra}.
There are also languages, such as BitML \cite{bitml}, a DSL for specifying
contracts that regulate transfers of Bitcoin among participants.
There is also much existing work on smart contract formal verification.
However, currently, there are no principled standard practices for
specifying and implementing stateful programs on the EUTxO ledger.
In this work, we propose to re-use an existing ledger specification standard
to specify stateful contracts.

Like many prominent platforms
\cite{tezos} \cite{ethereum} \cite{Nakamoto} \cite{nervos} \cite{zil},
the Cardano implementation of the EUTxO ledger \cite{alonzo} is specified as a
transition system. The reason for this
design choice is that
the evolution of the ledger takes place in atomic steps corresponding to
the application of a single transaction. What sets the Cardano specification apart,
however, is the formal rigour of its operational
small-step semantics specification \cite{steps}.
We propose the \emph{structured contract framework}
(SCF) as an extension of this approach to specification.
It enables users
to instantiate a small-step program specification that runs on the ledger
via the use of smart contract scripts.

Generalizing the constraint-emitting machine design pattern \cite{eutxoma},
the SCF formalizes the notion of stateful program running on the EUTxO ledger,
and what it means for it to be \emph{implemented correctly}. We do so by
requiring instantiation of a stateful program to include a proof of a \emph{simulation relation }
between its specification and the ledger specification.
Our generalization allows expressing invariants (or safety properties) of contracts
for which it was not previously done. For example, we can express invariants of stateful
contracts that are implemented across multiple different UTxOs. We can also
express invariants on the totality of tokens under a specific policy by interpreting
it as the state of a structured contract. %invariants of the ledger itself? POV
We argue that the SCF constitutes a new, principled approach to stateful
smart contract architecture that is amenable to formal analysis and suitable for
a wide range of smart contract applications.
The main contributions of this paper are :

\begin{itemize}
  \item[(i)] a definition of a simplified EUTxO ledger using small-steps semantics ;

  \item[(ii)] a definition of the structured contract formalism (SCF) ;

  \item[(iii)] a case study expressing the
  minting policy of a single NFT as a structured contract ;

  \item[(iv)] a case study demonstrating the use the SCF to define
  two distinct ledger implementations of a
  single specification, including one that is distributed across multiple
  UTxO entries and interacting scripts ;

\end{itemize}
