\section{Introduction}
\label{sec:intro}

Many modern cryptocurrency blockchains are smart contract-enabled, meaning, generally, that
they provide some support for executing user-defined code as part of block or transaction processing.
This code is used to specify agreements between untrusted parties that can be
automatically enforced without a trusted intermediary. Examples of such
contracts may include distributed exchanges (DEXs), escrow contracts, auctions,
etc.

There is a lot of variation
in the details of how smart contract support is implemented across different platforms.
In particular, on account-based platforms such as Ethereum \cite{ethereum}
and Tezos \cite{tezos}, smart contracts are inherently stateful, and their states
can be updated by transactions. Smart contracts in
the EUTxO model such as \cite{alonzo}, on the other hand, take the form of boolean predicates
on the data of the transaction executing them, and are inherently stateless.
In this model, transactions specify all the changes being done to the ledger
state, while contract predicates are used only to specify permissions
for performing the UTxO set updates specified by the transaction, such as spending
UTxOs or minting tokens.

Just within the UTxO model, are multiple existing approaches to implementing and formalizing specific
designs of stateful programs running on the ledger \cite{eutxoma} \cite{marlowe} \cite{hydra}.
There are also languages, such as BitML \cite{bitml}, a DSL for specifying
contracts that regulate transfers of Bitcoin among participants.
There is also much existing work on smart contract formal verification.
However, currently, there are no principled standard practices for using this smart contract
model for specifying and implementing stateful programs on the EUTxO ledger.
In this work, we propose to re-use the existing specification standard for the
ledger model to specify stateful contracts.

Like many prominent platforms
\cite{tezos} \cite{ethereum} \cite{Nakamoto} \cite{nervos} \cite{zil},
the EUTxO ledger is specified as a state transition system. It is given in terms of its operational
small-steps semantics describing the application of blocks and
transactions to a given ledger state \cite{alonzo}. The reason for this
design chice is that
the evolution of the ledger takes place in atomic steps corresponding to
transaction applications. We propose the \emph{structured contract framework}
(SCF) as an extension on the ledger semantics formalization.
It enables the users
to instantiate a small-steps program specification as a \emph{subsystem of the ledger}
via the use of smart contract scripts.

Permission scripts in the EUTxO model
control the change of the part of the ledger state for which they specify
permissions. Structured contracts make precise what it means to "control a part
of the ledger state" by specifying the evolution of that part as stateful
program, which happens in accordance with the permissions of the relevant scripts.
Generalizing the subsystem design pattern proposed in \cite{eutxoma},
the SCF formalizes the notion of stateful program running on the EUTxO ledger,
and what it means for it to be \emph{implemented correctly}. Moreover, unlike the
existing design pattern,
our generalization allows a stateful contract
to be distributed across multiple UTxOs and tokens, implemented via a suite of scripts.
We note that the EUTxO model guarantees transaction
commutativity \cite{properties}, making it a natural fit for building contracts
with state distributed across different parts of the ledger. These state components
be updated independently with predictable outcomes, regardless of the order in
which transactions are processed.

We present several examples to showcase the versatility of our approach to
contract specification and verification.
We discuss how the SCF can be
used to formalize and address existing challenges in the field of smart contract
verification in the EUTxO model.
We argue that the SCF constitutes a new, principled standard approach to stateful
smart contract architecture. It allows users to specify and implement
behaviour of all ledger subsystems programmable via smart contract script predicates.
The main contributions of this paper are :

\begin{itemize}
  \item[(i)] definition of a simplified EUTxO ledger via a small-steps semantics
  formalism ;

  \item[(ii)] definition of the structured contract formalism (SCF) as a ledger subsystem,
  consisting of a stateful program specification together with a proof obligation  ;

  \item[(iii)] a case study demonstrating the use of the SCF to specify and prove an EUTxO
  ledger property (the preservation of value), with a related example of expressing a minting policy as a
  structured contract ;

  \item[(iv)] a case study demonstrating the use the SCF to define and
  prove prove the correctness of two distinct ledger implementations of a
  single specification, including one that is distributed across multiple
  UTxO entries and interacting scripts ;

\end{itemize}
