\section{EUTxO ledger model}
\label{sec:ledger-rules}

The extended UTxO (EUTxO) ledger model is UTxO-based ledger model that supports the use of
user-defined Turing-complete scripts to specify conditions for spending (consuming)
UTxO entries as well as token minting and burning policies.
The EUTxO model can be expressed as a labelled transition system, which we define
in this section.

The state of the system is a UTxO set, and the transition labels are transactions.
As not every transaction is a valid transition for every UTxO set, we additionally
formulate the constraints which a transaction must satisfy to be valid in a given
UTxO set. Note that while in a realistic system, transactions are applied to the
ledger in blocks, block structure and block-specific ledger state data is secondary
to the discussion of smart contracts in this work.

\subsection{Transition Relation Specification}

To give the small-steps semantics which we use to express the transition
rules and types, we follow the set-theory based notation
outlined in \cite{steps}, and used in \cite{alonzo} \cite{structured}.
Non-standard notation used here is given in \ref{fig:notation:nonstandard}.

We denote a transition relation $\mathsf{TRANS}$ as a 4-tuple :

  \begin{equation*}
    \_ \vdash
    \var{\_} \trans{trans}{\_} \var{\_}
    \subseteq (\Env \times \State \times \Input \times \State)
  \end{equation*}

Membership $(\var{env},~s,~\var{inp},~s') \in \mathsf{TRANS}$ is also denoted by

  \begin{equation*}
    env \vdash
    \var{s} \trans{trans}{inp} \var{s'}
  \end{equation*}

Given a 4-tuple $(\var{env},~s,~i,~s')$, the role of each component
specifying the transition is as follows :

\begin{itemize}
  \item[(i)] $\var{env} \in \Env$ is the fixed \emph{environment} of the state transition
  update, e.g. the current slot number ;

  \item[(ii)] $s \in \State$ is the starting state
  to which an input is applied, e.g. the UTxO set, or a contract state ;

  \item[(iii)] $i \in \Input$ is the user input, e.g. a transaction, or input to
  a contract ;

  \item[(iv)] $s' \in \State$ is the end state
  obtained from the start state as the result of the application of the input,
  e.g. the updated UTxO state.
\end{itemize}

A specification $\TRANS$ is made up of one or more
\emph{transition rules}. That is, the only 4-tuples that are members of $\TRANS$ are those that
satisfy the preconditions in one of its transition rules.

\paragraph{Input vs. Environment. }

Note that there is no formal distinction between the use of input and environment.
However, making this distinction creates a useful separation between user-issued
input (e.g. a transaction), and block-level data. For example, slot numbers are
determined via blocks and a timekeeping mechanism at the consensus level,
not specified by users. We adopt this convention from the
EUTxO transition system specification \cite{alonzo}.


\subsection{Ledger Types}

The ledger model we use is an extended UTxO model with native multi-assets,
as introduced in \cite{structured}. It is based on
a previously introduced model
\cite{eutxoma}. We give an overview of the EUTxO state and transaction structure
below, and present the full set of type definitions
in Figures~\ref{fig:eutxo-types} and~\ref{fig:ctx-types}.
% We note that in the previously presented EUTxO models, unhashed scripts and
% transactions were never stored on-chain.

\paragraph{Scripts. } A $\Script$, or a \emph{smart contract},
is a piece of stateless user-defined code with a
boolean output. It is used to specify conditions under which a transaction is allowed
to perform a specific action. It is executed
whenever a transaction attempts to perform the action for which it specifies the
permissions. We do not specify the language in which scripts are written,
but we presume Turing-completeness.

The input to a script consists of (i) a summary of transaction data,
(ii) a pointer to the specific action (within the transaction) for which the script
is specifying the permission (e.g. the policy ID or the input),
(iii) and a piece of user-defined data we call a $\Redeemer$. A redeemer is defined
at the time of transaction construction (by the transaction author) for each action
requiring a script to be
run. We denote evaluation for scripts that control
minting of tokens as well as spending by $\applyMPScript{\_}$, followed by the
script arguments.

\paragraph{Value. } The type $\Value = \FinSup{\Policy}{\FinSup{\TokenName}{\Quantity}}$
represents bundles of multiple kinds of assets.
It uses a nested finite map data structure to associate a $\Quantity = \Z$
to every $\AssetID = \Policy \times \TokenName$. That is, the bundle contains a
non-zero integral quantity of assets with the asset IDs specified
in the domain of the map, and zero quantity of assets with all other IDs.
A token bundle containing no tokens is denoted by $0$. $\Value$ forms a group for
which we denote group the operation by $+$, as well as a partial order \cite{utxoma}.
An $\AssetID$
uniquely identifies a class of assets, wherein all assets are fungible. It is made up of :

\begin{itemize}
  \item[(i)] a $\Policy$, which is a script that is executed
  whenever a transaction
  mints assets under this policy to determine whether it is allowed to do so ;
  \item[(ii)] a $\TokenName$, which is some $\Data$ a user selects to differentiate between
  assets under the same policy. For readability, this differs from the $\TokenName$ type in \cite{eutxoma},
  where it is defined to be character string.
\end{itemize}

We construct a $\Value$ with a single asset, whose asset ID is given by
\[ \fun{oneT}~\fun{policy}~\fun{tokenName}~ \leteq~ \{~\fun{policy}~\mapsto~\{ \fun{tokenName} \mapsto 1 \}~\} \]

\paragraph{UTxO set. } The UTxO set
is a finite map $\UTxO = \OutputRef \mapsto \Output$.
The $\OutputRef = \Tx \times \Ix$ is the type of the key of the UTxO finite map, with $\Ix = \N$.
An output reference $(\var{tx}, \var{ix})$ is made up of (i) a transaction $\var{tx}$ that
created the output to which it points, and (ii) index of $\var{ix}$,
pointing to the place of a particular output in the list of outputs
of that transaction. This uniquely specifies an
output in a specific transaction.

An output $(a, v, d) \in \Output$ is made up of (i) an address $a$, which is a $\Script$,
and it specifies under
what conditions a given output can be spent by a transaction, (ii) an asset bundle $v$ locked
by this script, and (iii) an additional piece of data $d \in \Datum$, specified
by the transaction that adds this output to the UTxO.

\paragraph{$\Data$. } Is a type for encoding data that can be passed as arguments
to scripts, similar in structure to a CBOR encoding.
For details, see \cite{agdaspeceutxo}. It is also used in \cite{eutxoma},
and in the Cardano implementation \cite{cardano}.
The types $\Datum$ (data stored alongside value in an output) and
$\Redeemer$ (data specified by the user for spending a specific UTxO entry)
are both synonyms for the $\Data$ type. For each script, conversion functions
for both the datum and redeemer are defined to decode and encode those arguments
to and from the specific argument types expected by the script. We usually call the decoding
function $\fun{fromData}$, and the encoding one $\fun{toData}$ when the context
is unambiguous.

\paragraph{Transactions. } $\Tx$ is a data structure that specifies
the updates to be done to the UTxO set. A transaction $\var{tx} \in \Tx$
contains (i) a set of \emph{inputs} each referencing entries in the UTxO set that the transaction
is removing (spending), with their corresponding redeemers, (ii) a set of outputs, which get entered into the
UTxO set with the appropriately generated output references, (iii) a pair of slot
numbers representing the validity interval of the transaction, (iv) a $\Value$ being
minted by the transaction, (v) a redeemer for each
of the minting policies being executed, and (vi) the set of (public) keys that signed the
transaction, together with their signatures.

\paragraph{$\Slot$ number. }  A slot number is a natural number used to represent
the time at which a transaction is processed, with $\Slot = \N$.

\subsection{Ledger Transition Semantics}
\label{sec:ledgersem}

The type of the ledger transition system $\mathsf{LEDGER}$, that updates the UTxO
set by applying a given transaction in a specific environment (slot), is
denoted as follows :

  \begin{equation*}
    \_ \vdash
    \var{\_} \trans{ledger}{\_} \var{\_}
    \subseteq (\Slot \times \UTxO \times \Tx \times \UTxO)
  \end{equation*}

We introduce the function $\fun{checkTx}$, which
consists of the constraints that must be satisfied by a given tuple
in order to constitute a valid ledger update. This function performs the following
checks, specified in Figure \ref{fig:validity}, and consistent with the ledger rules
specified in \cite{eutxoma} \cite{structured} :

\begin{itemize}
  \item[(i)] The transaction has at least one input
  \item[(ii)] The current slot is within transaction validity interval
  \item[(iii)] All outputs have positive values
  \item[(iv)] All output references of transaction inputs exist in the UTxO
  \item[(v)] Value is preserved
  \item[(vi)] No output is double-spent
  \item[(vii)] All inputs validate
  \item[(viii)] Minting redeemers are present
  \item[(ix)] All minting scripts validate
  \item[(x)] All signatures are present
\end{itemize}

We use the function $\fun{checkTx}$ to define membership in the
$\mathsf{LEDGER}$ relation, and call this rule $\fun{ApplyTx}$.
More specifically, $\fun{ApplyTx}$ states that
$(\var{slot},~\var{utxo},~\var{tx},~\var{utxo'}) \in \mathsf{LEDGER}$
whenever $\fun{checkTx}~(\var{slot},~\var{utxo},~\var{tx})$ holds and $\var{utxo'}$
is given by $(\{~i\mapsto o \in \var{utxo} ~\mid~ i \notin \fun{getORefs}~{tx}~\}) \cup \fun{mkOuts}~{tx} $.

% \begin{figure}[htb]
  \begin{equation}
  \label{fig:ledger-rule}
    \inference[ApplyTx]
    {
    \var{utxo'}~\leteq~(\{~i\mapsto o \in \var{utxo} ~\mid~ i \notin \fun{getORefs}~{tx}~\}) \cup \fun{mkOuts}~{tx}
    \\ ~ \\
    \fun{checkTx}~(\var{slot},~\var{utxo},~\var{tx})
    \\ ~ \\
    }
    {
    \begin{array}{l}
      \var{slot}\\
    \end{array}
      \vdash
      \left(
      \begin{array}{r}
        \var{utxo} \\
      \end{array}
      \right)
      \trans{ledger}{tx}
      \left(
      \begin{array}{r}
        \varUpdate{\var{utxo'}}  \\
      \end{array}
      \right) \\
    }
  \end{equation}

The value $\var{utxo'}$ is calculated by removing the UTxO entries in $\var{utxo}$
corresponding to the output references of the transaction inputs, and adding the outputs of the
transaction to the UTxO set with correctly generated output references.
The functions used to compute the updated UTxO set are defined in \ref{fig:ctx-types}.

Unlike the
ledger model in \cite{eutxoma}, which contains a full list of all transactions in
the order they have been applied to the initial \emph{empty} ledger,
$\LEDGER$ does not specify an initial ledger state. For this reason, we
sometimes have to make additional assumptions about transactions.

In later sections,
when necessary, we assume certain conditions on
the starting state of the ledger trace to be true, such as the existence of a specific
entry in the UTxO set. Note also that $\LEDGER$
has no trivial transitions due to the fact that transactions without
inputs are not allowed by rule \ref{rule:has-input}, so the UTxO set is necessarily updated
by the transaction by (at minimum) removing its inputs.
An arbitrary labelled transition $\mathsf{TRANS}$ need not be deterministic,
and may allow multiple distinct end states for a given initial state, environment,
and input. However, the $\mathsf{LEDGER}$ system is \emph{deterministic}, as the
end state variable $\var{utxo'}$ is uniquely specified by in its only rule, $\mathsf{ApplyTx}$.
