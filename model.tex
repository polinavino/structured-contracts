\section{EUTxO ledger model}
\label{sec:ledger-rules}

The EUTxO ledger model is UTxO-based ledger model that supports the use of
user-defined Turing-complete scripts to specify conditions for spending (consuming)
UTxO entries as well as token minting and burning policies.
The EUTxO model has been previously expressed in terms of a ledger state containing a list of
transactions that have been validated, and a set of rules for validating incoming
transactions \cite{eutxoma}. We demonstrate here that it can be expressed as a labelled transition
system with the UTxO set as its state, specified in small-step semantics \cite{steps},
similar to the specification of a deployed EUTxO ledger \cite{alonzo}.
The transaction validation rules become constraints of the UTxO state transition
rule. Note that while in a realistic system, transactions are applied to the
ledger in blocks, we abstract away block structure here for simplicity.

\subsection{Transition Relation Specification}

For some $\Env, \State$ and $\Input$, the transition relation $\mathsf{TRANS}$
is a subset of a 4-tuple :
  \begin{equation*}
    \_ \vdash
    \var{\_} \trans{trans}{\_} \var{\_}
    \subseteq (\Env \times \State \times \Input \times \State)
  \end{equation*}

Membership $(\var{env},~s,~\var{inp},~s') \in \mathsf{TRANS}$ is also denoted by
  \begin{equation*}
    env \vdash
    \var{s} \trans{trans}{inp} \var{s'}
  \end{equation*}

where each component serves the following purpose in the state transition :

\begin{itemize}
  \item[(i)] $\var{env} \in \Env$ is the \emph{environment} ;

  \item[(ii)] $s \in \State$ is the \emph{starting state}
  to which an input is applied ;

  \item[(iii)] $i \in \Input$ is the \emph{input} ;

  \item[(iv)] $s' \in \State$ is the \emph{end state}
  computed from the start state as the result of the application of the input
  in the given environment.
\end{itemize}

A specification $\TRANS$ is made up of one or more
\emph{transition rules}. The only 4-tuples that are members of $\TRANS$ are those that
satisfy the preconditions in one of its transition rules.

\textbf{Input, environment, and labelled transition systems. }
The input and the environment together are used to calculate the possible
end state(s) for a given start state, making up the \emph{label} of the
transition between the start and end states. If the transition system is
deterministic, there is exactly one end state for a given start state and label.
We adopt the conventional distinction between environment and input due to
its usefulness in the blockchain context \cite{shelley}.
In particular, input comes from users, e.g. a transactions they submit.
The environment, on the other hand, is outside the user's control, such as
the blockchain time. Timekeeping is done at the consensus and block level,
which is outside the scope of this work.

\subsection{Ledger Types}
The ledger types and rules that we base this work on are, for the most part, similar to those
presented in existing EUTxO ledger research \cite{eutxoma}. We make some simplifications
in order to remove details not relevant to this work. We give an overview
of these for completeness, and clarify any omitted types
in Figures~\ref{fig:eutxo-types}~\ref{fig:ctx-types}.
Notation we use that is outside conventional set-theoretic notation is listed
in Figure \ref{fig:notation:nonstandard}, and explained in the text.
Here, $\B, \N, \Z$ denote the type of Booleans, natural numbers, and integers, respectively.
Some types described below are mutually recursive, so there is no natural
order in which to describe them.

% We note that in the previously presented EUTxO models, unhashed scripts and
% transactions were never stored on-chain.

\textbf{Value. } We define $\Value \leteq \FinSup{\Policy}{\FinSup{\TokenName}{\Quantity}}$.
A term of this type is a bundle of multiple kinds of assets.
The type of an identifier of a class of fungible assets is given by
$\AssetID \leteq \Policy \times \TokenName$.
$\Quantity \leteq \Z$ is an integral value.
For a given $v \in \Value$, the nested map associates a quantity of each asset
with a given asset ID to its asset ID. The quantities of all assets with IDs not
included in $v$ are $0$. A group is formed by $\Value$,
with the group operation denoted by $+$, and the empty map as the zero of the group.
$\Value$ also forms a partial order \cite{utxoma}.
The components of $\AssetID$ are :

\begin{itemize}
  \item[(i)] a script of type $\Policy \leteq \Script$, which executed any time a transaction is minting
  assets with this minting policy ;
  \item[(ii)] a $\TokenName \leteq [\type{Char}]$, selected by the user, and used to differentiate
  assets that have the same minting policy.
\end{itemize}

To define a $\Value$ with a single asset ID and quantity one, we use the following
function,
\[ \fun{oneT}~\fun{policy}~\fun{tokenName}~ \leteq~ \{~\fun{policy}~\mapsto~\{ \fun{tokenName} \mapsto 1 \}~\} \]

\textbf{UTxO set. } The type of the UTxO set is a finite map $\UTxO \leteq \OutputRef \mapsto \Output$.
The type of the key of this key-value map is $\OutputRef = \Tx \times \Ix$, with $\Ix = \N$.
A $(\var{tx}, \var{ix}) \in \OutputRef$ is called an output reference. It consists of a
transaction $\var{tx}$ that
created the output to which it points, and index of $\var{ix}$.
The index is the location of particular output in the list of outputs
of that transaction. The pair uniquely identifies a transaction
output.

An output $(a, v, d) \in \Output$ consists of (i) a script address $a \in \Script$,
which is run when the output is spent, (ii) an asset bundle $v \in \Value$ , and
(iii) a datum $d \in \Datum$, which is some additional data.

\textbf{$\Data$. } $\Data$ is a type used for representing data encoded in a specific
way, is similar in structure to a CBOR encoding, for details, see \cite{agdaspeceutxo}.
It was introduced in the definition of
the EUTxO ledger \cite{eutxoma}, and also used in its Cardano implementation \cite{cardano}.
Data of this type is passed as arguments to scripts.
The types $\Datum \leteq \Data$ and $\Redeemer \leteq \Data$
are both synonyms for the $\Data$ type. Conversion functions are required in order
for a script to interpret $\Data$-type inputs as the datatypes it is expecting.
When the context is clear, the decoding
function is called $\fun{fromData}$, and the encoding one is $\fun{toData}$.

\textbf{$\Slot$ number. }  A slot number $s \in \Slot \leteq \N$ is a natural
number used to represent the time at which a transaction is processed.

\textbf{Transactions. } The data structure $\Tx$ specifies
a set of updates to the UTxO set. A transaction $\var{tx} \in \Tx$
contains (i) a set of \emph{inputs} each referencing entries in the UTxO set that the transaction
is removing (spending), with their corresponding redeemers, (ii) a set of outputs, which get entered into the
UTxO set with the appropriately generated output references, (iii) a pair of slot
numbers representing the validity interval of the transaction, (iv) a $\Value$ being
minted by the transaction, (v) a redeemer for each
of the minting policies being executed, and (vi) the set of (public) keys that signed the
transaction, paired with their signatures.

\textbf{Scripts. } A smart contract, or \emph{script},
is a piece of stateless user-defined code with a
boolean output, and has the (opaque) type $\Script$. Scripts are associated
with performing a specific
action, such as spending an output, or minting assets. If a transaction
attempts to perform an action associated with a script, that script is executed
during transaction validation, and must return $\True$ (validate) given certain
inputs. A script specifies the
conditions a transaction must satisfy in order to be allowed to perform the
associated action. We do not specify the language in which scripts are written,
but we presume Turing-completeness. We use set-theoretic notation for script
pseudocode.

The input to a script consists of (i) a summary of transaction data,
(ii) a pointer to the specific action (within the transaction) for which the script
is specifying the permission,
(iii) and a piece of user-defined data we call a $\Redeemer$. A redeemer is defined
at the time of transaction construction (by the transaction author) for each action
requiring a script to be
run.

We denote evaluation of scripts that control
minting tokens and spending outputs by $\applyMPScript{\_}$, followed by the
script arguments.
For example, evaluating a minting policy script $s$ to validate
minting tokens under policy $p$, run by transaction $\var{tx}$ with redeemer $r$, is denoted by
$\applyMPScript{s}~r~(\var{tx},~p)$
Evaluating a script $q$ to validate spending input $i \in \inputs~\var{tx}$ with
datum $d$ and redeemer $r$, we write $\applyMPScript{q}~d~r~(\var{tx},~i)$

\subsection{Ledger Transition Semantics}
\label{sec:ledgersem}

Permissible updates to the UTxO set are given by the $\mathsf{LEDGER}$ transition
system,
  \begin{equation*}
    \_ \vdash
    \var{\_} \trans{ledger}{\_} \var{\_}
    \subseteq (\Slot \times \UTxO \times \Tx \times \UTxO)
  \end{equation*}

The output of the function $\fun{checkTx} : \Slot \times \UTxO \times \Tx \to \B$
determines whether a transaction is valid
in a given state and environment. The output of $\fun{checkTx}~(\var{slot}, \var{utxo}, \var{tx})$
is given by the conjunction of the following checks, which are consistent with the previously
specified EUTxO validation rules \cite{eutxoma} :

\begin{itemize}
  \item[(i)] The transaction has at least one input : $\inputs~\var{tx}~\neq~\{\}$
  \item[(ii)] The current slot is within transaction validity interval : $\var{slot} \in \fun{validityInterval}~\var{tx}$
  \item[(iii)] All outputs have positive values : $\forall o \in \outputs~\var{tx},~\val~\var{o} > 0$
  \item[(iv)] All output references of transaction inputs exist in the UTxO : \\
  $\forall (\var{oRef},~\var{o})~\in~\{(\outputref~\var{i},~\fun{output}~\var{i}) ~|~ i \in \inputs~\var{tx} \},
  ~\var{oRef}~ \mapsto~\var{o}~ \in ~\var{utxo} $
  \item[(v)] Value is preserved : $\mint~\var{tx} + \sum_{i \in \inputs~\var{tx},~(\outputref~i)\mapsto~o~\in~\var{utxo}} \val~\var{o} = \sum_{o \in \outputs~\var{tx}} \val~\var{o}$
  \item[(vi)] No output is double-spent : $\forall~i_1, i \in \inputs~\var{tx},~\outputref~\var{i} = \outputref~i_1 ~\Rightarrow~ i = i_1$
  \item[(vii)] All inputs validate : $\forall~ (i, o, r) \in \inputs~\var{tx},~ \applyScript{\validator~\var{o}}(\datum~\var{o},~r,~ (\var{tx}, (i, o, r))) $
  \item[(viii)] Minting redeemers are present : $\forall ~\var{pid}\mapsto \wcard~ \in (\mint~\var{tx}), ~~\exists
  (\var{pid},\wcard) ~\in~ \mintScsRdmrs~\var{tx}$
  \item[(ix)] All minting scripts validate : $\forall~ (p, \var{r}) \in \mintScsRdmrs~\var{tx},~ \applyMPScript{p}(\var{r}, (tx, p)) $
  \item[(x)] All signatures are present : $\forall~ (pk \mapsto s) \in \sigs~\var{tx},~ \checkSig(\var{tx}, pk, s) $
\end{itemize}

Membership in the $\LEDGER$ set is defined using $\fun{checkTx}$.
The single rule defining $\LEDGER$, called $\fun{ApplyTx}$, states that
$(\var{slot},~\var{utxo},~\var{tx},~\var{utxo'}) \in \mathsf{LEDGER}$
whenever $\fun{checkTx}~(\var{slot},~\var{utxo},~\var{tx})$ holds and $\var{utxo'}$
is given by $(\{~i\mapsto o \in \var{utxo} ~\mid~ i \notin \fun{getORefs}~{tx}~\}) \cup \fun{mkOuts}~{tx} $.
% \begin{figure}[htb]
  \begin{equation}
  \label{fig:ledger-rule}
    \inference[ApplyTx]
    {
    \var{utxo'}~\leteq~(\{~i\mapsto o \in \var{utxo} ~\mid~ i \notin \fun{getORefs}~{tx}~\}) \cup \fun{mkOuts}~{tx}
    \\ ~ \\
    \fun{checkTx}~(\var{slot},~\var{utxo},~\var{tx})
    \\ ~ \\
    }
    {
    \begin{array}{l}
      \var{slot}\\
    \end{array}
      \vdash
      \left(
      \begin{array}{r}
        \var{utxo} \\
      \end{array}
      \right)
      \trans{ledger}{tx}
      \left(
      \begin{array}{r}
        \varUpdate{\var{utxo'}}  \\
      \end{array}
      \right) \\
    }
  \end{equation}

The value $\var{utxo'}$ is calculated (deterministically) by removing the UTxO entries in $\var{utxo}$
corresponding to the output references of the transaction inputs, and adding the outputs of the
transaction $\var{tx}$ to the UTxO set with correctly generated output references.
The function $\fun{getORefs}$ computes a UTxO set containing only the output
references of transaction inputs, paired with the outputs contained in those inputs.
The function $\fun{mkOuts}$ computes a UTxO set containing exactly the outputs of
$\var{tx}$, each associated to the key $(\var{tx}, \var{ix})$. The index
$\var{ix}$ is the place of the associated output in the list of $\var{tx}$ outputs.
For details, see \ref{fig:ctx-types}.

Unlike the \emph{empty} ledger state, which is an initial ledger state in existing
EUTxO definitions \cite{eutxoma}, the ledger model we define by
$\LEDGER$ does not have an initial ledger state. For this reason, we
make additional assumptions about transactions, which could
be verified as properties or valid ledger traces starting from some valid initial state.
This is the subject of future work.
