
\documentclass[a4paper,UKenglish,cleveref, autoref, thm-restate, authorcolumns]{lipics-v2021}

\usepackage[T1]{fontenc}
% T1 fonts will be used to generate the final print and online PDFs,
% so please use T1 fonts in your manuscript whenever possible.
% Other font encondings may result in incorrect characters.
%
\usepackage{graphicx}
% Used for displaying a sample figure. If possible, figure files should
% be included in EPS format.
%
% If you use the hyperref package, please uncomment the following two lines
% to display URLs in blue roman font according to Springer's eBook style:
%\usepackage{color}
%\renewcommand\UrlFont{\color{blue}\rmfamily}
% correct bad hyphenation here
\hyphenation{}

%\usepackage{natbib}
\usepackage{url}


% *** MATHS PACKAGES ***
%
\usepackage{iohk}
%\usepackage[cmex10]{amsmath}
\usepackage{amssymb}
\usepackage{stmaryrd}
%\usepackage{amsthm}

% \usepackage[margin=2.5cm]{geometry}
\usepackage{microtype}
\usepackage{mathpazo} % nice fonts
\usepackage{amsmath}
\usepackage{amssymb}
%\usepackage{amsthm}
\usepackage{latexsym}
\usepackage{mathtools}
\usepackage{stmaryrd}
\usepackage{extarrows}
\usepackage{slashed}
%\usepackage[unicode=true,pdftex,pdfa,colorlinks=true]{hyperref}
%\usepackage{xcolor}
%\usepackage[capitalise,noabbrev,nameinlink]{cleveref}
%\usepackage{float}

% *** ALIGNMENT PACKAGES ***
%
\usepackage{array}
\usepackage{float}  %% Try to improve placement of figures.  Doesn't work well with subcaption package.
\usepackage{subcaption}
\usepackage{caption}

\usepackage{subfiles}
% \usepackage{geometry}
\usepackage{listings}
 \usepackage{xcolor}
\usepackage{verbatim}
\usepackage{listings}% http://ctan.org/pkg/listings
\lstset{
  basicstyle=\ttfamily,
  mathescape
}
\usepackage{alltt}
\usepackage{paralist}

% inference rules
\usepackage{semantic}

\usepackage{todonotes}

%
\newcommand{\todochak}[1]{\todo[inline,color=purple!40,author=chak]{#1}}
\newcommand{\todompj}[1]{\todo[inline,color=yellow!40,author=Michael]{#1}}
\newcommand{\todokwxm}[1]{\todo[inline,color=blue!20,author=kwxm]{#1}}
\newcommand{\todopv}[1]{\todo[inline,color=purple!40,author=polina]{#1}}

\newcommand{\red}[1]{\textcolor{red}{#1}}
\newcommand{\redfootnote}[1]{\red{\footnote{\red{#1}}}}
\newcommand{\blue}[1]{\textcolor{blue}{#1}}
\newcommand{\bluefootnote}[1]{\blue{\footnote{\blue{#1}}}}

%% A version of ^{\prime} for use in text mode
\makeatletter
\DeclareTextCommand{\textprime}{\encodingdefault}{%
  \mbox{$\m@th'\kern-\scriptspace$}%
}
\makeatother

\newcommand{\code}{\texttt}
\renewcommand{\i}{\textit}  % Just to speed up typing: replace these in the final version
\renewcommand{\t}{\texttt}  % Just to speed up typing: replace these in the final version
\newcommand{\s}{\textsf}  % Just to speed up typing: replace these in the final version
\newcommand{\msf}[1]{\ensuremath{\mathsf{#1}}}
\newcommand{\mi}[1]{\ensuremath{\mathit{#1}}}

%% A figure with rules above and below.
\newcommand\rfskip{3pt}
%\newenvironment{ruledfigure}[1]{\begin{figure}[#1]\hrule\vspace{\rfskip}}{\vspace{\rfskip}\hrule\end{figure}}
\newenvironment{ruledfigure}[1]{\begin{figure}[#1]}{\end{figure}}

%% Various text macros
\newcommand{\true}{\type{True}}
\newcommand{\false}{\type{False}}

\newcommand{\hash}[1]{\ensuremath{#1^{\#}}}

\newcommand\mapsTo{\ensuremath{\mapsto}}
\newcommand\cL{\ensuremath{\{}}
\newcommand\cR{\ensuremath{\}}}

\newcommand{\List}[1]{\ensuremath{\s{List}[#1]}}
\newcommand{\Set}[1]{\ensuremath{\mathbb{P}~#1}}  %{\ensuremath{\s{Set}[#1]}}
\newcommand{\FinSet}[1]{\ensuremath{\s{FinSet}[#1]}}
\newcommand{\Interval}[1]{\ensuremath{\s{Interval}[#1]}}
\newcommand{\FinSup}[2]{\ensuremath{\s{FinSup}[#1,\linebreak[0]#2]}}
% ^ \linebeak is to avoid a bad line break when we talk about finite
% maps.  We may be able to remove it in the final version.
\newcommand{\supp}{\msf{supp}}

\newcommand{\Script}{\ensuremath{\s{Script}}}
\newcommand{\scriptAddr}{\fun{scriptAddr}}
\newcommand{\ctx}{\ensuremath{\s{Context}}}
\newcommand{\vlctx}{\ensuremath{\s{ValidatorContext}}}
\newcommand{\mpsctx}{\ensuremath{\s{PolicyContext}}}
\newcommand{\toData}{\ensuremath{\s{toData}}}
\newcommand{\toTxData}{\ensuremath{\s{toTxData}}}
\newcommand{\fromData}{\msf{fromData}}

\newcommand{\emptymap}{\ensuremath{\{\}}}
\newcommand{\emptytype}{\star}
\newcommand{\emptytypeT}{\{\star\}}
\newcommand{\verify}{\msf{verify}}

\newcommand{\mkContext}{\ensuremath{\s{mkContext}}}
\newcommand{\mkTxInfo}{\ensuremath{\s{mkTxInfo}}}
\newcommand{\mkVlContext}{\ensuremath{\s{mkValidatorContext}}}
\newcommand{\mkMpsContext}{\ensuremath{\s{mkPolicyContext}}}
\newcommand{\checkSig}{\ensuremath{\s{checkSig}}}

\newcommand{\applyScript}[1]{\ensuremath{\llbracket#1\rrbracket}}
\newcommand{\applyMPScript}[1]{\ensuremath{\llbracket#1\rrbracket}}

\newcommand{\unionoverrideMinus}{\ensuremath{\mathbin{\cup_{-}}}}
% wildcard parameter
\newcommand{\wcard}[0]{\underline{\phantom{a}}}

% Macros for eutxo things.
\newcommand{\tx}{\fun{tx}}
\newcommand{\TxId}{\ensuremath{\s{TxId}}}
\newcommand{\TxInfo}{\ensuremath{\s{TxInfo}}}
\newcommand{\txId}{\msf{txId}}
\newcommand{\txrefid}{\fun{id}}
\newcommand{\Address}{\ensuremath{\s{Address}}}
\newcommand{\DataHash}{\ensuremath{\s{DataHash}}}
\newcommand{\hashData}{\fun{dataHash}}
\newcommand{\idx}{\fun{index}}
\newcommand{\inputs}{\fun{txins}}
\newcommand{\outputs}{\fun{outs}}
\newcommand{\Out}{\type{Output}}
\newcommand{\validityInterval}{\fun{validityInterval}}
\newcommand{\scripts}{\fun{scripts}}
\newcommand{\mint}{\fun{mint}}
\newcommand{\mintScripts}{\fun{mintScripts}}
\newcommand{\mintScsRdmrs}{\fun{mintScsRdmrs}}
\newcommand{\mintRdmrs}{\fun{mintRdmrs}}
\newcommand{\sigs}{\fun{sigs}}
\newcommand{\fee}{\fun{fee}}
\newcommand{\addr}{\fun{addr}}
\newcommand{\pubkey}{\fun{pubkey}}
\newcommand{\val}{\fun{value}}  %% \value is already defined
\newcommand{\Value}{\type{Value}}
\newcommand{\Redeemer}{\type{Redeemer}}
\newcommand{\TxOutRef}{\type{TxIn}}
\newcommand{\TxOut}{\type{TxOut}}
\newcommand{\ScriptContext}{\type{ScriptContext}}
\newcommand{\ScriptPurpose}{\type{ScriptPurpose}}
\newcommand{\Datum}{\type{Datum}}
\newcommand{\DCert}{\type{DCert}}
\newcommand{\LCTx}{\type{LCTx}}
\newcommand{\TxInInfo}{\type{TxInInfo}}

\newcommand{\FState}{\type{FState}}
\newcommand{\FInput}{\type{FInput}}
\newcommand{\Send}{\type{Send}}
\newcommand{\Receive}{\type{Receive}}
\newcommand{\SimplInput}{\type{SimplInput}}
\newcommand{\AccMsgs}{\type{AccMsgs}}
\newcommand{\Context}{\type{Context}}

\newcommand{\validator}{\fun{validator}}
\newcommand{\redeemer}{\fun{redeemer}}
\newcommand{\datum}{\fun{datum}}
\newcommand{\datumHash}{\fun{datumHash}}
\newcommand{\datumWits}{\fun{datumWitnesses}}
\newcommand{\Data}{\ensuremath{\s{Data}}}
\newcommand{\Input}{\ensuremath{\s{Input}}}
\newcommand{\Output}{\type{Output}}
\newcommand{\OutputRef}{\fun{OutputRef}}
\newcommand{\Signature}{\ensuremath{\s{Signature}}}
\newcommand{\Ledger}{\ensuremath{\s{Ledger}}}

\newcommand{\outputref}{\fun{outputRef}}
\newcommand{\outputrefs}{\fun{outputRefs}}
\newcommand{\txin}{\fun{in}}
\newcommand{\id}{\fun{id}}
\newcommand{\lookupTx}{\msf{lookupTx}}
\newcommand{\getSpent}{\msf{getSpentOutput}}

% \newcommand{\Tick}{\ensuremath{\s{Tick}}}
\newcommand{\Tick}{\type{Slot}}
\newcommand{\currentTick}{\msf{currentTick}}
\newcommand{\spent}{\msf{spentOutputs}}
\newcommand{\unspent}{\msf{unspentOutputs}}
\newcommand{\txunspent}{\msf{unspentTxOutputs}}
\newcommand{\eutxotx}{\msf{Tx}}


\newcommand{\consumes}[1]{\msf{consumes(#1)}}
\newcommand{\consumesOne}[1]{\msf{consumesOne(#1)}}
\newcommand{\cid}{\fun{cid}}
\newcommand{\inputValue}{\fun{inputValue}}
\newcommand{\rMin}{r_{\fun{min}}}
\newcommand{\rMax}{r_{\fun{max}}}

\newcommand{\utxotx}{\msf{Tx}}

\newcommand{\Quantity}{\ensuremath{\s{Quantity}}}
\newcommand{\Asset}{\ensuremath{\s{Asset}}}
\newcommand{\Policy}{\ensuremath{\s{PolicyID}}}
\newcommand{\Quantities}{\ensuremath{\s{Quantities}}}
\newcommand{\nativeCur}{\ensuremath{\mathrm{nativeC}}}
\newcommand{\nativeTok}{\ensuremath{\mathrm{nativeT}}}
\newcommand{\valC}{\mkValidator{\mathcal{C}}}
\newcommand{\polC}{\mkPolicy{\mathcal{C}}}
\newcommand\mkValidator[1]{\msf{validator}_#1}
\newcommand\mkPolicy[1]{\msf{policy}_#1}

\newcommand{\PublicKey}{\ensuremath{\s{PubKey}}}
\newcommand{\PubKey}{\ensuremath{\s{PubKey}}}
\newcommand{\PrivateKey}{\ensuremath{\s{PrivateKey}}}

\newcommand{\pkey}{\ensuremath{\pi_{\mathsf{p}}}}
\newcommand{\skey}{\ensuremath{\pi_{\mathsf{s}}}}

\newcommand\B{\ensuremath{\mathbb{B}}}
\newcommand\N{\ensuremath{\mathbb{N}}}
\newcommand\Z{\ensuremath{\mathbb{Z}}}
\renewcommand\H{\ensuremath{\mathbb{H}}}
%% \H is usually the Hungarian double acute accent
\newcommand{\emptyBs}{\ensuremath{\emptyset}}
\newcommand{\leteq}{\ensuremath{\mathrel{\mathop:}=}}
\newcommand{\Nt}{\ensuremath{\Diamond}}
\newcommand{\Bool}{\type{Bool}}
\newcommand{\Type}{\type{Type}}
\newcommand{\STRUC}{\type{STRUC}}
\newcommand{\BASIC}{\type{BASIC}}
\newcommand{\LEDGER}{\type{LEDGER}}
\newcommand{\LChanges}{\type{LChanges}}
\newcommand{\LVal}{\type{LVal}}
\newcommand{\Sb}{\mathbb{S}}

%ledger spec commands
\newcommand{\Request}{\type{Request}}
\newcommand{\LS}{\mathcal{L}\mathcal{S}}
\newcommand{\State}{\type{State}}
\newcommand{\CEMState}{\type{CEMState}}
% \newcommand{\AccID}{\type{AccID}}
\newcommand{\PCMT}{\type{PCMT}}
\newcommand{\Accts}{\type{Accts}}
% \newcommand{\Open}{\type{Open}}
% \newcommand{\OArgs}{\type{OArgs}}
% \newcommand{\Close}{\type{Close}}
% \newcommand{\CArgs}{\type{CArgs}}
% \newcommand{\Deposit}{\type{Deposit}}
% \newcommand{\DArgs}{\type{DArgs}}
\newcommand{\AMState}{\type{AMState}}
% \newcommand{\Withdraw}{\type{Withdraw}}
% \newcommand{\WArgs}{\type{WArgs}}
% \newcommand{\Transfer}{\type{Transfer}}
\newcommand{\TransferFrom}{\type{TransferFrom}}
\newcommand{\TransferTo}{\type{TransferTo}}
% \newcommand{\TArgs}{\type{TArgs}}
\newcommand{\TTArgs}{\type{TTArgs}}
\newcommand{\TFArgs}{\type{TFArgs}}
\newcommand{\CEMInput}{\type{CEMInput}}
\newcommand{\Tx}{\type{Tx}}
\newcommand{\ID}{\type{ID}}
\newcommand{\Th}{\mathcal{T}}
\newcommand{\Tu}{\mathcal{U}}
\newcommand{\T}{\type{T}}
\newcommand{\Err}{\type{Err}}
\newcommand{\ups}{\fun{update}}
\newcommand{\FHBMT}{\type{FHBMT}}
\newcommand{\UTxO}{\type{UTxO}}
\newcommand{\UTxOState}{\type{UTxOState}}
\newcommand{\TxIn}{\type{TxIn}}
\newcommand{\MsgIn}{\type{MsgIn}}

\newcommand{\CEMInit}{\type{CEMInit}}
\newcommand{\CEMStep}{\type{CEMStep}}

\newcommand{\CEMInputs}{\type{CEMInputs}}
\newcommand{\CEMStates}{\type{CEMStates}}

\newcommand{\PlutusVII}{\type{PlutusV2}}
\newcommand{\Credential}{\type{Credential}}
\newcommand{\Block}{\type{Block}}

\newcommand{\StateMachineState}{\type{StateMachineState}}
\newcommand{\StateMachineInput}{\type{StateMachineInput}}
\newcommand{\NFTSMState}{\type{NFTSMState}}
\newcommand{\NFTSMInput}{\type{NFTSMInput}}

\newcommand{\True}{\type{True}}
\newcommand{\False}{\type{False}}

\newcommand{\Ix}{\type{Ix}}
\newcommand{\Slot}{\type{Slot}}
\newcommand{\PParams}{\type{PParams}}
\newcommand{\Coin}{\type{Coin}}
\newcommand{\LState}{\type{LState}}
\newcommand{\Initialize}{\type{Initialize}}
\newcommand{\Progress}{\type{Progress}}

\newcommand{\OutputVal}{\type{OutputVal}}

\newcommand{\AssetID}{\type{AssetID}}
\newcommand{\TokenName}{\type{TokenName}}
\newcommand{\myThread}{\type{myThread}}
\newcommand{\pkSigns}{\type{pkSigns}}

\newcommand{\Counters}{\type{Counters}}
\newcommand{\CounterID}{\type{CounterID}}
\newcommand{\CS}{\type{CS}}
\newcommand{\CI}{\type{CI}}

\newcommand{\Env}{\type{Env}}
\newcommand{\FEnv}{\type{FEnv}}
\newcommand{\Id}{\type{Id}}

\usepackage{etoolbox}
\usepackage{tikz-qtree}
\usepackage{tikz}
\usetikzlibrary{matrix}

% For anonymisation
\newtoggle{anonymous}
\toggletrue{anonymous}
\iftoggle{anonymous}{
  \newcommand{\Cardano}{CHAIN}
  \newcommand{\Plutus}{LANG}
}{
  \newcommand{\Cardano}{Cardano}
  \newcommand{\Plutus}{Plutus Core}
}

% Names, for consistency
\newcommand{\UTXO}{UTxO}
\newcommand{\EUTXO}{E\UTXO{}}
\newcommand{\ExUTXO}{Extended \UTXO{}}

\newcommand{\SELF}{\type{SELF}}
\newcommand{\TRANS}{\type{TRANS}}
\newcommand{\POV}{\type{POV}}
\newcommand{\CEM}{\type{CEM}}
\newcommand{\Trc}{\type{Trc}}
\newcommand{\CEMS}{\type{CEMS}}
\newcommand{\SMUP}{\type{SMUP}}
\newcommand{\CEMsm}{CEM_{sm}}

\newcommand{\SR}{\type{SR}}
\newcommand{\MsgRdmr}{\type{MsgRdmr}}
\newcommand{\send}{\type{send}}
\newcommand{\receive}{\type{receive}}
\newcommand{\MSGS}{\type{MSGS}}
\newcommand{\Msg}{\type{Msg}}
\newcommand{\msgsTT}{\type{msgsTT}}
\newcommand{\msgsVal}{\type{msgsVal}}
\newcommand{\msgTkn}{\type{msgTkn}}

\newcommand{\Memory}{\type{Memory}}
\newcommand{\Params}{\type{Params}}
\newcommand{\HState}{\type{HState}}
\newcommand{\SType}{\type{SType}}
\newcommand{\HYDRA}{\type{HYDRA}}

\newcommand{\TOGGLE}{\type{TOGGLE}}
\newcommand{\Toggle}{\type{Toggle}}
\newcommand{\toggle}{\type{toggle}}
\newcommand{\Payout}{\type{Payout}}
\newcommand{\payout}{\type{payout}}
\newcommand{\myPKs}{\type{myPKs}}
\newcommand{\toggleTT}{\type{toggleTT}}
\newcommand{\toggleVal}{\type{toggleVal}}
\newcommand{\PartialOutput}{\type{PartialOutput}}

\newcommand{\UTXOma}{\UTXO$_{\textrm{ma}}$}
\newcommand{\EUTXOma}{\EUTXO$_{\textrm{ma}}$}

\newcommand\initial{\msf{initial}}
\newcommand\nft{\blacklozenge}
\newcommand\step{\msf{step}}
\newcommand\satisfies{\msf{satisfies}}
\newcommand\checkOutputs{\msf{checkOutputs}}
\newcommand\txeq{tx^\equiv}

% Account simulations
\newcommand{\threadTT}{\type{threadTT}}
\newcommand{\AccID}{\type{AccID}}
\newcommand{\AccData}{\type{AccData}}
\newcommand{\Accnts}{\type{Accnts}}
\newcommand{\OneUpd}{\type{OneUpd}}
\newcommand{\AccUpd}{\type{AccUpd}}
\newcommand{\Open}{\type{Open}}
\newcommand{\Oargs}{\type{Oargs}}
\newcommand{\Close}{\type{Close}}
\newcommand{\CArgs}{\type{CArgs}}
\newcommand{\Deposit}{\type{Deposit}}
\newcommand{\DArgs}{\type{DArgs}}
\newcommand{\Withdraw}{\type{Withdraw}}
\newcommand{\WArgs}{\type{WArgs}}
\newcommand{\Transfer}{\type{Transfer}}
\newcommand{\TArgs}{\type{TArgs}}
\newcommand{\ACCNTS}{\type{ACCNTS}}
\newcommand{\ACCOP}{\type{ACCOP}}

\newcommand{\PayIns}{\type{PayIns}}
\newcommand{\PayOuts}{\type{PayOuts}}
\newcommand{\Bag}{\type{Bag}}
\newcommand{\Rate}{\type{Rate}}
\newcommand{\OMap}{\type{OMap}}
\newcommand{\XOR}{\type{XOR}}

% relaxed float placement
\renewcommand{\topfraction}{.95}
\renewcommand{\bottomfraction}{.7}
\renewcommand{\textfraction}{.15}
\renewcommand{\floatpagefraction}{.66}
\renewcommand{\dbltopfraction}{.66}
\renewcommand{\dblfloatpagefraction}{.66}
\setcounter{topnumber}{9}
\setcounter{bottomnumber}{9}
\setcounter{totalnumber}{20}
\setcounter{dbltopnumber}{9}

\newcommand\CStep[1]{\ensuremath{
 #1 \xrightarrow{\hspace{5pt} i \hspace{5pt}} (#1' , \txeq)
%%\textsf{step}\, #1\, i \equiv \textsf{just}\, #1'
}}

\newcommand{\pp}{\nolinebreak\hspace{+0.3em}{\tiny\bf +}\nolinebreak\hspace{-.10em}{\tiny\bf +}\nolinebreak\hspace{+0.3em}\nolinebreak}



%This is a template for producing LIPIcs articles.
%See lipics-v2021-authors-guidelines.pdf for further information.
%for A4 paper format use option "a4paper", for US-letter use option "letterpaper"
%for british hyphenation rules use option "UKenglish", for american hyphenation rules use option "USenglish"
%for section-numbered lemmas etc., use "numberwithinsect"
%for enabling cleveref support, use "cleveref"
%for enabling autoref support, use "autoref"
%for anonymousing the authors (e.g. for double-blind review), add "anonymous"
%for enabling thm-restate support, use "thm-restate"
%for enabling a two-column layout for the author/affilation part (only applicable for > 6 authors), use "authorcolumns"
%for producing a PDF according the PDF/A standard, add "pdfa"

%\pdfoutput=1 %uncomment to ensure pdflatex processing (mandatatory e.g. to submit to arXiv)
%\hideLIPIcs  %uncomment to remove references to LIPIcs series (logo, DOI, ...), e.g. when preparing a pre-final version to be uploaded to arXiv or another public repository

%\graphicspath{{./graphics/}}%helpful if your graphic files are in another directory

\bibliographystyle{plainurl}% the mandatory bibstyle

\title{{Structured Contracts in the EUTxO Ledger Model}
\thanks{This work was supported by Input Output (iohk.io) through their funding of the Edinburgh Blockchain Technology Lab.}} %TODO Please add

%\titlerunning{Dummy short title} %TODO optional, please use if title is longer than one line

\author{Polina Vinogradova}{IOG, Singapore}{polina.vinogradova@iohk.io}{https://orcid.org/0000-0002-1825-0097}{}
\author{Manuel Chakravarty}{IOG, Singapore}{manuel.chakravarty@iohk.io}{https://orcid.org/0000-0002-1825-0097}{}
\author{Philip Wadler}{IOG, Singapore \and University of Edinburgh, UK}{philip.wadler@iohk.io}{https://orcid.org/0000-0002-1825-0097}{}
\author{James	Chapman}{IOG, Singapore}{james.chapman@iohk.io}{https://orcid.org/0000-0002-1825-0097}{}
\author{Tudor	Ferariu}{University of Edinburgh, UK}{s1408714@sms.ed.ac.uk}{https://orcid.org/0000-0002-1825-0097}{}
\author{Michael Peyton Jones}{IOG, Singapore}{michael.peyton-jones@iohk.io}{https://orcid.org/0000-0002-1825-0097}{}
\author{Jacco	Krijnen}{Utrecht University, Netherlands}{j.o.g.krijnen@uu.nl}{https://orcid.org/0000-0002-1825-0097}{}
\author{Orestis	Melkonian}{IOG, Singapore \and University of Edinburgh, UK}{orestis.melkonian@ed.ac.uk}{https://orcid.org/0000-0002-1825-0097}{}

%TODO mandatory, please use full name; only 1 author per \author macro; first two parameters are mandatory, other parameters can be empty. Please provide at least the name of the affiliation and the country. The full address is optional. Use additional curly braces to indicate the correct name splitting when the last name consists of multiple name parts.
% \author{Jane {Open Access}}{Dummy University Computing Laboratory, [optional: Address], Country \and My second affiliation, Country \and \url{http://www.myhomepage.edu} }{johnqpublic@dummyuni.org}{https://orcid.org/0000-0002-1825-0097}{This work was supported
% by Input Output (iohk.io) through their funding of the Edinburgh Blockchain Technology Lab.}%TODO mandatory,

% \author{Joan R. Public\footnote{Optional footnote, e.g. to mark corresponding author}}{Department of Informatics, Dummy College, [optional: Address], Country}{joanrpublic@dummycollege.org}{[orcid]}{[funding]}

% \author{Polina  Vinogradova\inst{1}\orcidID{0000-1111-2222-3333} \and
% Manuel Chakravarty\inst{1}\orcidID{0000-1111-2222-3333} \and %}{IOG} {manuel.chakravarty@iohk.io}{}{}
% Philip Wadler\inst{1,2}\orcidID{0000-1111-2222-3333} \and
% James	Chapman\inst{1}\orcidID{0000-1111-2222-3333} \and %}{IOG} {james.chapman@iohk.io}{}{}
% Tudor	Ferariu\inst{1,2}\orcidID{0000-1111-2222-3333} \and %}{University of Edinburgh, UK} {s1408714@sms.ed.ac.uk}{}{}
% Michael Peyton Jones\inst{1}\orcidID{0000-1111-2222-3333} \and %}{IOG} {michael.peyton-jones@iohk.io}{}{}
% Jacco	Krijnen\inst{1,3}\orcidID{0000-1111-2222-3333} \and %}{Utrecht University, Netherlands} {j.o.g.krijnen@uu.nl}{}{}
% Orestis	Melkonian\inst{1,2}\orcidID{0000-1111-2222-3333} } %}{IOG \& University of Edinburgh, UK} {orestis.melkonian@ed.ac.uk}{}{}


\authorrunning{P. Vinogradova et al.} %TODO mandatory. First: Use abbreviated first/middle names. Second (only in severe cases): Use first author plus 'et al.'

\Copyright{Polina Vinogradova et al.} %TODO mandatory, please use full first names. LIPIcs license is "CC-BY";  http://creativecommons.org/licenses/by/3.0/

\ccsdesc[100]{Security and privacy~Formal methods and theory of security} %TODO mandatory: Please choose ACM 2012 classifications from https://dl.acm.org/ccs/ccs_flat.cfm

\keywords{blockchain, ledger, smart contract, formal verification,
specification, transition system, UTxO, semantics} %TODO mandatory; please add comma-separated list of keywords

\category{} %optional, e.g. invited paper

\relatedversion{} %optional, e.g. full version hosted on arXiv, HAL, or other respository/website
%\relatedversiondetails[linktext={opt. text shown instead of the URL}, cite=DBLP:books/mk/GrayR93]{Classification (e.g. Full Version, Extended Version, Previous Version}{URL to related version} %linktext and cite are optional

%\supplement{}%optional, e.g. related research data, source code, ... hosted on a repository like zenodo, figshare, GitHub, ...
%\supplementdetails[linktext={opt. text shown instead of the URL}, cite=DBLP:books/mk/GrayR93, subcategory={Description, Subcategory}, swhid={Software Heritage Identifier}]{General Classification (e.g. Software, Dataset, Model, ...)}{URL to related version} %linktext, cite, and subcategory are optional

%\funding{(Optional) general funding statement \dots}%optional, to capture a funding statement, which applies to all authors. Please enter author specific funding statements as fifth argument of the \author macro.

% \acknowledgements{I want to thank \dots}%optional

%\nolinenumbers %uncomment to disable line numbering



%Editor-only macros:: begin (do not touch as author)%%%%%%%%%%%%%%%%%%%%%%%%%%%%%%%%%%
\EventEditors{John Q. Open and Joan R. Access}
\EventNoEds{2}
\EventLongTitle{42nd Conference on Very Important Topics (CVIT 2016)}
\EventShortTitle{CVIT 2016}
\EventAcronym{CVIT}
\EventYear{2016}
\EventDate{December 24--27, 2016}
\EventLocation{Little Whinging, United Kingdom}
\EventLogo{}
\SeriesVolume{42}
\ArticleNo{23}
%%%%%%%%%%%%%%%%%%%%%%%%%%%%%%%%%%%%%%%%%%%%%%%%%%%%%%

\begin{document}

\maketitle

%TODO mandatory: add short abstract of the document
\begin{abstract}
  The extended UTxO ledger is a kind of a UTxO-based cryptocurrency ledger that
  supports the use of smart contract scripts to specify permissions for performing
  certain actions, such as spending a UTxO or minting assets. There have been
  some attempts to standardize the implementation of stateful programs using this
  infrastructure, with varying degrees of success. In this work, we present a formalization of
  the very notion of correctly
  implementing stateful programs on the ledger, which we call the structured
  contract framework.

  Using a small-step semantics approach to program specification, the structured
  contracts framework establishes a subsystem relation between the ledger
  state transition system and the transition system of the program being specified.
  This gives users the tools for demonstrating the correctness of a transition
  system's ledger implementation.
  We argue that the framework is extremely versatile by presenting several
  highly distinct examples. In particular, we demonstrate how the framework allows
  for building distinct implementations of the same specification.
  We discuss useful potential applications of this framework to
  existing problems in smart contract verification.
\end{abstract}

\section{Introduction}
\label{sec:intro}

Many modern cryptocurrency blockchains are smart contract-enabled, meaning, generally, that
they provide some support for executing user-defined code as part of block or transaction processing.
This code is used to specify agreements between untrusted parties that can be
automatically enforced without a trusted intermediary. Examples of such
contracts may include distributed exchanges (DEXs), escrow contracts, auctions,
etc.

There is a lot of variation
in the details of how smart contract support is implemented across different platforms.
In particular, on account-based platforms such as Ethereum \cite{ethereum}
and Tezos \cite{tezos}, smart contracts are inherently stateful, and their states
can be updated by transactions. Smart contracts in
the extended UTxO (EUTxO) model, such as Cardano \cite{alonzo} and
\cite{ergo}, on the other hand, take the form of boolean predicates
on the data of the transaction executing them, and are inherently stateless.
In this model, transactions specify all the changes being done to the ledger
state, while contract predicates are used only to specify permissions
for performing the UTxO set updates specified by the transaction, such as spending
UTxOs or minting tokens.

Just within the UTxO model, there are multiple existing approaches to implementing and formalizing specific
designs of stateful programs running on the ledger \cite{eutxoma} \cite{marlowe} \cite{hydra}.
There are also languages, such as BitML \cite{bitml}, a DSL for specifying
contracts that regulate transfers of Bitcoin among participants.
There is also much existing work on smart contract formal verification.
However, currently, there are no principled standard practices for
specifying and implementing stateful programs on the EUTxO ledger.
In this work, we propose to re-use an existing ledger specification standard
to specify stateful contracts.

Like many prominent platforms
\cite{tezos} \cite{ethereum} \cite{Nakamoto} \cite{nervos} \cite{zil},
the Cardano implementation of the EUTxO ledger \cite{alonzo} is specified as a
transition system. The reason for this
design choice is that
the evolution of the ledger takes place in atomic steps corresponding to
the application of a single transaction. What sets the Cardano specification apart,
however, is the formal rigour of its operational
small-step semantics specification \cite{steps}.
We propose the \emph{structured contract framework}
(SCF) as an extension of this approach to specification.
It enables users
to instantiate a small-step program specification that runs on the ledger
via the use of smart contract scripts.

Generalizing the constraint-emitting machine design pattern \cite{eutxoma},
the SCF formalizes the notion of stateful program running on the EUTxO ledger,
and what it means for it to be \emph{implemented correctly}. We do so by
requiring instantiation of a stateful program to include a proof of a \emph{simulation relation }
between its specification and the ledger specification.
Our generalization allows expressing invariants (or safety properties) of contracts
for which it was not previously done. For example, we can express invariants of stateful
contracts that are implemented across multiple different UTxOs. We can also
express invariants on the totality of tokens under a specific policy by interpreting
it as the state of a structured contract. %invariants of the ledger itself? POV
We argue that the SCF constitutes a new, principled approach to stateful
smart contract architecture that is amenable to formal analysis and suitable for
a wide range of smart contract applications.
The main contributions of this paper are :

\begin{itemize}
  \item[(i)] a definition of a simplified EUTxO ledger using small-steps semantics ;

  \item[(ii)] a definition of the structured contract formalism (SCF) ;

  \item[(iii)] a case study expressing the
  minting policy of a single NFT as a structured contract ;

  \item[(iv)] a case study demonstrating the use the SCF to define
  two distinct ledger implementations of a
  single specification, including one that is distributed across multiple
  UTxO entries and interacting scripts ;

\end{itemize}

\section{EUTxO ledger model}
\label{sec:ledger-rules}

The extended UTxO (EUTxO) ledger model is UTxO-based ledger model that supports the use of
user-defined Turing-complete scripts to specify conditions for spending (consuming)
UTxO entries as well as token minting and burning policies.
The EUTxO model can be expressed as a labelled transition system, which we define
in this section.

The state of the system is a UTxO set, and the transition labels are transactions.
As not every transaction is a valid transition for every UTxO set, we additionally
formulate the constraints which a transaction must satisfy to be valid in a given
UTxO set. Note that while in a realistic system, transactions are applied to the
ledger in blocks, block structure and block-specific ledger state data is secondary
to the discussion of smart contracts in this work.

\subsection{Transition Relation Specification}

To give the small-steps semantics which we use to express the transition
rules and types, we follow the set-theory based notation
outlined in \cite{steps}, and used in \cite{alonzo} \cite{structured}.
Non-standard notation used here is given in \ref{fig:notation:nonstandard}.

We denote a transition relation $\mathsf{TRANS}$ as a 4-tuple :

  \begin{equation*}
    \_ \vdash
    \var{\_} \trans{trans}{\_} \var{\_}
    \subseteq (\Env \times \State \times \Input \times \State)
  \end{equation*}

Membership $(\var{env},~s,~\var{inp},~s') \in \mathsf{TRANS}$ is also denoted by

  \begin{equation*}
    env \vdash
    \var{s} \trans{trans}{inp} \var{s'}
  \end{equation*}

Given a 4-tuple $(\var{env},~s,~i,~s')$, the role of each component
specifying the transition is as follows :

\begin{itemize}
  \item[(i)] $\var{env} \in \Env$ is the fixed \emph{environment} of the state transition
  update, e.g. the current slot number ;

  \item[(ii)] $s \in \State$ is the starting state
  to which an input is applied, e.g. the UTxO set, or a contract state ;

  \item[(iii)] $i \in \Input$ is the user input, e.g. a transaction, or input to
  a contract ;

  \item[(iv)] $s' \in \State$ is the end state
  obtained from the start state as the result of the application of the input,
  e.g. the updated UTxO state.
\end{itemize}

A specification $\TRANS$ is made up of one or more
\emph{transition rules}. That is, the only 4-tuples that are members of $\TRANS$ are those that
satisfy the preconditions in one of its transition rules.

\paragraph{Input vs. Environment. }

Note that there is no formal distinction between the use of input and environment.
However, making this distinction creates a useful separation between user-issued
input (e.g. a transaction), and block-level data. For example, slot numbers are
determined via blocks and a timekeeping mechanism at the consensus level,
not specified by users. We adopt this convention from the
EUTxO transition system specification \cite{alonzo}.


\subsection{Ledger Types}

The ledger model we use is an extended UTxO model with native multi-assets,
as introduced in \cite{structured}. It is based on
a previously introduced model
\cite{eutxoma}. We give an overview of the EUTxO state and transaction structure
below, and present the full set of type definitions
in Figures~\ref{fig:eutxo-types} and~\ref{fig:ctx-types}.
% We note that in the previously presented EUTxO models, unhashed scripts and
% transactions were never stored on-chain.

\paragraph{Scripts. } A $\Script$, or a \emph{smart contract},
is a piece of stateless user-defined code with a
boolean output. It is used to specify conditions under which a transaction is allowed
to perform a specific action. It is executed
whenever a transaction attempts to perform the action for which it specifies the
permissions. We do not specify the language in which scripts are written,
but we presume Turing-completeness.

The input to a script consists of (i) a summary of transaction data,
(ii) a pointer to the specific action (within the transaction) for which the script
is specifying the permission (e.g. the policy ID or the input),
(iii) and a piece of user-defined data we call a $\Redeemer$. A redeemer is defined
at the time of transaction construction (by the transaction author) for each action
requiring a script to be
run. We denote evaluation for scripts that control
minting of tokens as well as spending by $\applyMPScript{\_}$, followed by the
script arguments.

\paragraph{Value. } The type $\Value = \FinSup{\Policy}{\FinSup{\TokenName}{\Quantity}}$
represents bundles of multiple kinds of assets.
It uses a nested finite map data structure to associate a $\Quantity = \Z$
to every $\AssetID = \Policy \times \TokenName$. That is, the bundle contains a
non-zero integral quantity of assets with the asset IDs specified
in the domain of the map, and zero quantity of assets with all other IDs.
A token bundle containing no tokens is denoted by $0$. $\Value$ forms a group for
which we denote group the operation by $+$, as well as a partial order \cite{utxoma}.
An $\AssetID$
uniquely identifies a class of assets, wherein all assets are fungible. It is made up of :

\begin{itemize}
  \item[(i)] a $\Policy$, which is a script that is executed
  whenever a transaction
  mints assets under this policy to determine whether it is allowed to do so ;
  \item[(ii)] a $\TokenName$, which is some $\Data$ a user selects to differentiate between
  assets under the same policy. For readability, this differs from the $\TokenName$ type in \cite{eutxoma},
  where it is defined to be character string.
\end{itemize}

We construct a $\Value$ with a single asset, whose asset ID is given by
\[ \fun{oneT}~\fun{policy}~\fun{tokenName}~ \leteq~ \{~\fun{policy}~\mapsto~\{ \fun{tokenName} \mapsto 1 \}~\} \]

\paragraph{UTxO set. } The UTxO set
is a finite map $\UTxO = \OutputRef \mapsto \Output$.
The $\OutputRef = \Tx \times \Ix$ is the type of the key of the UTxO finite map, with $\Ix = \N$.
An output reference $(\var{tx}, \var{ix})$ is made up of (i) a transaction $\var{tx}$ that
created the output to which it points, and (ii) index of $\var{ix}$,
pointing to the place of a particular output in the list of outputs
of that transaction. This uniquely specifies an
output in a specific transaction.

An output $(a, v, d) \in \Output$ is made up of (i) an address $a$, which is a $\Script$,
and it specifies under
what conditions a given output can be spent by a transaction, (ii) an asset bundle $v$ locked
by this script, and (iii) an additional piece of data $d \in \Datum$, specified
by the transaction that adds this output to the UTxO.

\paragraph{$\Data$. } Is a type for encoding data that can be passed as arguments
to scripts, similar in structure to a CBOR encoding.
For details, see \cite{agdaspeceutxo}. It is also used in \cite{eutxoma},
and in the Cardano implementation \cite{cardano}.
The types $\Datum$ (data stored alongside value in an output) and
$\Redeemer$ (data specified by the user for spending a specific UTxO entry)
are both synonyms for the $\Data$ type. For each script, conversion functions
for both the datum and redeemer are defined to decode and encode those arguments
to and from the specific argument types expected by the script. We usually call the decoding
function $\fun{fromData}$, and the encoding one $\fun{toData}$ when the context
is unambiguous.

\paragraph{Transactions. } $\Tx$ is a data structure that specifies
the updates to be done to the UTxO set. A transaction $\var{tx} \in \Tx$
contains (i) a set of \emph{inputs} each referencing entries in the UTxO set that the transaction
is removing (spending), with their corresponding redeemers, (ii) a set of outputs, which get entered into the
UTxO set with the appropriately generated output references, (iii) a pair of slot
numbers representing the validity interval of the transaction, (iv) a $\Value$ being
minted by the transaction, (v) a redeemer for each
of the minting policies being executed, and (vi) the set of (public) keys that signed the
transaction, together with their signatures.

\paragraph{$\Slot$ number. }  A slot number is a natural number used to represent
the time at which a transaction is processed, with $\Slot = \N$.

\subsection{Ledger Transition Semantics}
\label{sec:ledgersem}

The type of the ledger transition system $\mathsf{LEDGER}$, that updates the UTxO
set by applying a given transaction in a specific environment (slot), is
denoted as follows :

  \begin{equation*}
    \_ \vdash
    \var{\_} \trans{ledger}{\_} \var{\_}
    \subseteq (\Slot \times \UTxO \times \Tx \times \UTxO)
  \end{equation*}

We introduce the function $\fun{checkTx}$, which
consists of the constraints that must be satisfied by a given tuple
in order to constitute a valid ledger update. This function performs the following
checks, specified in Figure \ref{fig:validity}, and consistent with the ledger rules
specified in \cite{eutxoma} \cite{structured} :

\begin{itemize}
  \item[(i)] The transaction has at least one input
  \item[(ii)] The current slot is within transaction validity interval
  \item[(iii)] All outputs have positive values
  \item[(iv)] All output references of transaction inputs exist in the UTxO
  \item[(v)] Value is preserved
  \item[(vi)] No output is double-spent
  \item[(vii)] All inputs validate
  \item[(viii)] Minting redeemers are present
  \item[(ix)] All minting scripts validate
  \item[(x)] All signatures are present
\end{itemize}

We use the function $\fun{checkTx}$ to define membership in the
$\mathsf{LEDGER}$ relation, and call this rule $\fun{ApplyTx}$.
More specifically, $\fun{ApplyTx}$ states that
$(\var{slot},~\var{utxo},~\var{tx},~\var{utxo'}) \in \mathsf{LEDGER}$
whenever $\fun{checkTx}~(\var{slot},~\var{utxo},~\var{tx})$ holds and $\var{utxo'}$
is given by $(\{~i\mapsto o \in \var{utxo} ~\mid~ i \notin \fun{getORefs}~{tx}~\}) \cup \fun{mkOuts}~{tx} $.

% \begin{figure}[htb]
  \begin{equation}
  \label{fig:ledger-rule}
    \inference[ApplyTx]
    {
    \var{utxo'}~\leteq~(\{~i\mapsto o \in \var{utxo} ~\mid~ i \notin \fun{getORefs}~{tx}~\}) \cup \fun{mkOuts}~{tx}
    \\ ~ \\
    \fun{checkTx}~(\var{slot},~\var{utxo},~\var{tx})
    \\ ~ \\
    }
    {
    \begin{array}{l}
      \var{slot}\\
    \end{array}
      \vdash
      \left(
      \begin{array}{r}
        \var{utxo} \\
      \end{array}
      \right)
      \trans{ledger}{tx}
      \left(
      \begin{array}{r}
        \varUpdate{\var{utxo'}}  \\
      \end{array}
      \right) \\
    }
  \end{equation}

The value $\var{utxo'}$ is calculated by removing the UTxO entries in $\var{utxo}$
corresponding to the output references of the transaction inputs, and adding the outputs of the
transaction to the UTxO set with correctly generated output references.
The functions used to compute the updated UTxO set are defined in \ref{fig:ctx-types}.

Unlike the
ledger model in \cite{eutxoma}, which contains a full list of all transactions in
the order they have been applied to the initial \emph{empty} ledger,
$\LEDGER$ does not specify an initial ledger state. For this reason, we
sometimes have to make additional assumptions about transactions.

In later sections,
when necessary, we assume certain conditions on
the starting state of the ledger trace to be true, such as the existence of a specific
entry in the UTxO set. Note also that $\LEDGER$
has no trivial transitions due to the fact that transactions without
inputs are not allowed by rule \ref{rule:has-input}, so the UTxO set is necessarily updated
by the transaction by (at minimum) removing its inputs.
An arbitrary labelled transition $\mathsf{TRANS}$ need not be deterministic,
and may allow multiple distinct end states for a given initial state, environment,
and input. However, the $\mathsf{LEDGER}$ system is \emph{deterministic}, as the
end state variable $\var{utxo'}$ is uniquely specified by in its only rule, $\mathsf{ApplyTx}$.

\section{Simulations and the structured contract formalism}
\label{sec:struc}

The programs we are interested in specifying for the purpose of this work
are those that \emph{run on the ledger}. Intuitively, a stateful program is implemented
on the ledger whenever its state is observable in (i.e. computable from) the ledger state,
and whenever the ledger state is updated, the observed program state is updated in accordance
with the program's specification. We now formalize the notion of correctly implementing
a stateful program on the ledger using smart contract scripts.

The purpose of a smart contract script is to encode the conditions under which
a transaction \emph{can update a part of the ledger state} with which the script is
associated, e.g. change the total quantity of tokens under a given policy, or
remove some UTxO entries. This interpretation of the use of stateless
code on the ledger justifies a \emph{stateful} program model for representing
most programs running on the ledger. Stateful programs are implemented using one or more
interacting scripts controlling the updates of the corresponding data in the UTxO state.
The state of a program on the ledger is \emph{observed} by applying a projection
function to the ledger state which aggregates the relevant data.

A structured contract includes a specification, and a projection function
that computes the contract state from a given ledger state. It also requires
a proof of the integrity of the contract's implementation, establishing
a simulation relation between it and the ledger. That is, that the
scripts controlling the ledger data returned by the projection function
ensure the evolution of that data is according to the contract
specification. For example, the projection function may return the value and
datum of a UTxO containing a special NFT. This pair of makes up the state of
a given stateful contract. The script locking that UTxO
must guarantee that upon being spent, the NFT is always placed into a new UTxO
with a particular datum and value, which are computed according to the contract specification.
This approach to ensuring adherence to specification is called a \emph{thread token}
mechanism \cite{eutxoma}, and we will elaborate
on it in later sections.

\subsection{Simulations}

We instantiate the
definition a \emph{simulation} \cite{milner-pibook} with labelled
state transition systems expressed as small-step semantics specifications.

\textbf{Simulation definition.}
\label{def:simulation}
Let $\TRANS$ and $\STRUC$ be small-step labelled transition systems.
A \emph{simulation} of $\TRANS$ in $\STRUC$, denoted by
\[ (\STRUC, \sim_{\TRANS,\STRUC}, \sim_{\to,\TRANS,\STRUC}) \succeq \TRANS \]

Consists of of the following types together with the following relations :
\begin{equation*}
  \_ \vdash
  \var{\_} \trans{trans}{\_} \var{\_}
  \subseteq (\Env_{\TRANS} \times \State_{\TRANS} \times \Input_{\TRANS} \times \State_{\TRANS})
\end{equation*}
\begin{equation*}
  \_ \vdash
  \var{\_} \trans{struc}{\_} \var{\_}
  \subseteq (\Env_{\STRUC} \times \State_{\STRUC} \times \Input_{\STRUC} \times \State_{\STRUC})
\end{equation*}
\begin{align*}
  \wcard~\sim_{\TRANS,\STRUC}~\wcard ~&:~ \State_{\TRANS} \times \State_{\STRUC} \to \Bool \\
  \wcard~\sim_{\to,\TRANS,\STRUC} ~\wcard ~&:~ (\Env_{\TRANS} \times \Input_{\TRANS}) \times (\Env_{\STRUC} \times \Input_{\STRUC}) \to \Bool
\end{align*}

such that the following holds :
  \begin{equation}
    \inference[$\sim >$]
    {
      \\~\\
      (e,~i) \sim_{\to,\TRANS,\STRUC} (e', j) & u~\sim_{\TRANS,\STRUC}~s
      \\~\\
      {
        \begin{array}{c}
          e\\
        \end{array}
      }
      \vdash
      {
        \left(
          \begin{array}{r}
            \var{s} \\
          \end{array}
        \right)
      }
      \trans{trans}{\var{i}}
      {
        \left(
          \begin{array}{r}
            \var{s'} \\
          \end{array}
        \right)
      }
      \\~\\
    }
    {
      u'~\sim_{\TRANS,\STRUC}~s' ~~~~~
      \begin{array}{r}
        e'\\
      \end{array}
    \vdash
      (u)
      \trans{struc}{j}
      (\varUpdate{u'})
    }
  \end{equation}

%   \begin{equation}
%     \inference[$\sim >$]
%     {
%       \\~\\
%       \pi_{\STRUC}~u~ \neq~ \emptytype &
%       {
%         \begin{array}{c}
%           e\\
%         \end{array}
%       }
%       \vdash
%       {
%         \left(
%           \begin{array}{r}
%             \var{u} \\
%           \end{array}
%         \right)
%       }
%       \trans{ledger}{\var{t}}
%       {
%         \left(
%           \begin{array}{r}
%             \var{u'} \\
%           \end{array}
%         \right)
%       }
%       \\~\\
%     }
%     {
%       (\pi_{\STRUC}~u'~ \neq~ \emptytype) ~~\wedge~~
%       \begin{array}{r}
%         * \\
%       \end{array}
%     \vdash
%       (\pi_{\STRUC}~u)
%       \trans{struc}{\pi_{\Tx,\STRUC}~t}
%       (\varUpdate{\pi_{\STRUC}~u'})
%     }
%   \end{equation}


We write $\sim$ instead of $\sim_{\TRANS,\STRUC}$ whenever the subscript is
unambiguous. This relation states that a if a valid state $s \in \State_{\STRUC}$ is
associated with a valid UTxO state $u$, then any ledger transition starting in
$s \in \State_{\TRANS}$ is necessarily associated with a valid transition starting
in state $s$. Note that $\sim$ is a \emph{proof obligation} that must be
fulfilled as part of the definition, which \emph{does not define a rule}.
One can construct pairs of transition
systems with $\sim,~\sim_{\to}$ relations for which this proof obligation cannot
be fulfilled. However, \emph{if} it is possible, we have a simulation of $\TRANS$
in $\STRUC$.

% Note here that, given a contract state $s$ such that $\var{utxo} \sim s$,
% the only entries in the UTxO set are ones that contain some part of the
% representation of the state $s$. Then, any transaction applied to $\var{utxo}$
% will necessarily change some part of the state representation, and therefore
% correspond to a non-trivial state update.


\subsection{Structured contracts.}
\label{sec:struc-def}

The simulation definition we give is general, however, the rest of this work is
geared towards reasoning about the programmable
parts of the ledger, i.e. those where the permissions are controlled by user-defined
scripts. For this reason, we define a particular class of simulations of $\LEDGER$.
First of all, since scripts are not allowed to inspect block-level data (i.e. the
current slot number), we fix the environment of the structured contract specification
to be a singleton type $\emptytypeT$. Secondly, $\sim$ must be a
partial function, rather than a relation, which computes a unique contract state
for a given UTxO state (or fails, returning $\emptytypeT$). The
relation between arrows must also be expressible as a function which
computes a specific contract input value for any given transaction.

\textbf{Definition (Structured contract).}
Let
$(\STRUC, \sim_{\LEDGER,\STRUC}, \sim_{\to,\LEDGER,\STRUC}) \succeq \LEDGER$
be a simulation. We say that it is a \emph{structured contract} whenever
$\Env_{\STRUC} = \emptytype$, and there exist
two functions $\pi_{\STRUC} : \UTxO \to \State_{\STRUC} \cup \emptytypeT$,
$\pi_{\Tx,\STRUC} : \Tx \to \Input_{\STRUC}$ such that :
\begin{align*}
  \var{utxo}~\sim~s &\leteq (\pi_{\STRUC}~\var{utxo} ~=~s) \\
  (\var{slot},~\var{tx})~\sim_{\to,\LEDGER,\STRUC}~ (\emptytype, i) &\leteq (\pi_{\Tx,\STRUC}~\var{tx} = i)
\end{align*}

\textbf{Discussion. }
% This definition states that whenever for a given start state $\var{utxo}$
% with a corresponding contract start state $\pi~\var{utxo} \neq \emptytype$,
% each valid step $(\var{slot}, \var{utxo}, \var{tx}, \var{utxo'}) \in \LEDGER$
% at the ledger level necessarily has a corresponding valid
% step at the contract level,
% $(\emptytype, \pi~\var{utxo}, \pi_{\Tx}~\var{tx}, \pi~\var{utxo'}) \in \STRUC$.
We sometimes denote $\pi_{\Tx,\STRUC}$ and $\pi_{\STRUC}$
by $\pi$ and $\pi_{\Tx}$, respectively, when the context is clear.
We also denote the structured contracts by $(\STRUC,~\pi_{\STRUC},~\pi_{\Tx,\STRUC}) \succeq \LEDGER$.
We say that a $\var{utxo}$ is \emph{valid for $\STRUC$} whenever $\pi~\var{utxo} \neq \emptytype$.

It is possible for transactions to update the ledger state, but not the $\STRUC$ state.
There is no a special case for this in the simulation relation, and the $\STRUC$
specification rules must allow trivial steps if such ledger transactions are possible.
The $\STRUC$ specification may not be deterministic, however, since the
$\LEDGER$ is deterministic, given a ledger
step, there is a unique step in $\STRUC$ corresponding to the ledger one.
We do not assume that a valid contract state can be computed from an arbitrary
UTxO state. For this reason, the function $\pi_{\STRUC}$ is partial.
For example, it is possible that two NFTs exist in a given ledger
state. When programmed correctly, an NFT minting policy would not allow this
to happen. To reason about properties of such a policy, we must exclude ledger states
where the NFT uniqueness condition has a been violated in the start state.

Requiring $\sim >$ to be a bisimulation \cite{milner-pibook} between $\STRUC$ and
$\LEDGER$ is too restrictive, and excludes a lot of interesting contracts.
Defining a class of structured contracts for which
this is possible is a difficult problem, and we leave it for future work.

\section{NFT minting policy as a structured contract}
% \label{sec:pov}
%
% In this section we give an example of using the structured contract
% formalism to express global invariants of the ledger that
% can be derived from local invariants.
% The EUTxO ledger transition system $\LEDGER$ is said to satisfy the
% \emph{preservation of value (POV)} property. The \emph{local} POV is
% the property that each valid transaction mints exactly the difference
% between the value in the UTxOs in spends and those it creates.
% This is ensured by the ledger rule \ref{rule:value-is-preserved}, which is
% \[ \mint~\var{tx} + \sum_{i \in \inputs~\var{tx},~(\outputref~i)\mapsto~o~\in~\var{utxo}} \val~\var{o} = \sum_{o \in \outputs~\var{tx}} \val~\var{o} \]
%
% Suppose that $(\var{s},~\var{utxo},~\var{tx},~\var{utxo'}) \in \LEDGER$, and
% recall that updated UTxO set is computed as follows
% \[ \var{utxo'} \leteq (\{~i\mapsto o \in \var{utxo} ~\mid~ i \notin \fun{getORefs}~{tx}~\}) \cup \fun{mkOuts}~{tx} \]
%
% The \emph{global} POV, which we will prove from the local POV,
% states that the sum total of all the tokens
% recorded on the ledger is changed by exactly the amount minted or burned by the
% transaction,
% \[ \sum_{\var{or}\mapsto \var{out}\in \var{utxo'}}~\val~\var{out} ~= ~(\sum_{\var{or}\mapsto \var{out}\in \var{utxo}}~\val~\var{out}) ~+~(\fun{mint}~ \var{tx}) \]
%
% We can express this property as the following structured contract, $\mathsf{POV}$ : \newline
%
% % \begin{figure}[htb]
% \noindent \emph{Transition type}
%   \begin{equation*}
%     \_ \vdash
%     \var{\_} \trans{pov}{\_} \var{\_}
%     \in \powerset (\emptytype \times \Value \times \Tx \times \Value)
%   \end{equation*}
%   %
% \emph{Simulation relations}
%   \begin{align*}
%     \pi_{\Tx,\POV}~\var{tx}~ ~&\leteq~\var{tx} \\
%     \pi_{\POV}~\var{utxo}~&\leteq~\sum_{\wcard \mapsto \var{out}\in \var{utxo}}~\val~\var{out}
%   \end{align*}
%   %
% \emph{Transition rule}
%   \begin{equation}
%     \inference[UpdateValTotal]
%     {
%     i \leteq \fun{mint}~\var{tx}
%     \\ ~ \\
%     }
%     {
%     \begin{array}{l}
%       \\
%     \end{array}
%       \vdash
%       \left(
%       \begin{array}{r}
%         \var{s} \\
%       \end{array}
%       \right)
%       \trans{pov}{tx}
%       \left(
%       \begin{array}{r}
%         \varUpdate{\var{s}~+~\var{i}}  \\
%       \end{array}
%       \right) \\
%     }
%   \end{equation}
% %   \caption{Specification of the $\mathsf{UpdateValTotal}$ POV subsystem}
% %   \label{fig:pov-spec}
% % \end{figure}
%
% To complete the definition, it remains to prove
% $\sim >$, which requires that, given that
% $(\var{slot},~\var{utxo},~\var{tx},~\var{utxo'}) ~\in \LEDGER$,
% necessarily $(\emptytype,~\pi~\var{utxo},~\pi~\var{tx},~\pi~\var{utxo'}) \in \POV$.
% This simplifies to exactly our definition of the global POV.
% See \ref{fig:pov-pf} for the proof.
%
% % Prove the POV property for the subsystem
% % Automatically derive the POV property for the ledger
% % While the workflow of Fig.9 is:
% % Prove an ad-hoc property for the subsystem (not really formulated as the POV property of the subsystem)
% % Wow! It turns out it's exactly identical to the POV property for the ledger, QED.
%
%
% \subsection{NFT Minting policy specification.}
\label{sec:mit-example}

Our first structured contract example is expressing a \emph{specific minting policy}.
% This example differs from the general POV contract example in a
% noteworthy way. The POV contract represents a property of the ledger transition system.
Constructing structured contracts specifying the evolution of the quantity of tokens
under a specific policy is a tool for formal analysis of minting policy code. In particular,
we are able to express and prove the defining property of a specific NFT : at most one
such token can exist on the ledger. Instantiating an NFT as a structured contract
and expressing this property allows us to mimic something that is quite naturally
expressed for account-based blockchains with stateful NFT contracts, such as the
ERC-721 \cite{erc721}, but poses a challenge for EUTxO ledger program analysis.

We first pick an identifier for the policy we wish to express,
$\fun{myNFTPolicy}$. Before writing the policy code, we define a system
$\fun{NFT}$ to specify how we want
the total number of tokens on the ledger under this policy to behave.
Here, the state type is $\State \leteq \Value$, and $\Input \leteq \Tx$.
\begin{equation}
  \inference[UpdateNFTTotal]
  {
  i \leteq \{~\fun{myNFTPolicy}~\mapsto~\var{tkns}~\in~ \fun{mint}~\var{tx}~\} \\~\\
  \{\} \subseteq s \subseteq s + i \subseteq \fun{oneT}~\fun{myNFTPolicy}~[~]
  \\ ~ \\
  }
  {
  \begin{array}{l}
    \\
  \end{array}
    \vdash
    \left(
    \begin{array}{r}
      \var{s} \\
    \end{array}
    \right)
    \trans{nft}{tx}
    \left(
    \begin{array}{r}
      \varUpdate{\var{s}~+~\var{i}}  \\
    \end{array}
    \right) \\
  }
\end{equation}

The specification states that the only allowed transitions are (i) a constant one,
and (ii) adding a single NFT, given by $\fun{oneT}~\fun{myNFTPolicy}~[~]$, to the state if
one does not yet exist. An NFT whose total ledger quantity obeys this specification
can never be burned, and must be the only token under its policy. It also
does not require any authentication to be minted. To define the policy and
projection functions, we pick an output reference $\fun{myNFTRef}$ which we call an
\emph{anchor}. That is,
$\fun{myNFTRef}$ must be spent by the NFT-minting transaction as a
mechanism to ensure that no other transaction can mint another NFT under this policy.
Next, we define the projection function,
\begin{align*}
  & \pi~\var{utxo}~ \leteq~\begin{cases}
    s & \text{ if~~} (s ~=~ \fun{oneT}~\fun{myNFTPolicy}~[~]~~\wedge~~\neg~\fun{hasRef}) ~~\vee~~ s = 0\\
    \emptytype & \text{ otherwise } \\
  \end{cases} \\
  &  \where \\
  &  ~~~~s \leteq~ \{~p~\mapsto~ \var{tkns}~\mid~p\mapsto \var{tkns} \in \sum_{\wcard \mapsto \var{out}\in \var{utxo}}~\val~\var{out},~p = \fun{myNFTPolicy}~\}  \\
  &  ~~~~\fun{hasRef} \leteq (\fun{myNFTRef} \mapsto \wcard \in \var{utxo})
\end{align*}

Here, $\pi~\var{utxo}$ returns a non-$\emptytype$ result when either
no tokens under the $\fun{myNFTPolicy}$ policy exist, or only the token
$\fun{oneT}~\fun{myNFTPolicy}~[~]$ exists under this policy, and the anchor $\fun{myNFTRef}$
is not in the UTxO. We define the policy,
\begin{align*}
  \fun{myNFTPolicy} \leteq &~ \fun{mkMyNFTPolicy}~ \fun{myNFTRef}
  \nextdef
  \applyMPScript{\fun{mkMyNFTPolicy}~ \var{myRef}}~\wcard~(\var{tx},~\var{pid})~\leteq &~
  \exists ~(\var{myRef}, \wcard, \wcard)~\in~ \inputs~\var{tx}~\\
  & \wedge ~~\fun{oneT}~\var{pid}~[~] ~\in~ \mint~\var{tx}
  % \nextdef
  % \applyScript{\fun{myNFTScript}}~\wcard~\wcard~(\var{tx},~\var{i})~\leteq &~   (\fun{myNFTScript},~\wcard, ~\wcard)
  % ~\fun{oneT}~(\fun{mkMyNFTPolicy}~ (\var{tx},~\var{i}))~[~] ~=~ \mint~\var{tx}
\end{align*}

% is identified uniquely by the output reference $\fun{myNFTRef}$ that must be
% spent to mint it. To make minting possible, the transaction $\fun{tx}$ must first place the
% following output into the UTxO, with $\fun{myNFTRef} \leteq (\fun{tx}, \fun{ix})$,
% \[ \fun{myOut} \leteq \fun{myNFTRef} \mapsto (\fun{myNFTScript},~\wcard, ~\wcard) \]
%
% The reference $\fun{myNFTRef}$ is chosen by the contract author, and is only
% valid if it points to $\fun{myOut}$ in the UTxO.

To prove $\sim >$ for the $\NFT$ contract (see Appendix \ref{pf:nft} for a proof
sketch), we need to make an additional assumption stating that a transaction which
adds $\fun{myNFTRef}$ to the UTxO cannot be valid again later, once the NFT
has been minted :

\textbf{NFT re-minting protection.}
\label{sec:asm-nft}
For any $(\var{slot},~\var{utxo},~\var{tx},~\var{utxo'}) \in \LEDGER$, with
$\pi~\var{utxo}~= \fun{oneT}~\fun{myNFTPolicy}~[~]$, necessarily $\var{tx} \neq \fun{fst}~\mathsf{myNFTRef}$.

Under reasonable assumptions about the initial state of the ledger, this property
should be consequence of replay protection, which is a trace-based
safety property of the EUTxO $\LEDGER$. A full treatment of traces and properties is
the subject of future work. So, to ensure correct program behaviour, we introduce
the above assumption.
% A related conjecture, provenance, is also proved in \cite{eutxoma}
% for a ledger structure with a fixed initial state and similar validation rules.
% We note that this is an assumption that must be made for all contracts sourcing
% the uniqueness of their NFT identifier from the uniqueness of a an output
% reference across the entire ledger state history.

%  To prove this trace-based property, assumptions about
% the UTxO state prior to the minting of allowed initial UTxO state must be made.  is outside the scope of this work. A similar
% conjecture, called provenance, is made and proved in \cite{eutxoma}. This is possible
% because the ledger structure in that work is the complete list of all transactions, in order
% of their application. This makes it possible to reason about the full transaction
% history, rather than only

\textbf{$\fun{NFT}$ property example.}
At most one NFT under the policy $\fun{myNFTPolicy}$ can ever exist in any
$\var{utxo}$ that is valid for $\NFT$ :
for any $\var{utxo}$ such that $\fun{pi}~\var{utxo} \neq \emptytype$,
\[ \fun{pi}~\var{utxo}~\subseteq~ \fun{oneT}~\fun{myNFTPolicy}~[~] \]

This is immediate from the definition of $\fun{pi}$, however, this result is
meaningful. By definition of $\sim >$, and the fact
that $\NFT$ is a structured contract, it is not possible
to transition from a state valid for $\NFT$ to a state which is not valid for $\NFT$. That is,
with $(\var{slot}, \var{utxo}, \var{tx}, \var{utxo'}) \in \LEDGER$, the
updated state $\pi~\var{utxo'}$ must also always have at most one NFT under $\fun{myNFTPolicy}$.
This also implies that at most one can ever be minted by a valid transaction
applied to a $\var{utxo}$ valid for $\NFT$.
%
%
%
% for any $\var{tx}$ such that
% $i \leteq \{~\fun{myNFTPolicy}~\mapsto~\var{tkns}~\in~ \fun{mint}~\var{tx}~\}$,
% \[ \forall~ s,~i,~~ \{\} \subseteq s \subseteq s + i \subseteq \fun{oneT}~\fun{myNFTPolicy}~[~] \]
%
% This is immediate from the $\fun{NFT}$ specification. Note that this is a safety
% propery \cite{liveness}, as it can be checked by inspecting a finite number
% of individual UTxO states in a given trace.

\section{Multiple implementations of a single specification}
\label{sec:toggle}

In this section we present an example of a specification that has more than
one correct implementation, one of which is distributed
across multiple UTxO entries. The guarantee that the two implement the same specification
enables contract authors to meaningfully compare them across any relevant characteristics,
such as space usage, or parallelizability.

\subsection{Toggle specification}

We define a specification wherein
the state consists of two booleans, and only one can be $\true$
at a time. We set the contract input to be
be $\{\toggle \} \cup \emptytype$.
The two booleans in the state are both flipped by the input $\toggle$, and unchanged by $\emptytype$.
We define the transition system $\TOGGLE$ :
% \begin{figure}[htb]
  \begin{equation}
    \inference[DoNothing]
    { \\
    }
    {
    \begin{array}{l}
      \\
    \end{array}
      \vdash
      \left(
      \begin{array}{r}
        x, ~y
      \end{array}
      \right)
      \trans{TOGGLE}{\emptytype}
      \left(
      \begin{array}{r}
        x, ~y
      \end{array}
      \right) \\
    }
  \end{equation}
  \begin{equation}
    \inference[Toggle]
    {~\\
    }
    {
    \begin{array}{l}
      \\
    \end{array}
      \vdash
      \left(
      \begin{array}{r}
        x,~y
      \end{array}
      \right)
      \trans{TOGGLE}{toggle}
      \varUpdate{\left(
      \begin{array}{r}
        y,~x
      \end{array}
      \right) } \\
    }
  \end{equation}
%   \caption{Specification of the $\mathsf{TOGGLE}$ transition}
%   \label{fig:toggle-rule}
% \end{figure}

\subsection{Toggle implementations}

We present two implementations of the $\TOGGLE$ specification. The \emph{naive implementation}
is one that uses the datum of a single UTxO entry to store a representation of the full state of the
$\TOGGLE$ contract. The \emph{distributed implementation} uses datums in two distinct UTxO
entries to represent the first and the second value of the pair that is the $\TOGGLE$ state.

\textbf{Thread token scripts. }
We use the \emph{thread tokens} mechanism \cite{eutxoma} to ensure the unique identification of the
UTxO (or pair of UTxOs) from which the contract state is computed. In both
implementations, the thread token minting policy guarantees that they
are generated in quantity 1 by a transaction that spends a specific output reference $\mathsf{myRef}$,
similar to the NFT policy in Section \ref{sec:mit-example}.

For the naive implementation, one thread token NFT is sufficient to identify the
state-bearing UTxO. Upon minting, the policy requires the token to be placed
into a UTxO locked by a specific contract, which is passed as a parameter to
the minting policy. This contract (discussed below) ensures the correct evolution of the
contract state. The datum in the UTxO containing the thread token is the initial state of the
contract encoded as a pair of booleans
(by the partial decoder function $\fun{fromData}_{N} : \Data \to \B \cup \emptytypeT$).
It can be any pair of correctly encoded booleans.
See Figure \ref{fig:toggle-tt-n} for the policy pseudocode.

For the distributed implementation, two distinct NFTs are needed to identify
the UTxOs containing the $\TOGGLE$ state data. Both NFTs are under the same
minting policy and must be minted by a single transaction, but have distinct token
names, $"a"$ and $"b"$. Upon being minted,
the policy requires that they are placed in separate UTxO, locked by the same
contract (discussed below). The datum in each must be decodeable
(by $\fun{fromData}_{N} : \Data \to (\B \times \B) \cup \emptytypeT$) as a boolean.
See Figure \ref{fig:toggle-tt-d} for the policy pseudocode.

% are required
% To define unique identifiers that point to the $\TOGGLE$ contract representation in the UTxO,
% we borrow the design pattern of \emph{thread tokens} introduced in \cite{eutxoma}. The idea is that
% an NFT is minted and stored in a UTxO entry that contains the datum representing (all
% or part of) the contract
% state. Since the NFT is script-guaranteed to be unique, it marks the unique authentic
% state-storing UTxO.
% When the state is updated, the UTxO containing the thread token is spent, and the
% thread token is placed in a new entry containing the updated state representation.
%
% \textbf{Naive implementation. }
% The naive implementation state is stored within a single UTxO, and so requires
% one thread token to track the evolution of the subsystem state.
%
% An output reference $\mathsf{myRef} \in \OutputRef$ is
% selected from those existing in the UTxO to uniquely identify the
% thread token. It must be spent for the minting policy to validate
% as a mechanism for ensuring this policy can only
% validate once, minting exactly one thread token, and placing it into the
% appropriately constructed output.
% The minting policy for the thread tokens is constructed as follows :
%
%
%
% \textbf{Distributed implementation. }
% The minting policy for the two distributed implementation
% thread tokens is constructed by :
%
%
%
% Here, two distinct thread tokens must be minted and placed into separate UTxOs
% locked with the correct script. This is because each UTxO representing part of the
% toggle state requires a unique identifier. We distinguish between the tokens
% by adding $"a"$ to the token name of one, and $"b"$ to the other.

% The functions, whose type is specified in \ref{fig:to-data}, encode and decode the
% datums which the implementation scripts inspect.
% Datums appears in transactions and UTxOs in the encoded $\Data$ format.

\textbf{Validator scripts. }
We require different UTxO-locking scripts for our two distinct implementations.
Both scripts serve the following function : when $\toggle$ redeemers are specified,
the script must ensure that the thread
tokens are propagated into UTxOs that are locked by the same validator as
the spent UTxOs containing the thread tokens, and that the datums in those UTxOs
are correct. For the naive version, the datum in the new UTxO containing the thread
token must decode as a pair of booleans whose order is reversed as compared to
the booleans encoded in the datum of the spent UTxO that previously contained the thread token.
We define it by :
% \begin{ruledfigure}{t}
  \begin{align*}
    \applyScript{\toggleVal_N~\mathsf{myRef}}~(b, b')~r~(\var{tx},~\var{i})~\leteq &~
    \var{ttt}~ =~ \val~(\fun{output}~i)~~\wedge~~r = \toggle \\
    & ~~~~ \wedge~~\exists~o~\in~\outputs~\var{tx},~~(b',~b)~=~(\datum~o) \\
    & ~~~~ \wedge~~(\validator~o~=~\var{vi}) ~~\wedge ~~(\var{ttt} =~\val~o)\\
    & \where \\
    & ~~~~\var{vi} \leteq \validator~(\fun{output}~i) \\
    & ~~~~\var{ttt} \leteq \mathsf{oneT}~(\toggleTT_N~\mathsf{myRef}~\var{vi} )~(\fun{encode}~\var{vi})
  \end{align*}
% \caption{$\TOGGLE$ validator script for the naive implementation}
% \label{fig:toggle-val-n}
% \end{ruledfigure}

The function $\fun{encode} : \Script \to [\type{Char}]$ encodes a script as a string
for the purpose of specifying (via the token name) the output-locking script that
must persistently lock the thread token.

The distributed implementation script ensures that both the thread token-containing
UTxOs are spent simultaneously. Then, it checks that the booleans in the datums are switched places :
the one that was in the UTxO with token $"a"$ must now be in a new UTxO with
token $"b"$, and vice-versa. The validator script is given in Figure \ref{fig:toggle-val-d}.
\begin{ruledfigure}{t}
  \begin{align*}
    \applyScript{\toggleVal_D~&\mathsf{myRef}}~b~\toggle~(\var{tx},~\var{i})~\leteq ~
    (\var{tta}~ =~ \val~(\fun{output}~i)) \Rightarrow \\
    & ~~~~\exists~o,~o'\in~\outputs~\var{tx},~i' \in \fun{inputs}~\var{tx}, \\
    & ~~~~ \validator~o~=~\validator~o'~=~\var{vi}~\wedge~\\
    & ~~~~ \mathsf{tta}~ =~ \val~o \wedge \var{ttb}~ =~ \val~o' \wedge \val~(\fun{output}~i')~=~\var{ttb} \\
    & ~~~~ \datum~o~=~\datum~ (\fun{output}~i') \wedge \datum~o'~=~\datum~ (\fun{output}~i) \\
    & \wedge \\
    & (\var{ttb}~ =~ \val~(\fun{output}~i)) \Rightarrow \\
    & ~~~~\exists~o,~o'\in~\outputs~\var{tx},~i' \in \fun{inputs}~\var{tx}, \\
    & ~~~~ \validator~o~=~\validator~o'~=~\var{vi}~\wedge~\\
    & ~~~~ \mathsf{tta}~ =~ \val~o \wedge \var{ttb}~ =~ \val~o' \wedge \val~(\fun{output}~i')~=~\var{tta} \\
    & ~~~~ \datum~o~=~\datum~ (\fun{output}~i) \wedge \datum~o'~=~\datum~ (\fun{output}~i') \\
    & \wedge \\
    & (\var{tta}~ =~ \val~(\fun{output}~i)) \vee (\var{ttb}~ =~ \val~(\fun{output}~i)) \\
    & \where \\
    & ~~~~\var{vi} \leteq \validator~(\fun{output}~i) \\
    & ~~~~\var{tta} \leteq \mathsf{oneT}~(\toggleTT_D~\mathsf{myRef}~\var{vi})~(\fun{encode}~\var{vi}~++~ "a") \\
    & ~~~~\var{ttb} \leteq \mathsf{oneT}~(\toggleTT_D~\mathsf{myRef}~\var{vi})~(\fun{encode}~\var{vi}~++~"b")
  \end{align*}
\caption{$\TOGGLE$ validator script for the distributed implementation}
\label{fig:toggle-val-d}
\end{ruledfigure}

\textbf{Ledger representation.}
\label{sec:led-rep}
The state projection function computations return a valid contract state (i.e. a pair of
booleans) whenever the anchor reference $\mathsf{myRef}$ is not in the UTxO,
and thread tokens have been minted according to their policy and placed alongside
the appropriate datums and UTxO scripts. The input projection function returns
$\toggle$ whenever a transaction contains the thread tokens in its input(s),
and $\emptytype$ otherwise. For details,
see Figures \ref{fig:toggle-sim-n} \ref{fig:toggle-sim-d-id}.

\begin{ruledfigure}{t}
  \begin{align*}
    \pi_d~\var{utxo}&~ \leteq~\begin{cases}
      (a, b) & \text{ if } \mathsf{myRef}~\notin~\{ ~i~\mid~i\mapsto o \in  \var{utxo}~\} \\
      & \wedge~~ \exists ! ~(i\mapsto o,~i'\mapsto o')~\in~\var{utxo},~\fun{tta}~=~\val~o~~\wedge~~\fun{ttb}~=~\val~o'\\
      & ~~~~ \wedge~~\validator~o~=~\toggleVal_D~\mathsf{myRef}~=~\validator~o'~~\\
      & ~~~~\wedge~~\datum~o~=~a~~\wedge~~\datum~o~=~b \\
      \emptytype~& \text{otherwise}
    \end{cases}
    \nextdef
    \pi_{\Tx,d}~\var{tx}~ &\leteq~\begin{cases}
      \toggle & \text{ if } \exists~i,~i'~ \in~\inputs~\var{tx},~\val~(\fun{output}~i) = \fun{tta} ~~\wedge ~~\val~(\fun{output}~i') = \fun{ttb}\\
      \emptytype & \text{otherwise}
    \end{cases}
  \end{align*}
\caption{$\TOGGLE$ distributed projections}
\label{fig:toggle-sim-d-id}
\end{ruledfigure}

In Appendix \ref{sec:toggle-sim}, we give a proof sketch for the simulation
relations between $\TOGGLE$ and $\LEDGER$ to complete the instantiation of
the two versions of the structured contract. To avoid duplication of
thread tokens, we again need to make the
additional assumption that
for any $(\var{slot},~\var{utxo},~\var{tx},~\var{utxo'}) \in \LEDGER$, with
$\pi~\var{utxo}~= (a, b)$, necessarily $\var{tx} \neq \fun{fst}~\mathsf{myRef}$.


% \paragraph{$\LEDGER$ - $\TOGGLE$ state and transition relations}
%
% Here we first encounter a situation wherein not every UTxO has a corresponding
% $\TOGGLE$ state. In particular, we can only relate UTxO states to toggle states
% wherein the toggle contract has already been initialized, ie. thread token(s) minted
% and stored in a correctly-defined UTxO. We fix $\mathsf{myRef}$ to some specific
% output reference. We also instantiate the thread tokens as follows :
%

%
% \textbf{Naive implementation relations. }
%
% In the naive implementation, a ledger state is related to a $\TOGGLE$ state
% whenever (i) the $\TOGGLE$ state is a pair booleans where one is the negation of the
% other, and (ii) there is exactly one output on the ledger containing the thread token,
% and locked by the correct validator, and with a datum that decodes to the same pair of booleans.
%

%
% Next, we define the projections :
%

%
% \textbf{Distributed implementation relation. }
%
% In the distributed implementation, a ledger state is related to a $\TOGGLE$ state
% whenever there are exactly two outputs on the ledger each containing one of the
% the two thread tokens, $\fun{tta}$ and $\fun{ttb}$,
% locked by the correct validator, each with a datum that decodes to the the corresponding
% boolean in the $\TOGGLE$ state.
%

%
% The function $\pi_{\TOGGLE}$ is defined in the same way as for the naive
% implementation, but with $\sim$ for the naive implementation. The transaction
% projection is :
%

%
% For both implementations, we again need to make a variation on the NFT
% re-minting protection assumption,
%

%


\textbf{$\TOGGLE$ property example. }
\[ (\emptytype, (a,b), i, (c, d)) \in \TOGGLE ~~\Rightarrow~~
(c, d) = (b, a) \vee (c, d) = (a, b) \]

This property states that in any step of $\TOGGLE$, either the state booleans are swapped,
or stay the same. Its proof is immediate from the specification, regardless of the
implementation.
% Since a
% structured contract is guaranteed to have only valid steps represented on the
% ledger,
% And, because we have demonstrated the $\sim >$ relation, a related property can be
% stated for all valid ledger transactions applied to
% any state $\var{utxo}$ such that $\var{(a, b)} = \pi ~\var{utxo}$ :
%
% \[ \forall \var{utxo} \in \UTxO, (\wcard, \var{utxo}, \var{tx}, \var{utxo'}) \in \LEDGER ~~\Rightarrow~~
% \pi~\var{utxo} = \pi~\var{utxo'} \vee \pi~\var{utxo} = ((\pi~\var{utxo'})_2, (\pi~\var{utxo'})_1) \]
%
% Both properties require a constraint to hold for every state, and therefore
% are safety properties \cite{liveness}.

% \input{properties.tex}
% \clearpage
\section{Related work}
\label{sec:related}

Code errors and design flaws have been very costly for users
since the introduction of smart contracts on the Ethereum platform \cite{survey}.
Among the most high-profile and costly being the DAO hack \cite{DAO},
and more recently, a faulty NFT contract \cite{nftbug}. Formal methods are being
used to find, prevent, and mitigate vulnerabilities on different ledger models, and via different
approaches \cite{formal}.

Scilla \cite{scilla} is a intermediate-level language for writing smart contracts as state
machines on an account-based ledger model.
The Scilla authors have used Coq to reason about contracts written in Scilla, proving a variety of
temporal properties such as safety, liveness, and others. The goal of this work,
however, not to study stateful program behaviour, but rather, it is to formalize
the notion of correct implementations
of stateful programs on a platform where programs are inherently stateless. The purpose of the
properties we discuss is to exemplify how reasoning about a
specification trace guarantees the conclusions to hold for any
ledger representation inducing a correct its implementation. A full treatment
of lifting safety and liveness properties from specification to implementation
is the subject of future work. Additionally, we believe that the subsystem approach
we present in this work may be used to model on-chain interactions between
Scilla contracts.

The Bitcoin Modelling Language (BitML) \cite{bitml} allows the definition of smart
contracts running on Bitcoin by means of a restricted class of state machines.
The BitML state machines are less expressive than the class of specifications
considered in our model, since we assume that our stateless scripts are written
in a Turing-complete language. However, the goal of this language is similar to
ours - to guarantee that the behaviour of certain state machines (in the case of BitML, ones defined
using this language) if in accordance with the changes made by valid transactions, i.e. soundness.
The difference is that we present a framework in which one can define a state
transition system with an implementation (both using a Turing-complete language),
then prove soundness to achieve a "correct" implementation, whereas BitML allows users to
define a sound state machine from smaller class, then compile it to a \emph{specific}
implementation. With BitML, LTL formulas can be automatically
verified using a dedicated model checker. In the future, we plan to add support for LTL formulas in
our framework.

VeriSolid \cite{verisolid} synthesises Solidity smart contracts from a state machine specification, and
verifies temporal properties of the state machine using CTL. The underlying
ledger model for VeriSolid is, however, an account-based model, rather than the
EUTxO model we work with. Moreover, in contrast
to the VeriSolid approach, our approach
relies on the contract author to themselves to synthesize an implementation
that meets the requirements specific to the contract being built, and then
provides a proof obligation to show that implementation is correct. This allows
for more flexibility in the implementation, as well as in the logic
used in checking properties. Here, again, it may be of interest to use an approach
similar to the structured contracts framework to model interactions between
structured contracts.

CoSplit, presented in \cite{sharding}, is a static analysis tool for implementing
\emph{sharding} in an account-based blockchain. Sharding is the act of
separating contract state into smaller
fragments that can be affected by commuting operations, usually for the purposes of
increasing parallelism and scalability.
Our work allows users to compose contracts whose state is distributed across multiple
UTxOs and tokens on the ledger, and provides a way to formally guarantee that the update of the
full aggregated state is in accordance with the update of its ledger representation.
This application of structured contracts serves a similar purpose as for the EUTxO
ledger as sharding does for an account-based one, as it can be used to increase
parallelism and scalability. We note that the UTxO model is a natural fit for
such state separation, since one of the benefits of such a ledger is that
all operations either commute or fail \cite{parallelism}. Therefore, any
UTxO state representation,
with any (correct) implementation, will afford the relevant properties for a given
distributed contract.

On the EUTxO ledger, the constraint-emitting machine design pattern \cite{eutxoma} makes
a formal correctness guarantee
similar to the proof obligation we require as part of the definition of smart contracts.
However, it is limited to a ledger representation of contract state that is
strictly the datum and value of a
single UTxO entry, expressing dependencies on other scripts via a limited set of
possible constraints on
transactions. Our model allows the contract state to be computed from multiple
UTxO entries and tokens aggregated across the ledger state, with its evolution
coordinated by multiple different scripts. Another notable difference is that
the ledger model presented in \cite{eutxoma} is a list of UTxO-stype transactions,
rather than the UTxO set itself, with a unique initial state (the empty ledger).
Here, we are not able to review and reason about the full transaction history,
as is the case for existing realistic ledgers.

The K framework \cite{kframework} is a unifying formal semantics framework for
all programming languages, which has been used as a tool to perform
audits of smart contracts \cite{runtimev}. QuckCheck, a property-testing
library, has also been applied for the purposes of auditing
existing stateful EUTxO-implemented contracts \cite{quviq}. Other formal
methods audits of individual contracts acknowledge the complexity and uniqueness
of Turing-complete code in the EUTxO model \cite{tweag}. While formal in nature,
these individual audits and services do not present an overarching principle of
EUTxO smart contract verification in their approaches.

\section{Applications}
\label{sec:applications}

We have presented a framework which can be used to formalize not only the contracts
themselves, but also the
problems in the domain of smart contract verification.
We discuss several applications of this general framework
to existing contract verification issues.
We give examples of the problems that are most promising to be addressed
with this new tool. Each of these requires an in-depth investigation, which is
outside the scope of this paper :

\paragraph{Properties. } Program verification involves analyzing and providing guaranees
about the behaviour of program. In our case, these programs are structured
contracts, as well as the ledger itself. Trace-based properties, such as liveness
and safety properties,
are the standard for analyzing program behaviour \cite{liveness}.
Applying and/or
adjusting the definitions and theorems about properties to be used in the context of
structured contracts and ledgers will give high-assurance guarantees about
the behaviour of both \cite{properties}. Moreover, the subsystem relation in the SCF allows
for establishing a correspondence between ledger and contract properties via the
state projection function. This may give additional behaviour guarantees for
a contract's ledger representations.

\paragraph{Double satisfaction. } Double satisfaction is a situation in which
a single action performed by a transaction on the ledger satisfies the constraints
of more than one script being executed. This is a very broad, informal description,
and the situation it describes may not be a problem.
However, there are situations in which it would be - such as when a single payout
made by a transaction to a given address satisfies the payout requirements of
two distinct scripts run by the transaction.

So far, there has not been any formal treatment of exactly when double satisfaction
may be undesirable, even if it is a fairly intuitive answer in most cases.
The SCF may present a solution to making precise this distinction : it provides
a way to separate transaction \emph{constraint checking} from \emph{state updates}.
For example, one may include in a specification state the collection of pay-ins to be
consumed by the contract and pay-outs to be collected by the intended recipient.
This ensures that any correct implementation will mark such payments as made for or
by a \emph{specific contract instance}, thus mitigating problematic double satisfaction.

This approach of marking some data or assets as "for a specific contract, and from
a specific contract" is a
scheme that can be described as a kind of \emph{message-passing} \cite{messages}.
We can implement
it as and instance of the SCF. We can also use this same scheme for the following
challenge to tackle via the SCF:

\paragraph{Asynchronous or partial contract execution. } Asynchronous contract
execution refers to dependence between scripts implemented in a particular way.
It is achieved
by relaxing the requirement that dependent scripts must be executed within the same transaction,
in two distinct ones. We can allowing one script to execute a step,
simultaneously constructing a kind of proof artefact of its
execution on a given input. In the second transaction, which consumes the artefact,
the dependent script executes using the consumption of the artefact as proof of
the first script’s validation within a prior transaction.
This second script can behave as if the first contract was executed within the same transaction
with that specific input.
This type of scheme may be useful, for example, as a way to transfer assets
between contracts without having to run them both in a single transaction.

Another application of this scheme is allowing contracts whose code is "too large"
to be run in a single transaction to be split up into what is effectively
\emph{function calls}. Since message-passing records the contract that generated
a given message, as well as the input that contract was given, messages can
be used as artefacts of computations of function calls \cite{messages}.

\paragraph{Eliminating dependencies on opaque scripts. }
A script may include a constraint requiring another script to be run
within the same transaction, e.g. a particular token to be minted,
or some UTxO to be spent. Those scripts, may, in turn, contain additional
constraints requiring yet more scripts to be run. The SCF could allow us to
define ledger subsystems in which guarantees can be made about what scripts
will be required to run when the subsystem executes any step. While a good
implementation of a structured contract will intuitively not depend on any unnecessary
script executions, we may now be able to formalize this property.

Eliminating reliance on the validation of opaque scripts to advance a contract
state on the ledger is an important goal, especially in the context of proving
liveness properties such as liquidity. Achieving it is a step towards being
able to guarantee the existence of a valid ledger transaction corresponding to
each step in the specification of a contract state transition.


\section{Conclusion}
\label{sec:conclusion}

We have presented a novel approach to specifying and reasoning about behaviour of
stateful programs running on a EUTxO ledger, which we call the structured contract
formalism. Our formalism defines a robust
way to relate stateless predicate scripts executed at the ledger level to the
corresponding corresponding executions of a specific stateful program.
We used the well-established concept of a subsystem
to define this relation, and a small-step semantics style already in use in
existing systems (i.e. Cardano) for the specification of the ledger and contracts.
This work presents a broadly applicable and principled
way of reasoning about stateful programs on the EUTxO ledger.

This paper lays the groundwork for treating ledger-implemented programs as
formal subsystems of the ledger. This is done on a level that is very specific
to the details of the EUTxO ledger. In the future, we aim to generalize our
research to be applicable to other kinds of ledger transition systems.
We would also like to better align our findings with existing concepts in the
theory of simulation, concurrency, and distributed computation, in order to
apply the full gamut of results in those areas for studying stateful
programs on the ledger.

Another goal for future work is to
find a way to use the findings presented here to verify existing contracts.
This may be done by first constructing the ledger and transaction representation
projection functions, then building a state transition specification induced by
$\LEDGER$ for those projections. Properties of the resulting system can then
be studied.

A full mechanization in Agda of the results and definitions in this work is currently
under way, as it is a natural next step. We also intend to build a mechanization of this work
integrated with the Agda mechanization of a more sophisticated and realistic
ledger, the Cardano ledger. The Cardano
ledger is in a unique position to be amenable to the structured contract framework
approach to verification due to the existence of a mechanized small-steps specification
in Agda of the entire ledger (currently in development) \cite{agdaspec}.
Agda automation of proof generation for the subsystem proof obligations, such
as using an SMT-solver, will be a natural progression of the project.

% Another
% aspirational aspect of automation of this project is that of the translation
% from contract implementations in Agda to Plutus, the language used in the
% between Agda contract implementations and Plutus contracts

\appendix
\section{Appendix}

Notation :

\begin{figure}[htb]
  \begin{align*}
    \emptytype
    & :~\{\emptytype \}
    & \text{the one-element set, and its one inhabitant}
    \\
    \fun{fst}
    & :~(A \times B) \to A
    & \text{first projection}
    \\
    \var{Key} \mapsto \var{Value}
    & \subsetneq \{~ k \mapsto v ~\mid~ k \in \var{Key},~v \in \var{Value}~ \}
    & \text{finite map with unique keys}
    \\
    [~\fun{f}~\var{b}~\mid~\var{b} ~\leftarrow ~\var{myList}~]
    & :~[C]
    & \text{list comprehension, given $\fun{f} : B \to C$}
    \\
    \fun{map}
    & :~(A \to B) \to [A] \to [B]
    & \text{apply $\fun{map}$ to every element in given list}
    \\
    \fun{map}
    & :~(A \to B) \to \powerset{A} \to \powerset{B}
    & \text{apply $\fun{map}$ to every element in given set}
    % \\
    % \var{map}~ (\var{k})
    % & = v
    % & \text{where }~k~\mapsto~v~\in~\var{map}
    % \\
    % \var{set} \restrictdom \var{map}
    % & = \{ k \mapsto v \mid k \mapsto v \in \var{map}, ~ k \in \var{set} \}
    % & \text{domain restriction}
    % \\
    % \var{set} \subtractdom \var{map}
    % & = \{ k \mapsto v \mid k \mapsto v \in \var{map}, ~ k \notin \var{set} \}
    % & \text{domain exclusion}
    \\
    % \var{MyType}^?
     & \var{MyType}~ \cup ~\emptytype
    & \text{maybe type}
    % \\
    % \var{map} \restrictrange \var{set}
    % & = \{ k \mapsto v \mid k \mapsto v \in \var{map}, ~ v \in \var{set} \}
    % & \text{range restriction}
    % \\
    % \var{map} \subtractrange \var{set}
    % & = \{ k \mapsto v \mid k \mapsto v \in \var{map}, ~ v \notin \var{set} \}
    % & \text{range exclusion}
    % \\
    % A \triangle B
    % & = (A \setminus B) \cup (B \setminus A)
    % & \text{symmetric difference}
    % \\
    % M \unionoverrideRight N
    % & = (\dom N \subtractdom M)\cup N
    % & \text{union override right}
    % \\
    % % M \unionoverrideLeft N
    % % & = M \cup (\dom M \subtractdom N)
    % % & \text{union override left}
    % \\
    % M \unionoverridePlus N
    % & = (M \triangle N)
    % \cup \{k\mapsto v_1+v_2\mid {k\mapsto v_1}\in M \land {k\mapsto v_2}\in N \}
    % & \text{union override plus} \\
    % & & \text{(for monoidal values)}\\
    % \\
    % M \unionoverrideMinus N
    % & = (M \triangle N)
    % \cup \{k\mapsto v_1-v_2\mid {k\mapsto v_1}\in M \land {k\mapsto v_2}\in N \}
    % & \text{union override minus} \\
    % & & \text{(for group values)}\\
    % \seqof{\powerset{T}}
    % \\
    % \llbracket s \rrbracket
    % & = \fun{hash}~s
    % & \text{hash of $s$} \\
    % &
    % & \text{power-multi-set of type $T$}
  \end{align*}
  \caption{Notation}
  \label{fig:notation:nonstandard}
\end{figure}

% \begin{figure}[htb]
%   \begin{align*}
%     \fun{toMap}~ &:& \Ix \to [\Output] \to (\Ix \mapsto \Output) \\
%     \fun{toMap}~\wcard~[] &=& [~] \\
%     \fun{toMap}~\var{ix}~(u~ ::~\var{outs}) &=& \{~\var{ix}\mapsto u~\} \cup \{~(\fun{toMap}~(\var{ix}+1)~\var{outs})~\}\\
%     \nextdef
%     \fun{mkOuts}~ &:& \Tx \to \UTxO \\
%     \fun{mkOuts}~{tx} &=& \{~(\var{tx},~\var{ix}) \mapsto o~ \mid~(\var{ix} \mapsto o)\in~\fun{toMap}~0~(\outputs~\var{tx})~\}\\
%     \nextdef
%     \fun{getORefs}~&~:& \Tx \to \Set{\OutputRef} \\
%     \fun{getORefs}~{tx} &=& \{~\outputref~i~\mid~i~\in~\inputs~\var{tx} ~\} \\
%   \end{align*}
%   \caption{Auxiliary UTxO functions}
%   \label{fig:utxo-func}
% \end{figure}



\begin{ruledfigure}{t}
  \begin{displaymath}
    \begin{array}{rll}
      \multicolumn{3}{l}{\textsc{Basic types}}\\
     \B, \N, \Z && \mbox{the type of Booleans, natural numbers, and integers}\\
      \H{} && \mbox{the type of bytestrings: } \bigcup_{n=0}^{\infty}\{0,1\}^{8n}\\
      % (\phi_1 : T_1, \ldots, \phi_n : T_n) && \mbox{a record type with fields $\phi_1, \ldots, \phi_n$ of types $T_1, \ldots, T_n$}\\
      % \phi~\var{t} && \mbox{the value of $\phi$ for $t$, where $t$ has type $T$ and $\phi$ is a field of $T$}\\
      \powerset{T} && \mbox{the type of (finite) sets over $T$}\\
      $[T]$ && \mbox{the type of lists over $T$, with $\_[\_]$ as indexing and $|\_|$ as length}\\
      h::t && \mbox{the list with head $h$ and tail $t$}\\
%      x \mapsto f(x) && \mbox{an anonymous function}\\
%      \hash{c} && \mbox{a cryptographic collision-resistant hash of $c$}\\
      \Interval{A} && \mbox{the type of intervals over a totally-ordered set $A$}\\
      \FinSup{K}{M} && \mbox{the type of finitely supported functions from a type $K$ to a monoid $M$}\\
      \\
      \multicolumn{3}{l}{\textsc{Ledger primitives}}\\
      \Quantity = \Z && \mbox{an amount of an assets}\\
      \TokenName = [\type{Char}] && \mbox{token name string}\\
      \AssetID = \Policy \times \TokenName && \mbox{unique asset identifier}\\
      \Coin \in \AssetID && \mbox{asset ID of the primary currency }\\
      \Tick && \mbox{slot number representing chain time}\\
%      \Address && \mbox{an ``address'' in the blockchain}\\
      \Data && \mbox{a type of structured data}\\
%      \DataHash && \mbox{the hash of a value of type \Data{}}\\
      % \hashData : \Data \rightarrow \DataHash && \mbox{computes the hash of an value of type \Data}\\
%      \TxId && \mbox{the identifier of a transaction}\\
%      \txId : \eutxotx \rightarrow \TxId && \mbox{computes the identifier of a transaction}\\
%      \lookupTx : \Ledger \times \TxId \rightarrow \eutxotx{} && \mbox{retrieves the unique transaction with a given identifier}\\
      \Script && \mbox{the (opaque) type of scripts}\\
      \applyScript{\_}: \Script \rightarrow \Datum \times \Redeemer \times \vlctx \rightarrow \B && \mbox{applies a script to its arguments}\\
      \applyMPScript{\_}: \Script \rightarrow \Redeemer \times \mpsctx \rightarrow \B && \mbox{applies a script to its arguments}\\
      \checkSig : \eutxotx \to \pubkey \to \H \to \B && \mbox{checks that the given PK signed the transaction (excl. signatures)}\\
%      \scriptAddr : \Script \rightarrow \Address && \mbox{the address of a script}\\
\\
    \multicolumn{3}{l}{\textsc{Defined types}}\\
    \Ix  &=& \N\\
    \Policy  &=& \Script\\ %\Address
    \Redeemer  &=& \Data\\
    \Datum  &=& \Data\\
    \Signature &=& \pubkey \mapsto \H\\
    \\
    \Value   &=& \FinSup{\Policy}{\FinSup{\TokenName}{\Quantity}}\\
    % \\
    % \TxOut &=&(\addr: \Address, \val: \Value, \fun{datum}: \Datum)\\ %\datumHash: \DataHash)\\
    \\
    \OutputRef &= &(\txrefid: \Tx, \idx: \Ix)\\ %\TxId
    \\
    \Output &=&( \validator: \Script,\\
                & &\ \val : \Value, \\
                & &\ \datum: \Data )\\
    \\
    \Input &=&( \outputref: \sf{OutputRef},\\
                & &\ \fun{output}: \Output, \\
                & &\ \redeemer: \Redeemer)\\
    \\
    \eutxotx &=&(\inputs: \powerset{\Input},\\
               & &\ \outputs: [\Output],\\
               & &\ \fun{validityInterval}: \Interval{\Tick},\\
               & &\ \mint: \Value,\\
               & &\ \mintScsRdmrs: \Script \mapsto \Redeemer,\\
               & &\ \sigs: \Signature)\\
    \\
    \UTxO &=&\ \OutputRef \mapsto \Output \\
    \end{array}
  \end{displaymath}
  \caption{Primitives and basic types for the \EUTXOma{} model}
  \label{fig:eutxo-types}
\end{ruledfigure}
%
\begin{ruledfigure}{t}
  \begin{displaymath}
  \begin{array}{rll}
    % \s{OutputInfo}\s{ } &=&(\val: \Value,\\
    %                       & &\ \i{validatorHash}: \Address,\\
    %                       & &\ \datumHash: \DataHash)\\
    % \\
    % \s{InputInfo}\s{ } &=& (\outputref: \s{OutputRef},\\
    %                      & &\ \i{validatorHash}: \Address,\\
    %                      & &\ \i{datumHash}: \DataHash,\\
    %                      & &\ \i{redeemerHash}: \DataHash,\\
    %                      & &\ \s{value}: \Value)\\
    % \\
    %  \s{TxInfo}\s{ } &=&(\fun{inputInfo}: \List{\Input},\\
    %              & &\ \fun{outputInfo}: \List{\Output},\\
    %              & &\ \fun{validityInterval}: \Interval{\Tick},\\
    %              & &\ \mint: \Value,\\
    %              & &\ \mintRdmrs: \Policy \mapsto \Redeemer,\\
    %              & &\ \sigs: \FinSet{\pubkey})\\
    % \\
    \fun{toMap}~ &:& \Ix \to [\Output] \to (\Ix \mapsto \Output) \\
    \fun{toMap}~\wcard~\{\} &=& [~] \\
    \fun{toMap}~\var{ix}~[u;~\var{outs}] &=& \{~\var{ix}\mapsto u~\} \cup \{~(\fun{toMap}~(\var{ix}+1)~\var{outs})~\}\\
    \nextdef
    \fun{mkOuts}~ &:& \Tx \to \UTxO \\
    \fun{mkOuts}~{tx} &=& \{~(\var{tx},~\var{ix}) \mapsto o~ \mid~(\var{ix} \mapsto o)\in~\fun{toMap}~0~(\outputs~\var{tx})~\}\\
    \nextdef
    \fun{getORefs}~&~:& \Tx \to \powerset{\OutputRef} \\
    \fun{getORefs}~{tx} &=& \{~\outputref~i~\mid~i~\in~\inputs~\var{tx} ~\} \\
    \fun{getORefs}_U~{tx} &=& \{~\outputref~i~\mid~i~\in~\inputs~\var{tx},~U~(\outputref~i~\mapsto~\fun{output}~i) ~\} \\
    \nextdef
    \s{\vlctx}\s{ } &=& (\Tx, (\Tx,~\Input)) \\
    \s{\mpsctx}\s{ } &=& (\Tx, \Policy) \\
    \\
    % \mkTxInfo: \eutxotx \times \UTxO & \rightarrow &\Tx\\
    %                  & &   \mbox{\parbox[t]{55mm}{summarises a transaction}}\\
 %    \mkVlContext &:& \eutxotx \to \OutputRef \rightarrow \vlctx\\
 %                     & &   \mbox{\parbox[t]{55mm}{summarises a transaction for a validator script in the context of an input and a ledger state}}\\
 % \\
 %    \mkMpsContext &:& \eutxotx \to \Policy \rightarrow \mpsctx\\
 %                     & &   \mbox{\parbox[t]{55mm}{summarises a transaction for a minting policy script in the context of an currency and a ledger state}}\\
  \nextdef
  \fun{oneT} &:& \Policy \to \TokenName \to \Value \\
  \fun{oneT}~ p~n &\leteq& \{ ~p \mapsto \{~n~ \mapsto 1~\}~\} \\
  \end{array}
  \end{displaymath}
  \caption{Auxiliary functions for entering outputs into the UTxO set}
  \label{fig:ctx-types}
\end{ruledfigure}

\begin{ruledfigure}{t}
\begin{enumerate}
  \item
    \label{rule:has-input}
    \textbf{Transaction has at least one input}
    \begin{displaymath}
      \inputs~\var{tx}~\neq~\{\}
    \end{displaymath}

  \item
    \label{rule:slot-in-range}
    \textbf{The current slot is within the validity interval}
    \begin{displaymath}
      \var{slot} \in \fun{validityInterval}~\var{tx}
    \end{displaymath}

  \item
    \label{rule:all-outputs-are-positive}
    \textbf{All outputs have positive values}
    \begin{displaymath}
      \forall o \in \outputs~\var{tx},\ \val~\var{o} > 0
    \end{displaymath}

  \item
    \label{rule:all-inputs-refer-to-unspent-outputs}
    \textbf{All inputs refer to unspent outputs}
    \begin{displaymath}
      \forall (\var{oRef},~\var{o})~\in~\{(\outputref~\var{i},~\fun{output}~\var{i}) ~|~ i \in \inputs~\var{tx} \},
      ~\var{oRef}~ \mapsto~\var{o}~ \in ~\var{utxo}
    \end{displaymath}

  \item
    \label{rule:value-is-preserved}
    \textbf{Value is preserved}
    \begin{displaymath}
      \mint~\var{tx} + \sum_{i \in \inputs~\var{tx},~(\outputref~i)\mapsto~o~\in~\var{utxo}} \val~\var{o} = \sum_{o \in \outputs~\var{tx}} \val~\var{o}
    \end{displaymath}

  \item
    \label{rule:no-double-spending}
    \textbf{No output is double spent}
    \begin{displaymath}
      \textrm{If } i_1, i \in \inputs~\var{tx} \textrm{ and } \fun{fst}~ (\outputref~\var{i}) = \outputref~\var{i}
      \textrm{ then } \fun{fst}~i~ = i.
    \end{displaymath}

  \item
    \label{rule:all-inputs-validate}
    \textbf{All inputs validate}
    \begin{displaymath}
      \textrm{For all } i \in \inputs~\var{tx},\ \applyScript{\validator~\var{i}}(\datum~\var{i},\, \redeemer~\var{i},\, (\var{tx}, i)) = \true
    \end{displaymath}
  %
  % \item
  %   \label{rule:validator-scripts-hash}
  %   \textbf{Validator scripts match output addresses}
  %   \begin{displaymath}
  %     \textrm{For all } i \in tx.\inputs,\ \scriptAddr(i.\validator) = \var{utxo}(i).\addr
  %   \end{displaymath}
  %
  % \item
  %   \label{rule:datum-objects-hash}
  %   \textbf{Datum objects match output hashes}
  %   \begin{displaymath}
  %     \textrm{For all } i \in tx.\inputs,\ \hashData(i.\datum) = \var{utxo}(i).\datumHash
  %   \end{displaymath}
%
  \item
    \label{rule:forging}
    \textbf{Minting redeemers present}
    \begin{displaymath}
      \forall ~\var{pid}~ \in \supp(\mint~\var{tx}), ~~\exists
      (\var{pid},\wcard) ~\in~ \mintScsRdmrs~\var{tx}
    \end{displaymath}
  \medskip % items jammed together without this
  \item
    \label{rule:all-mpss-run}
    \textbf{All minting policy scripts validate}
    \begin{displaymath}
      \textrm{For all } (s, \var{rdmr}) \in \mintScsRdmrs~\var{tx},\ \applyMPScript{s}(\var{rdmr}, (tx, s)) = \true
    \end{displaymath}
  \medskip % items jammed together without this

  \item
    \label{rule:sigs-ok}
    \textbf{All signatures are correct}
    \begin{displaymath}
      \textrm{For all } (pk \mapsto s) \in \sigs~\var{tx},\ \checkSig(\var{tx}, pk, s) = \true
    \end{displaymath}

\end{enumerate}
\caption{Validity of a transaction $t$ in the \EUTXOma model}
\label{fig:validity}
\end{ruledfigure}
%

\begin{ruledfigure}{t}
  \begin{align*}
    \sum_{\var{or}\mapsto \var{out}\in \var{utxo'}}~\val~\var{out}
    & = ~\sum_{\var{or}\mapsto \var{out}\in ((\fun{getORefs}~\var{tx}) \subtractdom \var{utxo}) \cup \fun{mkOuts}~{tx}}~\val~\var{out} \\
    & = ~\sum_{\var{or}\mapsto \var{out}\in \var{utxo}}~\val~\var{out}
    ~-~\sum_{i \in \inputs~\var{tx},~(\outputref~i\mapsto~\var{out})\in \var{utxo}}~\val~\var{out}
    ~+~\sum_{\var{out} \in \outputs ~\var{tx}}~\val~\var{out} \\
    & = ~\sum_{\var{or}\mapsto \var{out}\in \var{utxo}}~\val~\var{out}
    ~-~\sum_{i \in \inputs~\var{tx},~(\outputref~i\mapsto~\var{out})\in \var{utxo}}~\val~\var{out}
    ~+~(\mint~\var{tx}) \\
    &~~~~~~~ + \sum_{i \in \inputs~\var{tx},~(\outputref~i\mapsto~\var{out})\in \var{utxo}}~\val~\var{out} \\
    & = ~\sum_{\var{or}\mapsto \var{out}\in \var{utxo}}~\val~\var{out}  ~+~(\mint~\var{tx})
  \end{align*}
\caption{Proof of the $\sim >$ constraint for POV}
\label{fig:pov-pf}
\end{ruledfigure}

\paragraph{$\sim >$ proof sketch for $\fun{NFT}$. }
\label{pf:nft}
Suppose $(\var{slot},~\var{utxo},~\var{tx},~\var{utxo'}) \in \LEDGER$.
\textbf{$\pi~\var{utxo} = 0$ : } By $\fun{hasRef}$, $\fun{myOut} \in \var{utxo}$.
If $i = 0$, the NFT is not minted, and $0 = s = s'$, and indeed
$(\emptytype, 0, \var{tx}, 0) \in \\fun{NFT}$. If $i \neq 0$, necessa

 $\fun{myOut}$ is spent, and $\fun{myOut}$ can only be
spent if $\fun{myNFT}$ is minted. Note here that in fulfilling the proof obligation
of the $\fun{NFT}$ structured contract, we require to show that the output reference
$\fun{myNFTRef}$ can never be added to the UTxO by $\var{tx}$, making it possible
to mint a second copy of $\fun{myNFT}$.

\begin{figure}
\begin{displaymath}
  \begin{array}{rll}
  \fun{toData}_{N} &:& (\B \times \B) \to \Data \\
  \fun{toData}_{D} &:& \B \to \Data \\
  \fun{fromData}_{N} &:& \Data \to (\B \times \B)\\
  \fun{fromData}_{D} &:& \Data \to \B \\
\end{array}
\end{displaymath}
\caption{Encoding and decoding $\TOGGLE$ script datums}
\label{fig:to-data}
\end{figure}



%%
%% Bibliography
%%

%% Please use bibtex,

\bibliography{structured-contracts-bib}

\appendix


\end{document}
