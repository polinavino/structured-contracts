\section{Simulations and the structured contract formalism}
\label{sec:struc}

The formal semantics discussed can be used to specify other stateful programs
besides $\LEDGER$. The programs we are interested in specifying for the purpose of this work
are those that \emph{run on the ledger}. Intuitively, a stateful program runs on the
ledger whenever its state is observable within the ledger state, and is always
updated by a transaction in accordance with the program's specification.
We now formalize the notion of correctly implementing
a stateful program on the ledger.

We note that the only tool for implementing programs on the ledger are smart
contract scripts. A smart contract script encodes the conditions under which
a transaction can update a part of the ledger state with which the script is
associated, e.g. change the total quantity of tokens under a given policy, or
remove some UTxO entries. For this reason,
we propose a \emph{stateful} model for specifying programs running on the
ledger. Such programs are implemented via a one or more
interacting scripts controlling the updates of the corresponding parts of the UTxO state.
The state of a program on the ledger is \emph{observed} by applying a projection
function to the ledger state which aggregates the relevant data.

We present several varied examples to demonstrate this.
In this section, the first example is not, in fact, a script-implemeneted structured
contract, but rather the specification of an existing ledger feature, the preservation
of value. We then present an interpretation of an NFT as structured contract,
as a way to demonstrate that minting policies constitute
a way to control observable parts of the ledger state.

\subsection{Simulations.}

We to formalize the notion of a program running on the ledger, we first state the
definition a \emph{simulation} \cite{milner-pibook}, instantiating it with lebelled
state transition systems expressed as small-steps semantics specifications.

\paragraph{Simulation definition.}
\label{def:simulation}

Let $\TRANS$ and $\STRUC$ be small-step labelled transition systems.
A \emph{simulation} of $\TRANS$ in $\STRUC$, denoted by
\[ (\STRUC, \sim_{\TRANS,\STRUC}, \sim_{\to,\TRANS,\STRUC}) \succeq \TRANS \]

Consists of of the following types together with the following relations :
\begin{equation*}
  \_ \vdash
  \var{\_} \trans{trans}{\_} \var{\_}
  \in \powerset (\Env_{\TRANS} \times \State_{\TRANS} \times \Input_{\TRANS} \times \State_{\TRANS})
\end{equation*}
\begin{equation*}
  \_ \vdash
  \var{\_} \trans{struc}{\_} \var{\_}
  \in \powerset (\Env_{\STRUC} \times \State_{\STRUC} \times \Input_{\STRUC} \times \State_{\STRUC})
\end{equation*}
\begin{align*}
  \wcard~\sim_{\TRANS,\STRUC}~\wcard ~&:~ \State_{\TRANS} \times \State_{\STRUC} \to \Bool \\
  \wcard~\sim_{\to,\TRANS,\STRUC} ~\wcard ~&:~ (\Env_{\TRANS} \times \Input_{\TRANS}) \times (\Env_{\STRUC} \times \Input_{\STRUC}) \to \Bool
\end{align*}

such that the following holds :
  \begin{equation}
    \inference[$\sim >$]
    {
      \\~\\
      (e,~i) \sim_{\to,\TRANS,\STRUC} (e', j) & u~\sim_{\TRANS,\STRUC}~s
      \\~\\
      {
        \begin{array}{c}
          e\\
        \end{array}
      }
      \vdash
      {
        \left(
          \begin{array}{r}
            \var{s} \\
          \end{array}
        \right)
      }
      \trans{trans}{\var{i}}
      {
        \left(
          \begin{array}{r}
            \var{s'} \\
          \end{array}
        \right)
      }
      \\~\\
    }
    {
      u'~\sim_{\TRANS,\STRUC}~s' ~~~~~
      \begin{array}{r}
        e'\\
      \end{array}
    \vdash
      (u)
      \trans{struc}{j}
      (\varUpdate{u'})
    }
  \end{equation}

  %
  % \begin{equation}
  %   \inference[$\sim >$]
  %   {
  %     \\~\\
  %     {
  %       \begin{array}{c}
  %         \var{slot}\\
  %       \end{array}
  %     }
  %     \vdash
  %     {
  %       \left(
  %         \begin{array}{r}
  %           \var{utxo} \\
  %         \end{array}
  %       \right)
  %     }
  %     \trans{ledger}{\var{tx}}
  %     {
  %       \left(
  %         \begin{array}{r}
  %           \var{utxo'} \\
  %         \end{array}
  %       \right)
  %     }
  %     \\~\\
  %   }
  %   {
  %     \begin{array}{r}
  %       \emptytype \\
  %     \end{array}
  %   \vdash
  %     (\pi~\var{utxo})
  %     \trans{struc}{\pi_{\Tx}~\var{tx}}
  %     (\pi~\varUpdate{\var{utxo'}})
  %   }
  % \end{equation}

We write $\sim$ instead of $\sim_{\TRANS,\STRUC}$ whenever the subscript is
unambiguous. Note that \ref{def:simulation} includes a \emph{proof obligation} that must be
fulfilled as part of the definition, which \emph{does not define a rule}.
One can construct pairs of transition
systems with $\sim,~\sim_{\to}$ relations for which this proof obligation cannot
be fulfilled. However, \emph{if} it is possible, we have a simulation of $\TRANS$
in $\STRUC$.

% Note here that, given a contract state $s$ such that $\var{utxo} \sim s$,
% the only entries in the UTxO set are ones that contain some part of the
% representation of the state $s$. Then, any transaction applied to $\var{utxo}$
% will necessarily change some part of the state representation, and therefore
% correspond to a non-trivial state update.


\subsection{Structured contracts.}
\label{sec:struc-def}

The simulation definition we give is general, however, the rest of this work is
geared towards reasoning about the programmable
parts of the ledger, ie. those where the permissions are controlled by user-defined
scripts. For this reason, we define a particular class of simulations of $\LEDGER$.

A \emph{structured contract} $\STRUC$ is a transition system such that
$(\STRUC, \sim_{\LEDGER,\STRUC}, \sim_{\to,\LEDGER,\STRUC}) \succeq \LEDGER $,
where $\Env_{\STRUC} \leteq \emptytype$, and
for some functions $\pi_{\STRUC} : \UTxO \to \State_{\STRUC}^?$ and $\pi_{\Tx,\STRUC} : \Tx \to \Input_{\STRUC}$, the relations $\sim$ and $\sim >$ are given by :
\begin{align*}
  \var{utxo}~\sim~s &\leteq (\pi_{\STRUC}~\var{utxo} ~=~s)
  \nextdef
  (\var{slot},~\var{tx})~\sim_{\to,\LEDGER,\STRUC}~ (\emptytype, i) &\leteq (\pi_{\Tx,\STRUC}~\var{tx} = i)
\end{align*}

This definition states that whenever for a given start state $\var{utxo}$
with a corresponding contract start state $\pi~\var{utxo} \neq \emptytype$,
each valid step $(\var{slot}, \var{utxo}, \var{tx}, \var{utxo'}) \in \LEDGER$
at the ledger level necessarily has a corresponding valid
step at the contract level,
$(\emptytype, \pi~\var{utxo}, \pi_{\Tx}~\var{tx}, \pi~\var{utxo'}) \in \STRUC$.
We sometimes denote both $\pi_{\Tx,\STRUC}$ and $\pi_{\STRUC}$
by $\pi$ or $\pi_{\Tx}$ when the context is clear.
We also denote the structured contracts by
\[ (\STRUC,~\pi_{\STRUC},~\pi_{\Tx,\STRUC}) \succeq \LEDGER\]

There may be transactions that update the ledger state, but not the $\STRUC$ state.
There is no a special case for this in the simulation relation, and the $\STRUC$
specification rules must allow trivial steps if such ledger transactions are possible.
The $\STRUC$ specification may not itself be deterministic, however, since the
$\LEDGER$ is deterministic, given a ledger
step, there is a unique step in $\STRUC$ corresponding to the ledger one.

The purpose of structured contracts is to represent all ledger
simulations that can be defined and implemented via scripts. Since scripts
cannot inspect any ledger environment data (i.e. the slot number), structured contracts
have a trivial environment.
The reason that the $\pi_{\State}$ a (partial) \emph{function} is that it must
be possible to extract a \emph{unique} contract state from the ledger state
to meaningfully observe the contracts execution trace corresponding to the ledger execution.
This function is \emph{partial} because a valid
$(\var{slot}, \var{utxo}, \var{tx}, \var{utxo'}) \in \LEDGER$ may have a start
state $\var{utxo}$ that violates some constraints necessary
to obtain a valid corresponding contract state. For example, the $\var{utxo}$
state may have multiple copies of what
was supposed to be an NFT (and therefore unique), and therefore does not map onto
the state of a contract when that state is the datum of a UTxO entry
containing that NFT.

Consider now the problem constructing a bisimulation \cite{milner-pibook}, which
requires that for a given step in a structured
contract, there exists a corresponding valid ledger transition step. Finding tuple
representing such a step requires ensuring that all constraints in $\fun{checkTx}$
are satisfied, including
the validation of all dependent scripts. Defining a class of structured contracts for which
this is possible is a difficult problem, and we leave it for future work.

\section{Minting and preservation of value as structured contracts}
\label{sec:pov}

In this section we give an example of using the structured contract
formalism to express global invariants of the ledger that
can be derived from local invariants.
The EUTxO ledger transition system $\LEDGER$ is said to satisfy the
\emph{preservation of value (POV)} property. A \emph{local} POV is
the property that each valid transaction mints exactly the difference
between the value in the UTxOs in spends and those it creates.
This is ensured by the ledger rule \ref{rule:value-is-preserved}, which is
\[ \mint~\var{tx} + \sum_{i \in \inputs~\var{tx},~(\outputref~i)\mapsto~o~\in~\var{utxo}} \val~\var{o} = \sum_{o \in \outputs~\var{tx}} \val~\var{o} \]

Suppose that $(\var{s},~\var{utxo},~\var{tx},~\var{utxo'}) \in \LEDGER$, and
recall that updated UTxO set is defined as follows
\[ \var{utxo'}~\leteq~(\fun{txins}~{tx} \subtractdom \var{utxo}) \cup \fun{mkOuts}~{tx} \]

The \emph{global} POV, which can be proved from the local POV,
states that the sum total of all the tokens
recorded on the ledger is changed by exactly the amount minted or burned by the
transaction,
\[ \sum_{\var{or}\mapsto \var{out}\in \var{utxo'}}~\val~\var{out} ~= ~(\sum_{\var{or}\mapsto \var{out}\in \var{utxo}}~\val~\var{out}) ~+~(\fun{mint}~ \var{tx}) \]

We can express this property via the following structured contract, $\mathsf{POV}$ : \newline

% \begin{figure}[htb]
\emph{Transition type}
  \begin{equation*}
    \_ \vdash
    \var{\_} \trans{pov}{\_} \var{\_}
    \in \powerset (\emptytype \times \Value \times \Tx \times \Value)
  \end{equation*}
  %
\emph{Simulation relations}
  \begin{align*}
    \pi_{\Tx,\POV}~\var{tx}~ ~&\leteq~\var{tx}
    \nextdef
    \pi_{\POV}~\var{utxo}~&\leteq~\sum_{\wcard \mapsto \var{out}\in \var{utxo}}~\val~\var{out}
  \end{align*}
  %
\emph{Transition rule}
  \begin{equation}
    \inference[UpdateValTotal]
    {
    i \leteq \fun{mint}~\var{tx}
    \\ ~ \\
    }
    {
    \begin{array}{l}
      \\
    \end{array}
      \vdash
      \left(
      \begin{array}{r}
        \var{s} \\
      \end{array}
      \right)
      \trans{pov}{tx}
      \left(
      \begin{array}{r}
        \varUpdate{\var{s}~+~\var{i}}  \\
      \end{array}
      \right) \\
    }
  \end{equation}
%   \caption{Specification of the $\mathsf{UpdateValTotal}$ POV subsystem}
%   \label{fig:pov-spec}
% \end{figure}

The value is
changed exactly by the amount specified in the mint field, so that the $\sim >$ proof
obligation corresponds exactly to the global POV formulation.
To complete the definition, it remains to prove
$\sim >$, which requires that, given that
$(\var{slot},~\var{utxo},~\var{tx},~\var{utxo'}) ~\in \LEDGER$,
necessarily $(\emptytype,~\pi~\var{utxo},~\pi~\var{tx},~\pi~\var{utxo'}) \in \POV$.

So, we must show that
\[ \sum_{\wcard \mapsto \var{out}\in \var{utxo'}}~\val~\var{out} = \sum_{\wcard \mapsto \var{out}\in \var{utxo}}~\val~\var{out} + \fun{mint}~\var{tx} \]

See \ref{fig:pov-pf} for the proof.

% Prove the POV property for the subsystem
% Automatically derive the POV property for the ledger
% While the workflow of Fig.9 is:
% Prove an ad-hoc property for the subsystem (not really formulated as the POV property of the subsystem)
% Wow! It turns out it's exactly identical to the POV property for the ledger, QED.


\subsection{NFT Minting policy specification.}
\label{sec:mit-example}

Re-using the strategy of adding up tokens in existence in the UTxO, we
can also define a contract expressing a \emph{specific minting policy}.
This example is meant to illustrate the flexibility and broad scope of the idea that
a contract "specifies the evolution of some part of the ledger state".
This example differs from the general POV contract example in a
noteworthy way. The POV contract represents a property of the ledger transition system.
Constructing contracts specifying the evolution of the quantity of tokens
under a specific policy, on the
other hand, is a tool for developing and verifying reliable minting policy code that behaves in
the specified way. It can be used to formalize and verify properties of the contract.

For a specific policy, we restrict the tokens summed up in the UTxO to
only those under that policy, which we will call $\fun{myNFTPolicy}$.
Here, the state type is still $\State \leteq \Value$, and $\Input \leteq \Tx$.
Rather than define the policy, we first write a small-steps contract $\fun{NFT}$
to specify when and what changes can be made to the totality of tokens
under that policy,
\begin{equation}
  \inference[UpdateNFTTotal]
  {
  i \leteq \{~\fun{myNFTPolicy}~\mapsto~\var{tkns}~\in~ \fun{mint}~\var{tx}~\} \\~\\
  \{\} \subseteq s \subseteq s + i \subseteq \fun{oneT}~\fun{myNFTPolicy}~""
  \\ ~ \\
  }
  {
  \begin{array}{l}
    \\
  \end{array}
    \vdash
    \left(
    \begin{array}{r}
      \var{s} \\
    \end{array}
    \right)
    \trans{nft}{tx}
    \left(
    \begin{array}{r}
      \varUpdate{\var{s}~+~\var{i}}  \\
    \end{array}
    \right) \\
  }
\end{equation}

The specification states that a transition may only occur whenever both the
start and the end state contain either no tokens, or only the NFT
$\fun{oneT}~\fun{myNFTPolicy}~""$, and that value of the start state is necessarily
contained in the value of the end state. Note that an NFT defined via this
contract can never be burned, and must be the only token under its policy. It also
does not require any authentication to be minted.
Next, we define the projection function,

\begin{align*}
  & \pi~\var{utxo}~ \leteq~\begin{cases}
    0 & \text{ if } \fun{hasRef}~~\wedge~~s = 0 \\
    s & \text{ if } s = \fun{oneT}~\fun{myNFTPolicy}~"" ~~\wedge~~ \neg~\fun{hasRef}\\
    \emptytype & \text{ otherwise } \\
  \end{cases} \\
  &  \where \\
  &  ~~~~s \leteq~ \{~p~\mapsto~ \var{tkns}~\mid~p\mapsto \var{tkns} \in \sum_{\wcard \mapsto \var{out}\in \var{utxo}}~\val~\var{out},~p = \fun{myNFTPolicy}~\} \\
  &  ~~~~\fun{hasRef} \leteq (\fun{myNFTRef} \mapsto (\fun{myNFTScript},~\wcard,~d) \in \var{utxo})
\end{align*}

Here, $\pi~\var{utxo}$ returns a non-$\emptytype$ result when either (i) the output
$\fun{myNFTRef} \mapsto (\fun{myNFTScript},~\wcard,~d)$ is in the UTxO set and no
NFTs under the policy $\fun{myNFTPolicy}$ have been minted yet, or (ii) the NFT under
the $\fun{myNFTPolicy}$ policy is already minted with the correct token name $""$
and quantity $1$, and there is no $\fun{myNFTRef}$ in the UTxO set output references.

With these projections, the $\fun{NFT}$ contract is implemented using two scripts,
the minting policy $\fun{myNFTPolicy}$, and the UTxO-locking script
$\fun{myNFTScript}$. We now define
scripts such that it possible to prove the required simulation
relation $\sim >$,
\begin{align*}
  \fun{myNFTPolicy} \leteq & \fun{mkMyNFTPolicy}~ \fun{myNFTRef}
  \nextdef
  \applyMPScript{\fun{mkMyNFTPolicy}~ \var{myRef}}~\wcard~(\var{tx},~\var{pid})~\leteq &~
  \exists (\var{myRef}, (\fun{myNFTScript},~\wcard, ~\wcard))~\\
  &~~~~\in~ \{ ~(\outputref~i, \fun{output}~i)~\mid~i \in \inputs~\var{tx}~\} \\
  & \wedge \\
  & \fun{oneT}~\var{pid}~"" ~=~ \mint~\var{tx}
  \nextdef
  \applyScript{\fun{myNFTScript}}~\wcard~\wcard~(\var{tx},~\var{i})~\leteq &~
  ~\fun{oneT}~(\fun{mkMyNFTPolicy}~ (\var{tx},~\var{i}))~"" ~=~ \mint~\var{tx}
\end{align*}

The specific NFT
\[ \fun{myNFT} \leteq \fun{oneT}~(\fun{mkMyNFTPolicy}~ \fun{myNFTRef})~"" \]

is identified uniquely by the output reference $\fun{myNFTRef}$ that must be
spent to mint it. To make minting possible, the transaction $\fun{tx}$ must first place the
following output into the UTxO, with $\fun{myNFTRef} \leteq (\fun{tx}, \fun{ix})$,
\[ \fun{myOut} \leteq \fun{myNFTRef} \mapsto (\fun{myNFTScript},~\wcard, ~\wcard) \]

The reference $\fun{myNFTRef}$ is chosen by the contract author, and is only
valid if it points to $\fun{myOut}$ in the UTxO. To prove $\sim >$ (see \ref{pf:nft}),
we need to make an additional assumption stating that a transaction which
adds $\fun{myOut}$ to the UTxO cannot be valid again later, once the NFT
has been minted :

\paragraph{NFT re-minting protection.}
For any $(\var{slot},~\var{utxo},~\var{tx},~\var{utxo'}) \in \LEDGER$, with
$\pi~\var{utxo}~= \fun{myNFT}$, necessarily $\var{tx} \neq \fun{fst}~\mathsf{myNFTRef}$.

This property is a consequence of replay protection. Replay protection is a trace-based
safety property
\cite{liveness} of the EUTxO $\LEDGER$, contingent on certain constraints being satisfied by initial
trace states \cite{properties}. A full treatment of traces and properties is outside
the scope of this work. So, to ensure correct program behaviour, we introduce
the above assumption.
A related conjecture, provenance, is also proved in \cite{eutxoma}
for a ledger structure with a fixed initial state and similar validation rules.
We note that this is an assumption that must be made for all contracts sourcing
the uniqueness of their NFT identifier from the uniqueness of a an output
reference across the entire ledger state history.

%  To prove this trace-based property, assumptions about
% the UTxO state prior to the minting of allowed initial UTxO state must be made.  is outside the scope of this work. A similar
% conjecture, called provenance, is made and proved in \cite{eutxoma}. This is possible
% because the ledger structure in that work is the complete list of all transactions, in order
% of their application. This makes it possible to reason about the full transaction
% history, rather than only

\paragraph{$\fun{NFT}$ property example.}
At most one NFT under the policy $\fun{myNFTPolicy}$ can ever exist.
If one does not exist, one can be minted, and
moreover, an existing NFT cannot be burned : for any $\var{tx}$ such that
$i \leteq \{~\fun{myNFTPolicy}~\mapsto~\var{tkns}~\in~ \fun{mint}~\var{tx}~\}$,
\[ \forall~ s,~i,~~ \{\} \subseteq s \subseteq s + i \subseteq \fun{oneT}~\fun{myNFTPolicy}~"" \]

This is immediate from the $\fun{NFT}$ specification. Note that this is a safety
propery \cite{liveness}, as it can be checked by inspecting a finite number
of individual states in a given trace. 
