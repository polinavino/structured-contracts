\section{Simulations and the structured contract formalism}
\label{sec:struc}

The programs we are interested in specifying for the purpose of this work
are those that \emph{run on the ledger}. Intuitively, a stateful program is implemented
on the ledger whenever its state is observable in (i.e. computable from) the ledger state,
and whenever the ledger state is updated, the observed program state is updated in accordance
with the program's specification. We now formalize the notion of correctly implementing
a stateful program on the ledger using smart contract scripts.

The purpose of a smart contract script is to encode the conditions under which
a transaction \emph{can update a part of the ledger state} with which the script is
associated, e.g. change the total quantity of tokens under a given policy, or
remove some UTxO entries. This interpretation of the use of stateless
code on the ledger justifies a \emph{stateful} program model for representing
most programs running on the ledger. Stateful programs are implemented using one or more
interacting scripts controlling the updates of the corresponding data in the UTxO state.
The state of a program on the ledger is \emph{observed} by applying a projection
function to the ledger state which aggregates the relevant data.

A structured contract includes a specification, and a projection function
that computes the contract state from a given ledger state. It also requires
a proof of the integrity of the contract's implementation, establishing
a simulation relation between it and the ledger. That is, that the
scripts controlling the ledger data returned by the projection function
ensure the evolution of that data is according to the contract
specification. For example, the projection function may return the value and
datum of a UTxO containing a special NFT. This pair of makes up the state of
a given stateful contract. The script locking that UTxO
must guarantee that upon being spent, the NFT is always placed into a new UTxO
with a particular datum and value, which are computed according to the contract specification.
This approach to ensuring adherence to specification is called a \emph{thread token}
mechanism \cite{eutxoma}, and we will elaborate
on it in later sections.

\subsection{Simulations}

We instantiate the
definition a \emph{simulation} \cite{milner-pibook} with labelled
state transition systems expressed as small-step semantics specifications.

\textbf{Simulation definition.}
\label{def:simulation}
Let $\TRANS$ and $\STRUC$ be small-step labelled transition systems.
A \emph{simulation} of $\TRANS$ in $\STRUC$, denoted by
\[ (\STRUC, \sim_{\TRANS,\STRUC}, \sim_{\to,\TRANS,\STRUC}) \succeq \TRANS \]

Consists of of the following types together with the following relations :
\begin{equation*}
  \_ \vdash
  \var{\_} \trans{trans}{\_} \var{\_}
  \subseteq (\Env_{\TRANS} \times \State_{\TRANS} \times \Input_{\TRANS} \times \State_{\TRANS})
\end{equation*}
\begin{equation*}
  \_ \vdash
  \var{\_} \trans{struc}{\_} \var{\_}
  \subseteq (\Env_{\STRUC} \times \State_{\STRUC} \times \Input_{\STRUC} \times \State_{\STRUC})
\end{equation*}
\begin{align*}
  \wcard~\sim_{\TRANS,\STRUC}~\wcard ~&:~ \State_{\TRANS} \times \State_{\STRUC} \to \Bool \\
  \wcard~\sim_{\to,\TRANS,\STRUC} ~\wcard ~&:~ (\Env_{\TRANS} \times \Input_{\TRANS}) \times (\Env_{\STRUC} \times \Input_{\STRUC}) \to \Bool
\end{align*}

such that the following holds :
  \begin{equation}
    \inference[$\sim >$]
    {
      \\~\\
      (e,~i) \sim_{\to,\TRANS,\STRUC} (e', j) & u~\sim_{\TRANS,\STRUC}~s
      \\~\\
      {
        \begin{array}{c}
          e\\
        \end{array}
      }
      \vdash
      {
        \left(
          \begin{array}{r}
            \var{s} \\
          \end{array}
        \right)
      }
      \trans{trans}{\var{i}}
      {
        \left(
          \begin{array}{r}
            \var{s'} \\
          \end{array}
        \right)
      }
      \\~\\
    }
    {
      u'~\sim_{\TRANS,\STRUC}~s' ~~~~~
      \begin{array}{r}
        e'\\
      \end{array}
    \vdash
      (u)
      \trans{struc}{j}
      (\varUpdate{u'})
    }
  \end{equation}

%   \begin{equation}
%     \inference[$\sim >$]
%     {
%       \\~\\
%       \pi_{\STRUC}~u~ \neq~ \emptytype &
%       {
%         \begin{array}{c}
%           e\\
%         \end{array}
%       }
%       \vdash
%       {
%         \left(
%           \begin{array}{r}
%             \var{u} \\
%           \end{array}
%         \right)
%       }
%       \trans{ledger}{\var{t}}
%       {
%         \left(
%           \begin{array}{r}
%             \var{u'} \\
%           \end{array}
%         \right)
%       }
%       \\~\\
%     }
%     {
%       (\pi_{\STRUC}~u'~ \neq~ \emptytype) ~~\wedge~~
%       \begin{array}{r}
%         * \\
%       \end{array}
%     \vdash
%       (\pi_{\STRUC}~u)
%       \trans{struc}{\pi_{\Tx,\STRUC}~t}
%       (\varUpdate{\pi_{\STRUC}~u'})
%     }
%   \end{equation}


We write $\sim$ instead of $\sim_{\TRANS,\STRUC}$ whenever the subscript is
unambiguous. This relation states that a if a valid state $s \in \State_{\STRUC}$ is
associated with a valid UTxO state $u$, then any ledger transition starting in
$s \in \State_{\TRANS}$ is necessarily associated with a valid transition starting
in state $s$. Note that $\sim$ is a \emph{proof obligation} that must be
fulfilled as part of the definition, which \emph{does not define a rule}.
One can construct pairs of transition
systems with $\sim,~\sim_{\to}$ relations for which this proof obligation cannot
be fulfilled. However, \emph{if} it is possible, we have a simulation of $\TRANS$
in $\STRUC$.

% Note here that, given a contract state $s$ such that $\var{utxo} \sim s$,
% the only entries in the UTxO set are ones that contain some part of the
% representation of the state $s$. Then, any transaction applied to $\var{utxo}$
% will necessarily change some part of the state representation, and therefore
% correspond to a non-trivial state update.


\subsection{Structured contracts.}
\label{sec:struc-def}

The simulation definition we give is general, however, the rest of this work is
geared towards reasoning about the programmable
parts of the ledger, i.e. those where the permissions are controlled by user-defined
scripts. For this reason, we define a particular class of simulations of $\LEDGER$.
First of all, since scripts are not allowed to inspect block-level data (i.e. the
current slot number), we fix the environment of the structured contract specification
to be a singleton type $\emptytypeT$. Secondly, $\sim$ must be a
partial function, rather than a relation, which computes a unique contract state
for a given UTxO state (or fails, returning $\emptytypeT$). The
relation between arrows must also be expressible as a function which
computes a specific contract input value for any given transaction.

\textbf{Definition (Structured contract).}
Let
$(\STRUC, \sim_{\LEDGER,\STRUC}, \sim_{\to,\LEDGER,\STRUC}) \succeq \LEDGER$
be a simulation. We say that it is a \emph{structured contract} whenever
$\Env_{\STRUC} = \emptytype$, and there exist
two functions $\pi_{\STRUC} : \UTxO \to \State_{\STRUC} \cup \emptytypeT$,
$\pi_{\Tx,\STRUC} : \Tx \to \Input_{\STRUC}$ such that :
\begin{align*}
  \var{utxo}~\sim~s &\leteq (\pi_{\STRUC}~\var{utxo} ~=~s) \\
  (\var{slot},~\var{tx})~\sim_{\to,\LEDGER,\STRUC}~ (\emptytype, i) &\leteq (\pi_{\Tx,\STRUC}~\var{tx} = i)
\end{align*}

\textbf{Discussion. }
% This definition states that whenever for a given start state $\var{utxo}$
% with a corresponding contract start state $\pi~\var{utxo} \neq \emptytype$,
% each valid step $(\var{slot}, \var{utxo}, \var{tx}, \var{utxo'}) \in \LEDGER$
% at the ledger level necessarily has a corresponding valid
% step at the contract level,
% $(\emptytype, \pi~\var{utxo}, \pi_{\Tx}~\var{tx}, \pi~\var{utxo'}) \in \STRUC$.
We sometimes denote $\pi_{\Tx,\STRUC}$ and $\pi_{\STRUC}$
by $\pi$ and $\pi_{\Tx}$, respectively, when the context is clear.
We also denote the structured contracts by $(\STRUC,~\pi_{\STRUC},~\pi_{\Tx,\STRUC}) \succeq \LEDGER$.
We say that a $\var{utxo}$ is \emph{valid for $\STRUC$} whenever $\pi~\var{utxo} \neq \emptytype$.

It is possible for transactions to update the ledger state, but not the $\STRUC$ state.
There is no a special case for this in the simulation relation, and the $\STRUC$
specification rules must allow trivial steps if such ledger transactions are possible.
The $\STRUC$ specification may not be deterministic, however, since the
$\LEDGER$ is deterministic, given a ledger
step, there is a unique step in $\STRUC$ corresponding to the ledger one.
We do not assume that a valid contract state can be computed from an arbitrary
UTxO state. For this reason, the function $\pi_{\STRUC}$ is partial.
For example, it is possible that two NFTs exist in a given ledger
state. When programmed correctly, an NFT minting policy would not allow this
to happen. To reason about properties of such a policy, we must exclude ledger states
where the NFT uniqueness condition has a been violated in the start state.

Requiring $\sim >$ to be a bisimulation \cite{milner-pibook} between $\STRUC$ and
$\LEDGER$ is too restrictive, and excludes a lot of interesting contracts.
Defining a class of structured contracts for which
this is possible is a difficult problem, and we leave it for future work.

\section{NFT minting policy as a structured contract}
% \label{sec:pov}
%
% In this section we give an example of using the structured contract
% formalism to express global invariants of the ledger that
% can be derived from local invariants.
% The EUTxO ledger transition system $\LEDGER$ is said to satisfy the
% \emph{preservation of value (POV)} property. The \emph{local} POV is
% the property that each valid transaction mints exactly the difference
% between the value in the UTxOs in spends and those it creates.
% This is ensured by the ledger rule \ref{rule:value-is-preserved}, which is
% \[ \mint~\var{tx} + \sum_{i \in \inputs~\var{tx},~(\outputref~i)\mapsto~o~\in~\var{utxo}} \val~\var{o} = \sum_{o \in \outputs~\var{tx}} \val~\var{o} \]
%
% Suppose that $(\var{s},~\var{utxo},~\var{tx},~\var{utxo'}) \in \LEDGER$, and
% recall that updated UTxO set is computed as follows
% \[ \var{utxo'} \leteq (\{~i\mapsto o \in \var{utxo} ~\mid~ i \notin \fun{getORefs}~{tx}~\}) \cup \fun{mkOuts}~{tx} \]
%
% The \emph{global} POV, which we will prove from the local POV,
% states that the sum total of all the tokens
% recorded on the ledger is changed by exactly the amount minted or burned by the
% transaction,
% \[ \sum_{\var{or}\mapsto \var{out}\in \var{utxo'}}~\val~\var{out} ~= ~(\sum_{\var{or}\mapsto \var{out}\in \var{utxo}}~\val~\var{out}) ~+~(\fun{mint}~ \var{tx}) \]
%
% We can express this property as the following structured contract, $\mathsf{POV}$ : \newline
%
% % \begin{figure}[htb]
% \noindent \emph{Transition type}
%   \begin{equation*}
%     \_ \vdash
%     \var{\_} \trans{pov}{\_} \var{\_}
%     \in \powerset (\emptytype \times \Value \times \Tx \times \Value)
%   \end{equation*}
%   %
% \emph{Simulation relations}
%   \begin{align*}
%     \pi_{\Tx,\POV}~\var{tx}~ ~&\leteq~\var{tx} \\
%     \pi_{\POV}~\var{utxo}~&\leteq~\sum_{\wcard \mapsto \var{out}\in \var{utxo}}~\val~\var{out}
%   \end{align*}
%   %
% \emph{Transition rule}
%   \begin{equation}
%     \inference[UpdateValTotal]
%     {
%     i \leteq \fun{mint}~\var{tx}
%     \\ ~ \\
%     }
%     {
%     \begin{array}{l}
%       \\
%     \end{array}
%       \vdash
%       \left(
%       \begin{array}{r}
%         \var{s} \\
%       \end{array}
%       \right)
%       \trans{pov}{tx}
%       \left(
%       \begin{array}{r}
%         \varUpdate{\var{s}~+~\var{i}}  \\
%       \end{array}
%       \right) \\
%     }
%   \end{equation}
% %   \caption{Specification of the $\mathsf{UpdateValTotal}$ POV subsystem}
% %   \label{fig:pov-spec}
% % \end{figure}
%
% To complete the definition, it remains to prove
% $\sim >$, which requires that, given that
% $(\var{slot},~\var{utxo},~\var{tx},~\var{utxo'}) ~\in \LEDGER$,
% necessarily $(\emptytype,~\pi~\var{utxo},~\pi~\var{tx},~\pi~\var{utxo'}) \in \POV$.
% This simplifies to exactly our definition of the global POV.
% See \ref{fig:pov-pf} for the proof.
%
% % Prove the POV property for the subsystem
% % Automatically derive the POV property for the ledger
% % While the workflow of Fig.9 is:
% % Prove an ad-hoc property for the subsystem (not really formulated as the POV property of the subsystem)
% % Wow! It turns out it's exactly identical to the POV property for the ledger, QED.
%
%
% \subsection{NFT Minting policy specification.}
\label{sec:mit-example}

Our first structured contract example is expressing a \emph{specific minting policy}.
% This example differs from the general POV contract example in a
% noteworthy way. The POV contract represents a property of the ledger transition system.
Constructing structured contracts specifying the evolution of the quantity of tokens
under a specific policy is a tool for formal analysis of minting policy code. In particular,
we are able to express and prove the defining property of a specific NFT : at most one
such token can exist on the ledger. Instantiating an NFT as a structured contract
and expressing this property allows us to mimic something that is quite naturally
expressed for account-based blockchains with stateful NFT contracts, such as the
ERC-721 \cite{erc721}, but poses a challenge for EUTxO ledger program analysis.

We first pick an identifier for the policy we wish to express,
$\fun{myNFTPolicy}$. Before writing the policy code, we define a system
$\fun{NFT}$ to specify how we want
the total number of tokens on the ledger under this policy to behave.
Here, the state type is $\State \leteq \Value$, and $\Input \leteq \Tx$.
\begin{equation}
  \inference[UpdateNFTTotal]
  {
  i \leteq \{~\fun{myNFTPolicy}~\mapsto~\var{tkns}~\in~ \fun{mint}~\var{tx}~\} \\~\\
  \{\} \subseteq s \subseteq s + i \subseteq \fun{oneT}~\fun{myNFTPolicy}~[~]
  \\ ~ \\
  }
  {
  \begin{array}{l}
    \\
  \end{array}
    \vdash
    \left(
    \begin{array}{r}
      \var{s} \\
    \end{array}
    \right)
    \trans{nft}{tx}
    \left(
    \begin{array}{r}
      \varUpdate{\var{s}~+~\var{i}}  \\
    \end{array}
    \right) \\
  }
\end{equation}

The specification states that the only allowed transitions are (i) a constant one,
and (ii) adding a single NFT, given by $\fun{oneT}~\fun{myNFTPolicy}~[~]$, to the state if
one does not yet exist. An NFT whose total ledger quantity obeys this specification
can never be burned, and must be the only token under its policy. It also
does not require any authentication to be minted. To define the policy and
projection functions, we pick an output reference $\fun{myNFTRef}$ which we call an
\emph{anchor}. That is,
$\fun{myNFTRef}$ must be spent by the NFT-minting transaction as a
mechanism to ensure that no other transaction can mint another NFT under this policy.
Next, we define the projection function,
\begin{align*}
  & \pi~\var{utxo}~ \leteq~\begin{cases}
    s & \text{ if~~} (s ~=~ \fun{oneT}~\fun{myNFTPolicy}~[~]~~\wedge~~\neg~\fun{hasRef}) ~~\vee~~ s = 0\\
    \emptytype & \text{ otherwise } \\
  \end{cases} \\
  &  \where \\
  &  ~~~~s \leteq~ \{~p~\mapsto~ \var{tkns}~\mid~p\mapsto \var{tkns} \in \sum_{\wcard \mapsto \var{out}\in \var{utxo}}~\val~\var{out},~p = \fun{myNFTPolicy}~\}  \\
  &  ~~~~\fun{hasRef} \leteq (\fun{myNFTRef} \mapsto \wcard \in \var{utxo})
\end{align*}

Here, $\pi~\var{utxo}$ returns a non-$\emptytype$ result when either
no tokens under the $\fun{myNFTPolicy}$ policy exist, or only the token
$\fun{oneT}~\fun{myNFTPolicy}~[~]$ exists under this policy, and the anchor $\fun{myNFTRef}$
is not in the UTxO. We define the policy,
\begin{align*}
  \fun{myNFTPolicy} \leteq &~ \fun{mkMyNFTPolicy}~ \fun{myNFTRef}
  \nextdef
  \applyMPScript{\fun{mkMyNFTPolicy}~ \var{myRef}}~\wcard~(\var{tx},~\var{pid})~\leteq &~
  \exists ~(\var{myRef}, \wcard, \wcard)~\in~ \inputs~\var{tx}~\\
  & \wedge ~~\fun{oneT}~\var{pid}~[~] ~\in~ \mint~\var{tx}
  % \nextdef
  % \applyScript{\fun{myNFTScript}}~\wcard~\wcard~(\var{tx},~\var{i})~\leteq &~   (\fun{myNFTScript},~\wcard, ~\wcard)
  % ~\fun{oneT}~(\fun{mkMyNFTPolicy}~ (\var{tx},~\var{i}))~[~] ~=~ \mint~\var{tx}
\end{align*}

% is identified uniquely by the output reference $\fun{myNFTRef}$ that must be
% spent to mint it. To make minting possible, the transaction $\fun{tx}$ must first place the
% following output into the UTxO, with $\fun{myNFTRef} \leteq (\fun{tx}, \fun{ix})$,
% \[ \fun{myOut} \leteq \fun{myNFTRef} \mapsto (\fun{myNFTScript},~\wcard, ~\wcard) \]
%
% The reference $\fun{myNFTRef}$ is chosen by the contract author, and is only
% valid if it points to $\fun{myOut}$ in the UTxO.

To prove $\sim >$ for the $\NFT$ contract (see Appendix \ref{pf:nft} for a proof
sketch), we need to make an additional assumption stating that a transaction which
adds $\fun{myNFTRef}$ to the UTxO cannot be valid again later, once the NFT
has been minted :

\textbf{NFT re-minting protection.}
\label{sec:asm-nft}
For any $(\var{slot},~\var{utxo},~\var{tx},~\var{utxo'}) \in \LEDGER$, with
$\pi~\var{utxo}~= \fun{oneT}~\fun{myNFTPolicy}~[~]$, necessarily $\var{tx} \neq \fun{fst}~\mathsf{myNFTRef}$.

Under reasonable assumptions about the initial state of the ledger, this property
should be consequence of replay protection, which is a trace-based
safety property of the EUTxO $\LEDGER$. A full treatment of traces and properties is
the subject of future work. So, to ensure correct program behaviour, we introduce
the above assumption.
% A related conjecture, provenance, is also proved in \cite{eutxoma}
% for a ledger structure with a fixed initial state and similar validation rules.
% We note that this is an assumption that must be made for all contracts sourcing
% the uniqueness of their NFT identifier from the uniqueness of a an output
% reference across the entire ledger state history.

%  To prove this trace-based property, assumptions about
% the UTxO state prior to the minting of allowed initial UTxO state must be made.  is outside the scope of this work. A similar
% conjecture, called provenance, is made and proved in \cite{eutxoma}. This is possible
% because the ledger structure in that work is the complete list of all transactions, in order
% of their application. This makes it possible to reason about the full transaction
% history, rather than only

\textbf{$\fun{NFT}$ property example.}
At most one NFT under the policy $\fun{myNFTPolicy}$ can ever exist in any
$\var{utxo}$ that is valid for $\NFT$ :
for any $\var{utxo}$ such that $\fun{pi}~\var{utxo} \neq \emptytype$,
\[ \fun{pi}~\var{utxo}~\subseteq~ \fun{oneT}~\fun{myNFTPolicy}~[~] \]

This is immediate from the definition of $\fun{pi}$, however, this result is
meaningful. By definition of $\sim >$, and the fact
that $\NFT$ is a structured contract, it is not possible
to transition from a state valid for $\NFT$ to a state which is not valid for $\NFT$. That is,
with $(\var{slot}, \var{utxo}, \var{tx}, \var{utxo'}) \in \LEDGER$, the
updated state $\pi~\var{utxo'}$ must also always have at most one NFT under $\fun{myNFTPolicy}$.
This also implies that at most one can ever be minted by a valid transaction
applied to a $\var{utxo}$ valid for $\NFT$.
%
%
%
% for any $\var{tx}$ such that
% $i \leteq \{~\fun{myNFTPolicy}~\mapsto~\var{tkns}~\in~ \fun{mint}~\var{tx}~\}$,
% \[ \forall~ s,~i,~~ \{\} \subseteq s \subseteq s + i \subseteq \fun{oneT}~\fun{myNFTPolicy}~[~] \]
%
% This is immediate from the $\fun{NFT}$ specification. Note that this is a safety
% propery \cite{liveness}, as it can be checked by inspecting a finite number
% of individual UTxO states in a given trace.
