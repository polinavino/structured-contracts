\section{Multiple implementations of a single specification}
\label{sec:toggle}

In this section we present an example of a specification that has more than
one correct implementation, one of which is distributed
across multiple UTxO entries. The guarantee that the two implement the same specification
enables contract authors to meaningfully compare them across any relevant characteristics,
such as space usage, or parallelizability.

\subsection{Toggle specification}

We define a specification wherein
the state consists of two booleans, and only one can be $\true$
at a time. We set the contract input to be
be $\{\toggle \} \cup \emptytype$.
The two booleans in the state are both flipped by the input $\toggle$, and unchanged by $\emptytype$.
We define the transition system $\TOGGLE$ :
% \begin{figure}[htb]
  \begin{equation}
    \inference[DoNothing]
    { \\
    }
    {
    \begin{array}{l}
      \\
    \end{array}
      \vdash
      \left(
      \begin{array}{r}
        x, ~y
      \end{array}
      \right)
      \trans{TOGGLE}{\emptytype}
      \left(
      \begin{array}{r}
        x, ~y
      \end{array}
      \right) \\
    }
  \end{equation}
  \begin{equation}
    \inference[Toggle]
    {~\\
    }
    {
    \begin{array}{l}
      \\
    \end{array}
      \vdash
      \left(
      \begin{array}{r}
        x,~y
      \end{array}
      \right)
      \trans{TOGGLE}{toggle}
      \varUpdate{\left(
      \begin{array}{r}
        y,~x
      \end{array}
      \right) } \\
    }
  \end{equation}
%   \caption{Specification of the $\mathsf{TOGGLE}$ transition}
%   \label{fig:toggle-rule}
% \end{figure}

\subsection{Toggle implementations}

We present two implementations of the $\TOGGLE$ specification. The \emph{naive implementation}
is one that uses the datum of a single UTxO entry to store a representation of the full state of the
$\TOGGLE$ contract. The \emph{distributed implementation} uses datums in two distinct UTxO
entries to represent the first and the second value of the pair that is the $\TOGGLE$ state.

\textbf{Thread token scripts. }
We use the \emph{thread tokens} mechanism \cite{eutxoma} to ensure the unique identification of the
UTxO (or pair of UTxOs) from which the contract state is computed. In both
implementations, the thread token minting policy guarantees that they
are generated in quantity 1 by a transaction that spends a specific output reference $\mathsf{myRef}$,
similar to the NFT policy in Section \ref{sec:mit-example}.

For the naive implementation, one thread token NFT is sufficient to identify the
state-bearing UTxO. Upon minting, the policy requires the token to be placed
into a UTxO locked by a specific contract, which is passed as a parameter to
the minting policy. This contract (discussed below) ensures the correct evolution of the
contract state. The datum in the UTxO containing the thread token is the initial state of the
contract encoded as a pair of booleans
(by the partial decoder function $\fun{fromData}_{N} : \Data \to \B \cup \emptytypeT$).
It can be any pair of correctly encoded booleans.
See Figure \ref{fig:toggle-tt-n} for the policy pseudocode.

For the distributed implementation, two distinct NFTs are needed to identify
the UTxOs containing the $\TOGGLE$ state data. Both NFTs are under the same
minting policy and must be minted by a single transaction, but have distinct token
names, $"a"$ and $"b"$. Upon being minted,
the policy requires that they are placed in separate UTxO, locked by the same
contract (discussed below). The datum in each must be decodeable
(by $\fun{fromData}_{N} : \Data \to (\B \times \B) \cup \emptytypeT$) as a boolean.
See Figure \ref{fig:toggle-tt-d} for the policy pseudocode.

% are required
% To define unique identifiers that point to the $\TOGGLE$ contract representation in the UTxO,
% we borrow the design pattern of \emph{thread tokens} introduced in \cite{eutxoma}. The idea is that
% an NFT is minted and stored in a UTxO entry that contains the datum representing (all
% or part of) the contract
% state. Since the NFT is script-guaranteed to be unique, it marks the unique authentic
% state-storing UTxO.
% When the state is updated, the UTxO containing the thread token is spent, and the
% thread token is placed in a new entry containing the updated state representation.
%
% \textbf{Naive implementation. }
% The naive implementation state is stored within a single UTxO, and so requires
% one thread token to track the evolution of the subsystem state.
%
% An output reference $\mathsf{myRef} \in \OutputRef$ is
% selected from those existing in the UTxO to uniquely identify the
% thread token. It must be spent for the minting policy to validate
% as a mechanism for ensuring this policy can only
% validate once, minting exactly one thread token, and placing it into the
% appropriately constructed output.
% The minting policy for the thread tokens is constructed as follows :
%
%
%
% \textbf{Distributed implementation. }
% The minting policy for the two distributed implementation
% thread tokens is constructed by :
%
%
%
% Here, two distinct thread tokens must be minted and placed into separate UTxOs
% locked with the correct script. This is because each UTxO representing part of the
% toggle state requires a unique identifier. We distinguish between the tokens
% by adding $"a"$ to the token name of one, and $"b"$ to the other.

% The functions, whose type is specified in \ref{fig:to-data}, encode and decode the
% datums which the implementation scripts inspect.
% Datums appears in transactions and UTxOs in the encoded $\Data$ format.

\textbf{Validator scripts. }
We require different UTxO-locking scripts for our two distinct implementations.
Both scripts serve the following function : when $\toggle$ redeemers are specified,
the script must ensure that the thread
tokens are propagated into UTxOs that are locked by the same validator as
the spent UTxOs containing the thread tokens, and that the datums in those UTxOs
are correct. For the naive version, the datum in the new UTxO containing the thread
token must decode as a pair of booleans whose order is reversed as compared to
the booleans encoded in the datum of the spent UTxO that previously contained the thread token.
We define it by :
% \begin{ruledfigure}{t}
  \begin{align*}
    \applyScript{\toggleVal_N~\mathsf{myRef}}~(b, b')~r~(\var{tx},~\var{i})~\leteq &~
    \var{ttt}~ =~ \val~(\fun{output}~i)~~\wedge~~r = \toggle \\
    & ~~~~ \wedge~~\exists~o~\in~\outputs~\var{tx},~~(b',~b)~=~(\datum~o) \\
    & ~~~~ \wedge~~(\validator~o~=~\var{vi}) ~~\wedge ~~(\var{ttt} =~\val~o)\\
    & \where \\
    & ~~~~\var{vi} \leteq \validator~(\fun{output}~i) \\
    & ~~~~\var{ttt} \leteq \mathsf{oneT}~(\toggleTT_N~\mathsf{myRef}~\var{vi} )~(\fun{encode}~\var{vi})
  \end{align*}
% \caption{$\TOGGLE$ validator script for the naive implementation}
% \label{fig:toggle-val-n}
% \end{ruledfigure}

The function $\fun{encode} : \Script \to [\type{Char}]$ encodes a script as a string
for the purpose of specifying (via the token name) the output-locking script that
must persistently lock the thread token.

The distributed implementation script ensures that both the thread token-containing
UTxOs are spent simultaneously. Then, it checks that the booleans in the datums are switched places :
the one that was in the UTxO with token $"a"$ must now be in a new UTxO with
token $"b"$, and vice-versa. The validator script is given in Figure \ref{fig:toggle-val-d}.
\begin{ruledfigure}{t}
  \begin{align*}
    \applyScript{\toggleVal_D~&\mathsf{myRef}}~b~\toggle~(\var{tx},~\var{i})~\leteq ~
    (\var{tta}~ =~ \val~(\fun{output}~i)) \Rightarrow \\
    & ~~~~\exists~o,~o'\in~\outputs~\var{tx},~i' \in \fun{inputs}~\var{tx}, \\
    & ~~~~ \validator~o~=~\validator~o'~=~\var{vi}~\wedge~\\
    & ~~~~ \mathsf{tta}~ =~ \val~o \wedge \var{ttb}~ =~ \val~o' \wedge \val~(\fun{output}~i')~=~\var{ttb} \\
    & ~~~~ \datum~o~=~\datum~ (\fun{output}~i') \wedge \datum~o'~=~\datum~ (\fun{output}~i) \\
    & \wedge \\
    & (\var{ttb}~ =~ \val~(\fun{output}~i)) \Rightarrow \\
    & ~~~~\exists~o,~o'\in~\outputs~\var{tx},~i' \in \fun{inputs}~\var{tx}, \\
    & ~~~~ \validator~o~=~\validator~o'~=~\var{vi}~\wedge~\\
    & ~~~~ \mathsf{tta}~ =~ \val~o \wedge \var{ttb}~ =~ \val~o' \wedge \val~(\fun{output}~i')~=~\var{tta} \\
    & ~~~~ \datum~o~=~\datum~ (\fun{output}~i) \wedge \datum~o'~=~\datum~ (\fun{output}~i') \\
    & \wedge \\
    & (\var{tta}~ =~ \val~(\fun{output}~i)) \vee (\var{ttb}~ =~ \val~(\fun{output}~i)) \\
    & \where \\
    & ~~~~\var{vi} \leteq \validator~(\fun{output}~i) \\
    & ~~~~\var{tta} \leteq \mathsf{oneT}~(\toggleTT_D~\mathsf{myRef}~\var{vi})~(\fun{encode}~\var{vi}~++~ "a") \\
    & ~~~~\var{ttb} \leteq \mathsf{oneT}~(\toggleTT_D~\mathsf{myRef}~\var{vi})~(\fun{encode}~\var{vi}~++~"b")
  \end{align*}
\caption{$\TOGGLE$ validator script for the distributed implementation}
\label{fig:toggle-val-d}
\end{ruledfigure}

\textbf{Ledger representation.}
\label{sec:led-rep}
The state projection function computations return a valid contract state (i.e. a pair of
booleans) whenever the anchor reference $\mathsf{myRef}$ is not in the UTxO,
and thread tokens have been minted according to their policy and placed alongside
the appropriate datums and UTxO scripts. The input projection function returns
$\toggle$ whenever a transaction contains the thread tokens in its input(s),
and $\emptytype$ otherwise. For details,
see Figures \ref{fig:toggle-sim-n} \ref{fig:toggle-sim-d-id}.

\begin{ruledfigure}{t}
  \begin{align*}
    \pi_d~\var{utxo}&~ \leteq~\begin{cases}
      (a, b) & \text{ if } \mathsf{myRef}~\notin~\{ ~i~\mid~i\mapsto o \in  \var{utxo}~\} \\
      & \wedge~~ \exists ! ~(i\mapsto o,~i'\mapsto o')~\in~\var{utxo},~\fun{tta}~=~\val~o~~\wedge~~\fun{ttb}~=~\val~o'\\
      & ~~~~ \wedge~~\validator~o~=~\toggleVal_D~\mathsf{myRef}~=~\validator~o'~~\\
      & ~~~~\wedge~~\datum~o~=~a~~\wedge~~\datum~o~=~b \\
      \emptytype~& \text{otherwise}
    \end{cases}
    \nextdef
    \pi_{\Tx,d}~\var{tx}~ &\leteq~\begin{cases}
      \toggle & \text{ if } \exists~i,~i'~ \in~\inputs~\var{tx},~\val~(\fun{output}~i) = \fun{tta} ~~\wedge ~~\val~(\fun{output}~i') = \fun{ttb}\\
      \emptytype & \text{otherwise}
    \end{cases}
  \end{align*}
\caption{$\TOGGLE$ distributed projections}
\label{fig:toggle-sim-d-id}
\end{ruledfigure}

In Appendix \ref{sec:toggle-sim}, we give a proof sketch for the simulation
relations between $\TOGGLE$ and $\LEDGER$ to complete the instantiation of
the two versions of the structured contract. To avoid duplication of
thread tokens, we again need to make the
additional assumption that
for any $(\var{slot},~\var{utxo},~\var{tx},~\var{utxo'}) \in \LEDGER$, with
$\pi~\var{utxo}~= (a, b)$, necessarily $\var{tx} \neq \fun{fst}~\mathsf{myRef}$.


% \paragraph{$\LEDGER$ - $\TOGGLE$ state and transition relations}
%
% Here we first encounter a situation wherein not every UTxO has a corresponding
% $\TOGGLE$ state. In particular, we can only relate UTxO states to toggle states
% wherein the toggle contract has already been initialized, ie. thread token(s) minted
% and stored in a correctly-defined UTxO. We fix $\mathsf{myRef}$ to some specific
% output reference. We also instantiate the thread tokens as follows :
%

%
% \textbf{Naive implementation relations. }
%
% In the naive implementation, a ledger state is related to a $\TOGGLE$ state
% whenever (i) the $\TOGGLE$ state is a pair booleans where one is the negation of the
% other, and (ii) there is exactly one output on the ledger containing the thread token,
% and locked by the correct validator, and with a datum that decodes to the same pair of booleans.
%

%
% Next, we define the projections :
%

%
% \textbf{Distributed implementation relation. }
%
% In the distributed implementation, a ledger state is related to a $\TOGGLE$ state
% whenever there are exactly two outputs on the ledger each containing one of the
% the two thread tokens, $\fun{tta}$ and $\fun{ttb}$,
% locked by the correct validator, each with a datum that decodes to the the corresponding
% boolean in the $\TOGGLE$ state.
%

%
% The function $\pi_{\TOGGLE}$ is defined in the same way as for the naive
% implementation, but with $\sim$ for the naive implementation. The transaction
% projection is :
%

%
% For both implementations, we again need to make a variation on the NFT
% re-minting protection assumption,
%

%


\textbf{$\TOGGLE$ property example. }
\[ (\emptytype, (a,b), i, (c, d)) \in \TOGGLE ~~\Rightarrow~~
(c, d) = (b, a) \vee (c, d) = (a, b) \]

This property states that in any step of $\TOGGLE$, either the state booleans are swapped,
or stay the same. Its proof is immediate from the specification, regardless of the
implementation.
% Since a
% structured contract is guaranteed to have only valid steps represented on the
% ledger,
% And, because we have demonstrated the $\sim >$ relation, a related property can be
% stated for all valid ledger transactions applied to
% any state $\var{utxo}$ such that $\var{(a, b)} = \pi ~\var{utxo}$ :
%
% \[ \forall \var{utxo} \in \UTxO, (\wcard, \var{utxo}, \var{tx}, \var{utxo'}) \in \LEDGER ~~\Rightarrow~~
% \pi~\var{utxo} = \pi~\var{utxo'} \vee \pi~\var{utxo} = ((\pi~\var{utxo'})_2, (\pi~\var{utxo'})_1) \]
%
% Both properties require a constraint to hold for every state, and therefore
% are safety properties \cite{liveness}.
