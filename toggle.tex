\section{Multiple implementations of a single specification example}
\label{sec:toggle}

Script interaction in the UTxO ledger refers the dependency of one
script on the successful validation of another that is run within the same transaction.
A feature of the structured contract formalism is that
it allows for specifying stateful programs that can be implemented with
multiple interacting UTxO-locking and minting policy scripts, which we demonstrate
in this section.

Using this example, we additionally demonstrate that our formalism allows
for different implementations
of the same specification. This enables contract authors to compare
distinct implementations across any relevant properties, such as space usage,
or parallelizability.

\subsection{Toggle specification}

We define a specification wherein
the state consists of two booleans, and only one can be $\true$
at a time. We set the contract input to be
be $\{\toggle \} \uniondistinct \emptytype$.
The two booleans in the state are interchanged by $\toggle$, and not by $\emptytype$.
We define the transition system $\TOGGLE$ :
% \begin{figure}[htb]
  \begin{equation}
    \inference[DoNothing]
    { \\
    }
    {
    \begin{array}{l}
      \\
    \end{array}
      \vdash
      \left(
      \begin{array}{r}
        x, ~y
      \end{array}
      \right)
      \trans{TOGGLE}{\emptytype}
      \left(
      \begin{array}{r}
        x, ~y
      \end{array}
      \right) \\
    }
  \end{equation}
  \begin{equation}
    \inference[Toggle]
    {~\\
    }
    {
    \begin{array}{l}
      \\
    \end{array}
      \vdash
      \left(
      \begin{array}{r}
        x,~y
      \end{array}
      \right)
      \trans{Toggle}{toggle}
      \varUpdate{\left(
      \begin{array}{r}
        y,~x
      \end{array}
      \right) } \\
    }
  \end{equation}
%   \caption{Specification of the $\mathsf{TOGGLE}$ transition}
%   \label{fig:toggle-rule}
% \end{figure}

\subsection{Toggle implementations}

We present two implementations of this specification. The \emph{naive implementation}
is one that uses the datum of a single UTxO entry to store a representation of the state of the
$\TOGGLE$ contract. The \emph{distributed implementation} uses datums in two distinct UTxO
entries to represent the first and the second value of the pair that is the $\TOGGLE$ state.
Even with this basic example, several interacting scripts must be defined to implement
its specification. We present pseudocode for the required scripts.

\paragraph{Thread token scripts}

To define unique identifiers that point to the $\TOGGLE$ contract representation in the UTxO,
we borrow the design pattern of \emph{thread tokens} introduced in \cite{eutxoma}. The idea is that
an NFT is minted and stored in a UTxO entry that contains the datum representing (all
or part of) the contract
state. Since the NFT is script-guaranteed to be unique, it marks the unique authentic
state-storing UTxO.
When the state is updated, the UTxO containing the thread token is spent, and the
thread token is placed in a new entry containing the updated state representation.

\textbf{Naive implementation. }
The naive implementation state is stored within a single UTxO, and so requires
one thread token to track the evolution of the subsystem state.

An output reference $\mathsf{myRef} \in \OutputRef$ is
selected from those existing in the UTxO to uniquely identify the
thread token. It must be spent for the minting policy to validate
as a mechanism for ensuring this policy can only
validate once, minting exactly one thread token, and placing it into the
appropriately constructed output.
The minting policy for the thread tokens is constructed as follows :

% \begin{ruledfigure}{t}
  \begin{align*}
    \toggleTT_N~:& \OutputRef \to \Script \to \Script \\
    \applyMPScript{\toggleTT_N~ \mathsf{myRef}~s}~\wcard~(\var{tx},~\var{pid})~\leteq&
    \mathsf{myRef}~\in~ \{ ~\outputref~i~\mid~i \in \inputs~\var{tx}~\} \\
    & \wedge \\
    & \fun{oneT}~\var{pid}~(\fun{encode}~s) ~=~ \mint~\var{tx} \\
    & \wedge  \\
    &\exists~\var{o}~\in~\fun{outputs}~\var{tx}, \\
    &~~~~~\val~\var{o}~=~\mathsf{oneT}~\var{pid}~(\fun{encode}~s) \\
    & ~~~~~\wedge ~\validator~\var{o}~=~s~~\wedge ~~\fun{fromData}_N~(\datum~\var{o})~\neq~\emptytype
  \end{align*}
% \caption{$\TOGGLE$ thread token minting policy for the naive implementation}
% \label{fig:toggle-tt-n}
% \end{ruledfigure}

\textbf{Distributed implementation. }
The minting policy for the two distributed implementation
thread tokens is constructed by :

% \begin{ruledfigure}{t}
  \begin{align*}
    \toggleTT_D~:& \OutputRef \to \Script \to \Script \\
    \applyMPScript{\toggleTT_D~\mathsf{myRef}~s}~\wcard~(\var{tx},~\var{pid})~\leteq &
    \fun{myRef}~\in~ \{ ~\outputref~i~\mid~i \in \inputs~\var{tx}~\} \\
    & \wedge \\
    & \var{tta} ~+~ \var{ttb} ~=~ \mint~\var{tx} \\
    & \wedge  \\
    &\exists~\var{oa},~\var{ob}~\in~\fun{outputs}~\var{tx}, \\
    &~~~~~\val~\var{oa}~=~\var{tta}~ \\
    &~~~~~\wedge ~\val~\var{ob}~=~\var{ttb}~\\
    & ~~~~~\wedge ~\validator~\var{oa}~=~\validator~\var{ob}~=~s \\
    & ~~~~~\wedge ~\fun{fromData}_D~(\datum~\var{oa})~\neq~\emptytype~ \\
    & ~~~~~\wedge~\fun{fromData}_D~(\datum~\var{ob})~\neq~\emptytype~~ \\
    & \where \\
    & ~~~~\var{tta} \leteq \mathsf{oneT}~\var{pid}~(\fun{encode}~s, "a") \\
    & ~~~~\var{ttb} \leteq \mathsf{oneT}~\var{pid}~(\fun{encode}~s, "b")
  \end{align*}
% \caption{$\TOGGLE$ thread token minting policy for the distributed implementation}
% \label{fig:toggle-tt-d}
% \end{ruledfigure}

Here, two distinct thread tokens must be minted and placed into separate UTxOs
locked with the correct script. This is because each UTxO representing part of the
toggle state requires a unique identifier. We distinguish between the tokens
by adding $"a"$ to the token name of one, and $"b"$ to the other.

\paragraph{Validator scripts}

We again require different UTxO-locking scripts for our two distinct implementations.
The functions, whose type is specified in \ref{fig:to-data}, encode and decode the
datums which the implementation scripts inspect.
Datums appears in transactions and UTxOs in the encoded $\Data$ format.

\textbf{Naive implementation. }
The naive implementation script has two main functions : ensuring that the thread
token is propagated correctly into a new UTxO, and that the datum of that UTxO
contais the toggled state of the UTxO that previously contained the thread token.
The type of the datum of outputs locked by $\toggleVal_N$
is $\B \times \B$. The validator script is given by :

% \begin{ruledfigure}{t}
  \begin{align*}
    \applyScript{\toggleVal_N~\mathsf{myRef}}~(b, b')~\wcard~(\var{tx},~\var{i})~\leteq &~
    \var{ttt}~ =~ \val~(\fun{output}~i) \\
    & ~~~~ \wedge~ \\
    & \exists~o~\in~\outputs~\var{tx},~~(b',~b)~=~(\datum~o) \\
    &  ~~~~~~~~ \wedge~~\\
    & ~~~~\validator~o~=~\var{vi} \\
    &  ~~~~~~~~ \wedge \\
    & ~~~~ \var{ttt} =~\val~o\\
    & \where \\
    & ~~~~\var{vi} \leteq \validator~(\fun{output}~i) \\
    & ~~~~\var{ttt} \leteq \mathsf{oneT}~(\toggleTT_N~\mathsf{myRef}~\var{vi} )~(\fun{encode}~\var{vi})
  \end{align*}
% \caption{$\TOGGLE$ validator script for the naive implementation}
% \label{fig:toggle-val-n}
% \end{ruledfigure}

\textbf{Distributed implementation. }
The distributed implementation script ensures that the boolean in each of the
datums of the spent UTxO entries containing the two thread tokens
is the negation of the boolean of the newly created UTxOs with each of those tokens,
respectively. The validator script is given by :

% \begin{ruledfigure}{t}
  \begin{align*}
    \applyScript{\toggleVal_D~\mathsf{myRef}}~b~\wcard~(\var{tx},~\var{i})~\leteq &~
    (\var{tta}~ =~ \val~(\fun{output}~i)) \Rightarrow \\
    & ~~~~\exists~o,~o'\in~\outputs~\var{tx},~i' \in \fun{inputs}~\var{tx}, \\
    & ~~~~ \validator~o~=~\validator~o'~=~\var{vi}~\wedge~\\
    & ~~~~ \mathsf{tta}~ =~ \val~o \wedge \var{ttb}~ =~ \val~o' \wedge \val~(\fun{output}~i')~=~\var{ttb}) \\
    & ~~~~ \datum~o~=~\datum~ (\fun{output}~i') \wedge \datum~o'~=~\datum~ (\fun{output}~i) \\
    & \wedge \\
    & (\var{ttb}~ =~ \val~(\fun{output}~i)) \Rightarrow \\
    & ~~~~\exists~o,~o'\in~\outputs~\var{tx},~i' \in \fun{inputs}~\var{tx}, \\
    & ~~~~ \validator~o~=~\validator~o'~=~\var{vi}~\wedge~\\
    & ~~~~ \mathsf{tta}~ =~ \val~o \wedge \var{ttb}~ =~ \val~o' \wedge \val~(\fun{output}~i')~=~\var{tta}) \\
    & ~~~~ \datum~o~=~\datum~ (\fun{output}~i) \wedge \datum~o'~=~\datum~ (\fun{output}~i') \\
    & \wedge \\
    & (\var{tta}~ =~ \val~(\fun{output}~i)) \vee (\var{ttb}~ =~ \val~(\fun{output}~i)) \\
    & \where \\
    & ~~~~\var{vi} \leteq \validator~(\fun{output}~i) \\
    & ~~~~\var{tta} \leteq \mathsf{oneT}~(\toggleTT_D~\mathsf{myRef}~\var{vi})~(\fun{encode}~\var{vi}, "a") \\
    & ~~~~\var{ttb} \leteq \mathsf{oneT}~(\toggleTT_D~\mathsf{myRef}~\var{vi})~(\fun{encode}~\var{vi}, "b")
  \end{align*}
% \caption{$\TOGGLE$ validator script for the distributed implementation}
% \label{fig:toggle-val-d}
% \end{ruledfigure}

\paragraph{$\LEDGER$ - $\TOGGLE$ state and transition relations}

Here we first encounter a situation wherein not every UTxO has a corresponding
$\TOGGLE$ state. In particular, we can only relate UTxO states to toggle states
wherein the toggle contract has already been initialized, ie. thread token(s) minted
and stored in a correctly-defined UTxO. We fix $\mathsf{myRef}$ to some specific
output reference. We also instantiate the thread tokens as follows :

\begin{align*}
  \fun{ttt} & \leteq \mathsf{oneT}~(\toggleTT_N~\mathsf{myRef})~(\fun{encode}~(\toggleVal_N~\mathsf{myRef})) \\
  \fun{tta} &\leteq \mathsf{oneT}~(\toggleTT_D~\mathsf{myRef})~(\fun{encode}~(\toggleVal_D~\mathsf{myRef}), "a") \\
  \fun{ttb} &\leteq \mathsf{oneT}~(\toggleTT_D~\mathsf{myRef})~(\fun{encode}~(\toggleVal_D~\mathsf{myRef}), "b")
\end{align*}

\textbf{Naive implementation relations. }

In the naive implementation, a ledger state is related to a $\TOGGLE$ state
whenever (i) the $\TOGGLE$ state is a pair booleans where one is the negation of the
other, and (ii) there is exactly one output on the ledger containing the thread token,
and locked by the correct validator, and with a datum that decodes to the same pair of booleans.

\begin{align*}
  \var{utxo}~\sim~(a,~b) &\leteq \mathsf{myRef}~\notin~\{ ~i~\mid~i\mapsto o \in  \var{utxo}~\} \\
  & \wedge \exists ! ~i\mapsto o~\in~\var{utxo},~\fun{ttt}~=~\val~o~ \\
  & ~~~~ \wedge~~\validator~o~=~\toggleVal_N~\mathsf{myRef}~~\wedge~~\datum~o~=~(a,~b)
\end{align*}

Next, we define the projections :

\begin{align*}
  \pi_{\TOGGLE}~\var{utxo}~ \leteq~\begin{cases}
    (a, b) & \text{ if } \var{utxo}~\sim~(a,~b) \\
    \emptytype & \text{otherwise}
  \end{cases}
\end{align*}

\begin{align*}
  \pi_{\TOGGLE,\Tx}~\var{tx}~ \leteq~\begin{cases}
    \toggle & \text{ if } \exists~i~ \in~\inputs~\var{tx},~\val~(\fun{output}~i) = \fun{ttt} \\
    \emptytype & \text{otherwise}
  \end{cases}
\end{align*}

\textbf{Distributed implementation relation. }

In the distributed implementation, a ledger state is related to a $\TOGGLE$ state
whenever there are exactly two outputs on the ledger each containing one of the
the two thread tokens, $\fun{tta}$ and $\fun{ttb}$,
locked by the correct validator, each with a datum that decodes to the the corresponding
boolean in the $\TOGGLE$ state.

\begin{align*}
  \var{utxo}~\sim~(a,~b) &\leteq \mathsf{myRef}~\notin~\{ ~i~\mid~i\mapsto o \in  \var{utxo}~\} \\
  & \wedge \\
  & \exists ! ~(i\mapsto o,~i'\mapsto o')~\in~\var{utxo},~\fun{tta}~=~\val~o~~\wedge~~\fun{ttb}~=~\val~o'\\
  & ~~~~ \wedge~~\validator~o~=~\toggleVal_D~\mathsf{myRef}~=~\validator~o'~~
  \wedge~~\datum~o~=~a~~\wedge~~\datum~o~=~b
\end{align*}

The function $\pi_{\TOGGLE}$ is defined in the same way as for the naive
implementation, but with $\sim$ for the naive implementation. The transaction
projection is :

\begin{align*}
  \pi_{\TOGGLE,\Tx}~\var{tx}~ \leteq~\begin{cases}
    \toggle & \text{ if } \exists~i,~i'~ \in~\inputs~\var{tx},~\val~(\fun{output}~i) = \fun{tta} ~~\wedge ~~\val~(\fun{output}~i') = \fun{ttb}\\
    \emptytype & \text{otherwise}
  \end{cases}
\end{align*}

For both implementations, we again need to make a variation on the NFT
re-minting protection assumption,

For any $(\var{slot},~\var{utxo},~\var{tx},~\var{utxo'}) \in \LEDGER$, with
$\pi~\var{utxo}~= (a, b)$, necessarily $\var{tx} \neq \fun{fst}~\mathsf{myRef}$.

\paragraph{$\sim >$ relation proof sketch }

Suppose that $(\var{slot},~\var{utxo},~\var{tx},~\var{utxo'})$, $\var{utxo}~\sim~(a,~b)$,
and $\pi~\var{utxo} = (a,~b)$. There are two possibilities, $\pi~\var{tx} = \emptytype$
and $\pi~\var{tx} = \toggle$, for each of which we must prove that
$(\emptytype,~\pi~\var{utxo},~\pi~\var{tx},~\pi~\var{utxo'})$ and $\var{utxo}~\sim~\pi~\var{utxo'}$,
i.e. that $\pi~\var{utxo'} \neq \emptytype$.

We first observe that each of the thread tokens in either implementation
is present in an input of the transaction if and only
if it is present in the output :

$\var{utxo}~\sim~(a,~b)$ implies that the token already exists in the UTxO set and is unique.
The minting policy cannot be satisfied, as the constraint therein that

\[\mathsf{myRef} \in \{ ~\outputref~i~\mid~i \in \inputs~\var{tx}~\} \]

cannot be satisfied by the definition of $\sim~(a,~b)$.
So, since thread tokens cannot be minted or burned, and by the POV rule
\ref{rule:value-is-preserved}, we can make the required conclusion.

\textbf{Naive implementation. }

\begin{itemize}
  \item[(i)] $\pi~\var{tx} = \emptytype$ :
  \[ \neg ~(\exists !~i~ \in~\inputs~\var{tx},~\val~(\fun{output}~i) = \fun{ttt}) \]

  since an additional token $\fun{ttt}$ cannot be minted,

  \[ \neg ~(\exists ~i~ \in~\inputs~\var{tx},~\val~(\fun{output}~i) = \fun{ttt}) \]

  therefore, the (unique) UTxO containing $\fun{ttt}$ remains in $\var{utxo'}$,
  and no additional outputs containing $\fun{ttt}$ appear in $\var{utxo'}$,
  thus guaranteeing the uniqueness of $\fun{ttt}$ in $\var{utxo'}$.
  As shown above, $\mathsf{myRef}$ is also not added to the inputs of $\var{utxo'}$.
  The value, datum, and validator of the UTxO containing $\fun{ttt}$
  are unchanged, so that $\pi~\var{utxo'} = \pi~\var{utxo} \sim (a, b)$.
  Then,

  \[ (\emptytype, \pi~\var{utxo}, \pi~\var{tx}, \pi~\var{utxo'}) = (\emptytype, (a, b), \emptytype, (a, b)) \in \TOGGLE \]

  \item[(i)] $\pi~\var{tx} = \toggle$ :
  \[ \exists !~i~ \in~\inputs~\var{tx},~\val~(\fun{output}~i) = \fun{ttt} \]

  so that the (unique) UTxO containing $\fun{ttt}$ is spent,
  and no $\fun{ttt}$ tokens are minted or burned. Therefore, by POV,
  the transaction must create a single output in $\var{utxo'}$
  with that token. By definition of $\toggleVal_N$, that output must have a toggled datum
  $(b, a)$, no other tokens, and a $\toggleVal_N$ script, so that
  $\pi~\var{utxo'} \sim (b, a)$.
  Again, $\mathsf{myRef}$ is not added to the inputs of $\var{utxo'}$.
  Then,

  \[ (\emptytype, \pi~\var{utxo}, \pi~\var{tx}, \pi~\var{utxo'}) = (\emptytype, (a, b), \emptytype, (b, a)) \in \TOGGLE \]

\end{itemize}

\textbf{Distributed implementation. }

The proof for the distributed implementation is similar to the one for the
naive implementation, except we reason about two inputs and two outputs
containing two thread tokens.

\paragraph{$\TOGGLE$ property example. }
We can prove a property that the $\TOGGLE$ specification upholds, stating that
either booleans are switched, or stay the same :

\[ \forall (a, b) \in \State, (\emptytype, (a,b), i, (c, d)) \in \TOGGLE ~~\Rightarrow~~
(c, d) = (b, a) \vee (c, d) = (a, b) \]

And, because we have demonstrated the $\sim >$ relation, a related property can be
stated for all valid ledger transactions applied to
any state $\var{utxo}$ such that $\var{(a, b)} = \pi ~\var{utxo}$ :

\[ \forall \var{utxo} \in \UTxO, (\wcard, \var{utxo}, \var{tx}, \var{utxo'}) \in \LEDGER ~~\Rightarrow~~
\pi~\var{utxo} = \pi~\var{utxo'} \vee \pi~\var{utxo} = ((\pi~\var{utxo'})_2, (\pi~\var{utxo'})_1) \]

Both properties require a constraint to hold for every state, and therefore
are safety properties \cite{liveness}.

% \section{$\LEDGER$ as a subsystem of a structured contract}
%
% We also show that we are able to define interacting scripts to
% implement this specification in a way that guarantees it is always possible to
% find a define a valid transaction that corresponds to the toggling action.

% Recall the definition of $\STRUC$ being a \emph{subsystem} of $\TRANS$. For
% structured contracts, $\TRANS = \LEDGER$.
% As mentioned above, not all structured contract specification transitions will
% have a corresponding $\LEDGER$ transition.
% Here, we first demonstrate that $\LEDGER$ can be viewed as a subsystem of $\TOGGLE$,
% so that for any $\TOGGLE$ transition, there is a $\LEDGER$ one (with the
% $\sim$ relation satisfied by the appropriate states).
% We then provide examples of specification predicates that depend on the validation
% of opaque scripts, and therefore make it impossible to guarantee their validation.
%
% \subsection{$\LEDGER$ as a $\TOGGLE$ subsystem}
%
% In this section we fix $\STRUC \leteq \LEDGER$, and $\TRANS \leteq \TOGGLE$,
% and demonstrate that
%
% \[ (\LEDGER,~\sim >,~ \sim) \succeq \TOGGLE\]
%
% Let us build a transaction that updates $\TOGGLE$ in the naive implementation.
% The distributed implementation construction can be done via a similar approach.
% Suppose $\var{utxo} \sim (a, b)$.
% We define a $\var{tx}$ for the only non-trivial $\TOGGLE$ state update,
% $(\emptytype, (a, b), \toggle, (b, a))$ :
%
% \begin{align*}
%   \var{tx} &\leteq (\inputs = \{~(i,~o,~\toggle) ~\mid~i\mapsto o \in \var{utxo},~\val~o~=~\fun{ttt}~\},\\
%              & \fun{outputs}~=~[~(\toggleVal_N~\mathsf{myRef},~\fun{ttt},~(b, a))~],\\
%              & \fun{validityInterval}~=~(\mathsf{t},~\mathsf{t+l})\\
%              & \mint~=~ \{\},\\
%              & \mintScsRdmrs~=~\{\},\\
%              & \sigs~=~\{\})\\
% \end{align*}
%
% For some $\mathsf{t,~l} \in \Tick$. We can now prove that
%
% \[ (\emptytype, (a, b), \toggle, (b, a)) ~\sim >~(\var{t},~\var{utxo},~\var{tx}, (\fun{getORefs}~{tx} \subtractdom \var{utxo}) \cup \fun{mkOuts}~{tx}) \in \LEDGER \]
%
% by checking each ledger rule. Most of the proofs are trivial, but we sketch two
% here. To show that the set of inputs is non-empty, rule \ref{rule:has-input} we argue :
% by $\var{utxo} \sim (a, b)$, an output $(\toggleVal_N~\mathsf{myRef},~\fun{ttt},~(b, a))$
% exists in the UTxO set. To show that all inputs validate, rule
% \ref{rule:all-inputs-validate}, we note that $\mathsf{ttt}$ is in the
% inputs of $\var{tx}$, and that the transaction also contains exactly the output
% required by $\toggleVal_N$, which satisfies the constraints of the script.
%
% We observe that
%
% \[ \fun{getORefs}~{tx} \subtractdom \var{utxo}) \cup \fun{mkOuts}~{tx}~ \sim~ (b, a) \]
%
% since there is still exactly one $\fun{ttt}$ token (by the POV),
% and it is in the output $(\toggleVal_N~\mathsf{myRef},~\fun{ttt},~(b, a))$, as
% required by $\sim$.
%
% \subsection{When $\LEDGER$ is not a subsystem of $\STRUC$}
%
% Situations in which there is no transition in $\LEDGER$ that corresponds to a
% transition in $\STRUC$ are common. Here we give two prominent categories of examples :
%
% \paragraph{Pay-ins. } Suppose $\State_{\STRUC} \leteq \Value \times ... $, and
% a constraint on any implementation of the $\STRUC$ specification is that the
% $\Value$ in the state is computed from the actual value found in some particular
% UTxO entry or entries. This is the case for any contract that is expected to accept
% pay-ins and make pay-outs on the ledger, which includes most financial contracts.
%
%   \begin{equation}
%     \inference[PayIn]
%     {~\\
%     x~\leteq~\fun{payIn}~i & ... \\
%     }
%     {
%     \begin{array}{l}
%       \\
%     \end{array}
%       \vdash
%       \left(
%       \begin{array}{r}
%         v,~...
%       \end{array}
%       \right)
%       \trans{Toggle}{i}
%       \left(
%       \begin{array}{r}
%         v~+~x,~...
%       \end{array}
%       \right) \\
%     }
%   \end{equation}
%
% To guarantee that a transition $(\var{slot},~\var{utxo},~\var{tx}, \var{utxo'}) \in \LEDGER$
% exists for a given $(\emptytype, (v, ...),~i,~(v+x,~...))$, we must prove that
% the value $x$ exists in $\var{utxo}$ when $\var{utxo} \sim (v, ...)$. Moreover,
% we need to show that the value $x$ in the UTxO set
% can be spent and used to pay into the contract, and that no transition on the
% ledger $(\var{slot},~\var{utxo},~\var{tx'}, \var{utxo''})$ can spend
% the "pay-in" value in $\var{utxo}$, making a pay-in is no longer possible
% starting in the state $\var{utxo''} \sim (v, ...)$. Proposals for specifying the
% ledger-contract state relation to address this issue is the subject of future work.
%
% \paragraph{Interacting contracts. } Another situation in which it is difficult to
% guarantee a ledger transition corresponding to a specification step is
% in the case of interacting contracts. Suppose that $\STRUC$,
% as a predicate in the rules of its specification, requires that another structured
% contract $\STRUC'$ is updated simultaneously, by, for example, requiring that its thread
% token is included in on the the transaction inputs. The $\STRUC$ specification says nothing
% about the constraints on the update of $\STRUC'$ that it requires to be performed.
%
% Structured contracts are an excellent tool for writing specifications that can be provably
% implemented by several pieces of interdependent code. The interaction of $\STRUC$
% and $\STRUC'$ can likely be given using a single specification, capturing their combined
% state update, using the result below. Note, however, that in the general case,
% determining the full set of scripts need to validate in order for a ledger transaction
% corresponding to a specification step to be valid is a difficult problem, and is
% the subject of future work.
%
% \subsection{Combining structured contracts}
% \label{sec:combining}
%
% An advantage of the structured contract formalism is that,
% given two (possibly interacting) structured contracts $\STRUC_1$ and $\STRUC_2$,
% there is a straightforward way to combine these into a single transition system
% that fully specifies the interaction. That is, as a system without dependency on scripts
% whose validation is required via, e.g. a constraint on the contract input
% necessitating any associated ledger transaction to contrain a particular input
% locked by a particular script.
%
% Let $(\pi_{\STRUC-1},~\STRUC_1) ,~(\pi_{\STRUC'-2},~\STRUC_2) \succeq \LEDGER$ be structured contracts.
% We can then define another structured contract $\STRUC$
%
% \[\STRUC \subseteq \emptytype \times (\State_1 \times \State_2) \times \Tx \times (\State_1 \times \State_2) \]
%
% \[ \STRUC \leteq \{ (\wcard, (s_0, s_1), i, (s'_0, s'_1)) ~\mid~(\wcard, s_0, i, s'_0)~\in~\STRUC_1~\wedge ~(\wcard, s_1, i, s'_1)~\in~\STRUC_2~ \} \]
%
% with $\pi_{\STRUC}~\var{utxo} \leteq (\pi_{\STRUC-1}~\var{utxo},~\pi_{\STRUC-2}~\var{utxo})$.
